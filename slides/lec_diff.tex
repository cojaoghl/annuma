\documentclass{beamer}
\usepackage{amsmath,graphics}
\usepackage{amssymb}

\usetheme{default}
\usepackage{xcolor}

\definecolor{solarizedBase03}{HTML}{002B36}
\definecolor{solarizedBase02}{HTML}{073642}
\definecolor{solarizedBase01}{HTML}{586e75}
\definecolor{solarizedBase00}{HTML}{657b83}
\definecolor{solarizedBase0}{HTML}{839496}
\definecolor{solarizedBase1}{HTML}{93a1a1}
\definecolor{solarizedBase2}{HTML}{EEE8D5}
\definecolor{solarizedBase3}{HTML}{FDF6E3}
\definecolor{solarizedYellow}{HTML}{B58900}
\definecolor{solarizedOrange}{HTML}{CB4B16}
\definecolor{solarizedRed}{HTML}{DC322F}
\definecolor{solarizedMagenta}{HTML}{D33682}
\definecolor{solarizedViolet}{HTML}{6C71C4}
%\definecolor{solarizedBlue}{HTML}{268BD2}
\definecolor{solarizedBlue}{HTML}{134676}
\definecolor{solarizedCyan}{HTML}{2AA198}
\definecolor{solarizedGreen}{HTML}{859900}
\definecolor{myBlue}{HTML}{162DB0}%{261CA4}
\setbeamercolor*{item}{fg=myBlue}
\setbeamercolor{normal text}{fg=solarizedBase03, bg=solarizedBase3}
\setbeamercolor{alerted text}{fg=myBlue}
\setbeamercolor{example text}{fg=myBlue, bg=solarizedBase3}
\setbeamercolor*{frametitle}{fg=solarizedRed}
\setbeamercolor*{title}{fg=solarizedRed}
\setbeamercolor{block title}{fg=myBlue, bg=solarizedBase3}
\setbeameroption{hide notes}
\setbeamertemplate{note page}[plain]
\beamertemplatenavigationsymbolsempty
\usefonttheme{professionalfonts}
\usefonttheme{serif}

\usepackage{fourier}

\def\vec#1{\mathchoice{\mbox{\boldmath$\displaystyle#1$}}
{\mbox{\boldmath$\textstyle#1$}}
{\mbox{\boldmath$\scriptstyle#1$}}
{\mbox{\boldmath$\scriptscriptstyle#1$}}}
\definecolor{OwnGrey}{rgb}{0.560,0.000,0.000} % #999999
\definecolor{OwnBlue}{rgb}{0.121,0.398,0.711} % #1f64b0
\definecolor{red4}{rgb}{0.5,0,0}
\definecolor{blue4}{rgb}{0,0,0.5}
\definecolor{Blue}{rgb}{0,0,0.66}
\definecolor{LightBlue}{rgb}{0.9,0.9,1}
\definecolor{Green}{rgb}{0,0.5,0}
\definecolor{LightGreen}{rgb}{0.9,1,0.9}
\definecolor{Red}{rgb}{0.9,0,0}
\definecolor{LightRed}{rgb}{1,0.9,0.9}
\definecolor{White}{gray}{1}
\definecolor{Black}{gray}{0}
\definecolor{LightGray}{gray}{0.8}
\definecolor{Orange}{rgb}{0.1,0.2,1}
\setbeamerfont{sidebar right}{size=\scriptsize}
\setbeamercolor{sidebar right}{fg=Black}

\renewcommand{\emph}[1]{{\textcolor{solarizedRed}{\itshape #1}}}

\newcommand\cA{\mathcal A}
\newcommand\cB{\mathcal B}
\newcommand\cC{\mathcal C}
\newcommand\cD{\mathcal D}
\newcommand\cE{\mathcal E}
\newcommand\cF{\mathcal F}
\newcommand\cG{\mathcal G}
\newcommand\cH{\mathcal H}
\newcommand\cI{\mathcal I}
\newcommand\cJ{\mathcal J}
\newcommand\cK{\mathcal K}
\newcommand\cL{\mathcal L}
\newcommand\cM{\mathcal M}
\newcommand\cN{\mathcal N}
\newcommand\cO{\mathcal O}
\newcommand\cP{\mathcal P}
\newcommand\cQ{\mathcal Q}
\newcommand\cR{\mathcal R}
\newcommand\cS{\mathcal S}
\newcommand\cT{\mathcal T}
\newcommand\cU{\mathcal U}
\newcommand\cV{\mathcal V}
\newcommand\cW{\mathcal W}
\newcommand\cX{\mathcal X}
\newcommand\cY{\mathcal Y}
\newcommand\cZ{\mathcal Z}

\newcommand\fA{\mathfrak A}
\newcommand\fB{\mathfrak B}
\newcommand\fC{\mathfrak C}
\newcommand\fD{\mathfrak D}
\newcommand\fE{\mathfrak E}
\newcommand\fF{\mathfrak F}
\newcommand\fG{\mathfrak G}
\newcommand\fH{\mathfrak H}
\newcommand\fI{\mathfrak I}
\newcommand\fJ{\mathfrak J}
\newcommand\fK{\mathfrak K}
\newcommand\fL{\mathfrak L}
\newcommand\fM{\mathfrak M}
\newcommand\fN{\mathfrak N}
\newcommand\fO{\mathfrak O}
\newcommand\fP{\mathfrak P}
\newcommand\fQ{\mathfrak Q}
\newcommand\fR{\mathfrak R}
\newcommand\fS{\mathfrak S}
\newcommand\fT{\mathfrak T}
\newcommand\fU{\mathfrak U}
\newcommand\fV{\mathfrak V}
\newcommand\fW{\mathfrak W}
\newcommand\fX{\mathfrak X}
\newcommand\fY{\mathfrak Y}
\newcommand\fZ{\mathfrak Z}

\newcommand\fa{\mathfrak a}
\newcommand\fb{\mathfrak b}
\newcommand\fc{\mathfrak c}
\newcommand\fd{\mathfrak d}
\newcommand\fe{\mathfrak e}
\newcommand\ff{\mathfrak f}
\newcommand\fg{\mathfrak g}
\newcommand\fh{\mathfrak h}
%\newcommand\fi{\mathfrak i}
\newcommand\fj{\mathfrak j}
\newcommand\fk{\mathfrak k}
\newcommand\fl{\mathfrak l}
\newcommand\fm{\mathfrak m}
\newcommand\fn{\mathfrak n}
\newcommand\fo{\mathfrak o}
\newcommand\fp{\mathfrak p}
\newcommand\fq{\mathfrak q}
\newcommand\fr{\mathfrak r}
\newcommand\fs{\mathfrak s}
\newcommand\ft{\mathfrak t}
\newcommand\fu{\mathfrak u}
\newcommand\fv{\mathfrak v}
\newcommand\fw{\mathfrak w}
\newcommand\fx{\mathfrak x}
\newcommand\fy{\mathfrak y}
\newcommand\fz{\mathfrak z}

\newcommand\vA{\vec A}
\newcommand\vB{\vec B}
\newcommand\vC{\vec C}
\newcommand\vD{\vec D}
\newcommand\vE{\vec E}
\newcommand\vF{\vec F}
\newcommand\vG{\vec G}
\newcommand\vH{\vec H}
\newcommand\vI{\vec I}
\newcommand\vJ{\vec J}
\newcommand\vK{\vec K}
\newcommand\vL{\vec L}
\newcommand\vM{\vec M}
\newcommand\vN{\vec N}
\newcommand\vO{\vec O}
\newcommand\vP{\vec P}
\newcommand\vQ{\vec Q}
\newcommand\vR{\vec R}
\newcommand\vS{\vec S}
\newcommand\vT{\vec T}
\newcommand\vU{\vec U}
\newcommand\vV{\vec V}
\newcommand\vW{\vec W}
\newcommand\vX{\vec X}
\newcommand\vY{\vec Y}
\newcommand\vZ{\vec Z}

\newcommand\va{\vec a}
\newcommand\vb{\vec b}
\newcommand\vc{\vec c}
\newcommand\vd{\vec d}
\newcommand\ve{\vec e}
\newcommand\vf{\vec f}
\newcommand\vg{\vec g}
\newcommand\vh{\vec h}
\newcommand\vi{\vec i}
\newcommand\vj{\vec j}
\newcommand\vk{\vec k}
\newcommand\vl{\vec l}
\newcommand\vm{\vec m}
\newcommand\vn{\vec n}
\newcommand\vo{\vec o}
\newcommand\vp{\vec p}
\newcommand\vq{\vec q}
\newcommand\vr{\vec r}
\newcommand\vs{\vec s}
\newcommand\vt{\vec t}
\newcommand\vu{\vec u}
\newcommand\vv{\vec v}
\newcommand\vw{\vec w}
\newcommand\vx{\vec x}
\newcommand\vy{\vec y}
\newcommand\vz{\vec z}

\renewcommand\AA{\mathbb A}
\newcommand\NN{\mathbb N}
\newcommand\ZZ{\mathbb Z}
\newcommand\PP{\mathbb P}
\newcommand\QQ{\mathbb Q}
\newcommand\RR{\mathbb R}
\newcommand\RRpos{\mathbb R_{\geq0}}
\renewcommand\SS{\mathbb S}
\newcommand\CC{\mathbb C}

\newcommand{\ord}{\mathrm{ord}}
\newcommand{\id}{\mathrm{id}}
\newcommand{\pr}{\mathrm{P}}
\newcommand{\Vol}{\mathrm{vol}}
\newcommand\norm[1]{\left\|{#1}\right\|} 
\newcommand\sign{\mathrm{sign}}
\newcommand{\eps}{\varepsilon}
\newcommand{\abs}[1]{\left|#1\right|}
\newcommand\bc[1]{\left({#1}\right)} 
\newcommand\cbc[1]{\left\{{#1}\right\}} 
\newcommand\bcfr[2]{\bc{\frac{#1}{#2}}} 
\newcommand{\bck}[1]{\left\langle{#1}\right\rangle} 
\newcommand\brk[1]{\left\lbrack{#1}\right\rbrack} 
\newcommand\scal[2]{\bck{{#1},{#2}}} 
\newcommand{\vecone}{\mathbb{1}}
\newcommand{\tensor}{\otimes}
\newcommand{\diag}{\mathrm{diag}}
\newcommand{\ggt}{\mathrm{ggT}}
\newcommand{\kgv}{\mathrm{kgV}}
\newcommand{\trans}{\top}

\newcommand{\Karonski}{Karo\'nski}
\newcommand{\Erdos}{Erd\H{o}s}
\newcommand{\Renyi}{R\'enyi}
\newcommand{\Lovasz}{Lov\'asz}
\newcommand{\Juhasz}{Juh\'asz}
\newcommand{\Bollobas}{Bollob\'as}
\newcommand{\Furedi}{F\"uredi}
\newcommand{\Komlos}{Koml\'os}
\newcommand{\Luczak}{\L uczak}
\newcommand{\Kucera}{Ku\v{c}era}
\newcommand{\Szemeredi}{Szemer\'edi}

\renewcommand{\ae}{\"a}
\renewcommand{\oe}{\"o}
\newcommand{\ue}{\"u}
\newcommand{\Ae}{\"A}
\newcommand{\Oe}{\"O}
\newcommand{\Ue}{\"U}

\newcommand{\im}{\mathrm{im}}
\newcommand{\rrk}{\mathrm{zrg}}
\newcommand{\crk}{\mathrm{srg}}
\newcommand{\rk}{\mathrm{rg}}
\newcommand{\GL}{\mathrm{GL}}
\newcommand{\SL}{\mathrm{SL}}
\newcommand{\SO}{\mathrm{SO}}
\newcommand{\nul}{\mathrm{nul}}
\newcommand{\eig}{\mathrm{eig}}

\newcommand{\mytitle}{Die Ableitung}

\title[Annuma]{\mytitle}
\author[Amin Coja-Oghlan]{Amin Coja-Oghlan}
\institute[Frankfurt]{JWGUFFM}
\date{}

\begin{document}

\frame[plain]{\titlepage}

\begin{frame}\frametitle{\mytitle}
	\begin{block}{Worum geht es?}
		\begin{itemize}
			\item Die Ableitung einer Funktion gibt an, wie sich die Funktion an einer Stelle ver\ae ndert.
			\item Stellen Sie sich den Funktionswert $f(t)$ als die Position eines Objektes zur Zeit $t$ vor.
			\item Dann gibt die Ableitung $f'(t)$ die Geschwindigkeit an, mit der sich das Objekt bewegt.
		\end{itemize}
	\end{block}
\end{frame}

\begin{frame}\frametitle{\mytitle}
	\begin{block}{Definition}
		\begin{itemize}
			\item Sei $A\subseteq\RR$ eine Menge.
			\item Wir definieren $\bar A$ als die Menge aller Zahlen $z\in\RR$, so da\ss\ es eine Folge $(a_n)_n$ mit $a_n\in A$ gibt, so da\ss
				\begin{align*}
					z=\lim_{n\to\infty}a_n.
				\end{align*}
			\item Die Menge $\bar A$ hei\ss t der \emph{Abschluss} von $A$.
			\item {\itshape Die Menge $\bar A$ enth\ae lt $A$ als Teilmenge. Au\ss erdem enth\ae lt $\bar A$ die ``Randpunkte'' von $A$.}
		\end{itemize}
	\end{block}
\end{frame}

\begin{frame}\frametitle{\mytitle}
	\begin{block}{Beispiel}
		\begin{itemize}
			\item Seien $a<b$ reelle Zahlen.
			\item Sei $A=(a,b)=\cbc{x\in\RR:a<x<b}$ das offene Intervall von $a$ bis $b$.
			\item Dann ist $\bar A=[a,b]$ das abgeschlossene Intervall.
		\end{itemize}
	\end{block}
\end{frame}

\begin{frame}\frametitle{\mytitle}
	\begin{block}{Beispiel}
		\begin{itemize}
			\item Jede reelle Zahl ist Grenzwert einer Folge rationaler Zahlen.
			\item Daher gilt $\bar\QQ=\RR$.
		\end{itemize}
	\end{block}
\end{frame}

\begin{frame}\frametitle{\mytitle}
	\begin{block}{Definition}
		\begin{itemize}
			\item Sei $f:A\to B$ eine Funktion.
			\item Sei $a\in\bar A$ und $z\in\RR$.
			\item Wir sagen, \emph{$f(x)$ konvergiert gegen $z$ f\ue r $x$ gegen $a$}, wenn f\ue r jede Folge $(a_n)_n$ reeller Zahlen $a_n\in A$ mit $\lim_{n\to\infty}a_n=a$ gilt
				\begin{align*}
					\lim_{n\to\infty}f(a_n)=z.
				\end{align*}
			\item \alert{Sprechweise:} der Grenzwert von $f(x)$ f\ue r $x\to a$ ist gleich $z$.
		\end{itemize}
	\end{block}
\end{frame}

\begin{frame}\frametitle{\mytitle}
	\begin{block}{Beispiel}
		\begin{itemize}
			\item Betrachte die Funktion $f:\RR\to\RR$, $x\mapsto x^2+3x+4$.
			\item F\ue r jede Folge $(a_n)_n$ mit $\lim_{n\to\infty}a_n=0$ gilt nach den Rechenregeln f\ue r Grenzwerte
				\begin{align*}
					\lim_{n\to\infty}f(a_n)=\lim_{n\to\infty}a_n^2+3a_n+4=0^2+3\cdot 0+4=4.
				\end{align*}
		\end{itemize}
	\end{block}
\end{frame}

\begin{frame}\frametitle{\mytitle}
	\begin{block}{Definition}
		\begin{itemize}
			\item Sei $A\subseteq\RR$.
			\item Wir definieren das \emph{Innere von $A$} als die Menge $A^\circ$ aller $x\in\RR$ zu denen eine Zahl $\eps>0$ existiert, so da\ss\
				\begin{align*}
					[x-\eps,x+\eps]\subseteq A.
				\end{align*}
			\item {\itshape $A$ enth\ae lt also ein kleines Intervall von Zahlen um $x$ herum.}
		\end{itemize}
	\end{block}
\end{frame}

\begin{frame}\frametitle{\mytitle}
	\begin{block}{Beispiel}
		\begin{itemize}
			\item Sei $A=[0,1]\cup\cbc{2}$.
			\item Dann ist $A^\circ=(0,1)$ das offene Intervall von $0$ bis $1$.
			\item Denn wenn $x\in(0,1)$, dann gibt es eine Zahl $\eps>0$ mit $[x-\eps,x+\eps]\subseteq A$.
			\item Aber wenn $x\not\in A$ oder $x\in\{0,1,2\}$, dann gibt es keine solche Zahl $\eps>0$.
		\end{itemize}
	\end{block}
\end{frame}

\begin{frame}\frametitle{\mytitle}
	\begin{block}{Beispiel (fortgesetzt)}
		\begin{itemize}
			\item Wir k\oe nnen das Innere und den Abschluss kombinieren:
				\begin{align*}
					\overline{A^\circ}=[0,1].
				\end{align*}
			\item Der isolierte Punkt $2$ geht bei dieser Operation also verloren.
		\end{itemize}
	\end{block}
\end{frame}

\begin{frame}\frametitle{\mytitle}
	\begin{block}{Definition}
		\begin{itemize}
			\item Sei $f:A\to\RR$ eine Funktion.
			\item Sei $z\in\overline{A^\circ}$.
			\item Wir nennen $f$ \emph{differenzierbar in $z$}, wenn der Grenzwert
				\begin{align*}
					\lim_{x\to z}\frac{f(x)-f(z)}{x-z}
				\end{align*}
				existiert.
			\item In diesem Fall hei\ss t dieser Grenzwert die \emph{Ableitung von $f$ in $z$}, geschrieben als
				\begin{align*}
					f'(z)=\lim_{x\to z}\frac{f(x)-f(z)}{x-z}.
				\end{align*}
		\end{itemize}
	\end{block}
\end{frame}

\begin{frame}\frametitle{\mytitle}
	\begin{block}{Definition (fortgesetzt)}
		\begin{itemize}
			\item Genauer definieren wir eine Funktion 
				\begin{align*}
					A\setminus\cbc z\to\RR,\qquad x\mapsto\frac{f(x)-f(z)}{x-z}.
				\end{align*}
			\item Wir bilden den Grenzwert obiger Funktion f\ue r $x\to z$:
				\begin{align*}
					\lim_{x\to z}\frac{f(x)-f(z)}{x-z}
				\end{align*}
		\end{itemize}
	\end{block}
\end{frame}

\begin{frame}\frametitle{\mytitle}
	\begin{block}{Intuition: Geschwindigkeit}
		\begin{itemize}
			\item Sie nehmen an einem Triathlon teil.
			\item Ihre Position entlang der Fahrradstrecke kann durch eine Funktion $$f:[0,T]\to[0,40]$$
				beschrieben werden, die zu jedem Zeitpunkt Ihren Standort entlang der 40km langen Strecke angibt.
			\item Dabei ist $T$ die Gesamtzeit, die Sie f\ue r die Fahrradstrecke ben\oe tigt haben.
			\item Die Strecke ist bergig und naturgem\ae\ss\ fahren Sie bergab schneller als bergauf.
			\item Wie k\oe nnen wir Ihre Geschwindigkeit \alert{zu einem bestimmten Zeitpunkt} messen?
		\end{itemize}
	\end{block}
\end{frame}

\begin{frame}\frametitle{\mytitle}
	\begin{block}{Intuition: Geschwindigkeit}
		\begin{itemize}
			\item Die Funktion $f$ ist beim Triathlon monton wachsend und $f(0)=0$, $f(1)=40$.
			\item Wir k\oe nnen daher Ihre durchschnittliche Geschwindigkeit berechnen als
				\begin{align*}
					\frac{f(1)-f(0)}{T}=\frac{40}{T}.
				\end{align*}
			\item Aber wie steht es um Ihre Geschwindigkeit \alert{zum Zeitpunkt} $t\in[0,T]$?
		\end{itemize}
	\end{block}
\end{frame}

\begin{frame}\frametitle{\mytitle}
	\begin{block}{Intuition: Geschwindigkeit}
		\begin{itemize}
			\item Als N\ae herung k\oe nnten wir die Strecke berechnen, die Sie in 1s zur\ue ckgelegt haben, und diese durch 1s teilen:
				\begin{align*}
					\frac{f(t+1\mathrm s)-f(t)}{1\mathrm s}
				\end{align*}
			\item Eigentlich gibt es keinen guten Grund von $t$ aus ``nach oben'' zu z\ae hlen.
			\item Ebensogut k\oe nnten wir berechnen
				\begin{align*}
				&	\frac{f(t-1\mathrm s)-f(t)}{1\mathrm s}\\
				&	\frac{f(t-0.5\mathrm s)-f(t+0.5\mathrm s)}{1\mathrm s}\\
				&	\frac{f(t-0.25\mathrm s)-f(t+0.75\mathrm s)}{1\mathrm s}\\
				\end{align*}
		\end{itemize}
	\end{block}
\end{frame}

\begin{frame}\frametitle{\mytitle}
	\begin{block}{Intuition: Geschwindigkeit}
		\begin{itemize}
			\item Wenn wir genau genug messen k\oe nnten, spr\ae che auch nichts dagegen, statt 1s ein k\ue rzeres Intervall zu verwenden:
				\begin{align*}
					&\frac{f(t+0.1\mathrm s)-f(t)}{0.1\mathrm s}\\
					&\frac{f(t+0.01\mathrm s)-f(t)}{0.01\mathrm s}\\
					&\qquad\qquad\vdots
				\end{align*}
		\end{itemize}
	\end{block}
\end{frame}

\begin{frame}\frametitle{\mytitle}
	\begin{block}{Intuition: Geschwindigkeit}
		\begin{itemize}
			\item Ultimativ m\oe chten wir also wissen, sie sich Ihre Position in einem \alert{beliebig kurzen} Zeitraum \alert{um den Zeitpunkt $t$ herum} ver\ae ndert.
			\item So kommen wir auf die Definition
				\begin{align*}
					f'(t)&=\mbox{Geschwindigkeit zur Zeit t}\\
						 &=\lim_{s\to t}\frac{f(t)-f(s)}{t-s}
				\end{align*}
		\end{itemize}
	\end{block}
\end{frame}

\begin{frame}\frametitle{\mytitle}
	\begin{block}{Beispiel}
		\begin{itemize}
			\item Die Funktion $f:\RR\to\RR$, $x\mapsto x$ ist in jedem Punkt $z\in\RR$ differenzierbar.
			\item Denn
				\begin{align*}
					\lim_{x\to z}\frac{f(z)-f(x)}{z-x}=\lim_{x\to z}\frac{z-x}{z-x}=\lim_{x\to z}1=1.
				\end{align*}
			\item Also $f'(z)=1$ f\ue r alle $z\in\RR$.
			\item \itshape Stellen Sie sich eine Bewegung mit gleichm\ae ssiger Geschwindigkeit vor.
		\end{itemize}
	\end{block}
\end{frame}

\begin{frame}\frametitle{\mytitle}
	\begin{block}{Beispiel}
		\begin{itemize}
			\item Die Funktion $f:\RR\to\RR$, $x\mapsto |x|$ ist in jedem Punkt $z\in\RR\setminus\cbc 0$ differenzierbar.
			\item Denn 
				\begin{align*}
					f'(z)&=\lim_{x\to z}\frac{z-x}{z-x}=1&&\mbox{f\ue r }z>0\\
					f'(z)&=\lim_{x\to z}\frac{-z-(-x)}{z-x}=-1&&\mbox{f\ue r }z<0
				\end{align*}
		\end{itemize}
	\end{block}
\end{frame}

\begin{frame}\frametitle{\mytitle}
	\begin{block}{Beispiel (fortgesetzt)}
		\begin{itemize}
			\item Jedoch ist die Funktion im Punkt $z=0$ \alert{nicht} differenzierbar.
			\item Denn die Folge $a_n=(-1)^n/n$ konvergiert gegen $0$, aber 
				\begin{align*}
					\frac{f(a_n)-f(0)}{a_n-0}&=\frac{f(a_n)}{a_n}=\frac{1/n}{1/n}=1&&\mbox{wenn $n$ gerade},\\
					\frac{f(a_n)-f(0)}{a_n-0}&=\frac{f(a_n)}{a_n}=\frac{1/n}{-1/n}=-1&&\mbox{wenn $n$ ungerade}.
				\end{align*}
			\item Daher alterniert die Folge $\frac{f(a_n)-f(0)}{a_n-0}$: $-1$,1,$-1$,1,\ldots
			\item Der Grenzwert
				\begin{align*}
					\lim_{n\to\infty}\frac{f(a_n)-f(0)}{a_n-0}
				\end{align*}
				existiert also nicht.
		\end{itemize}
	\end{block}
\end{frame}

\begin{frame}\frametitle{\mytitle}
	\begin{block}{Beispiel}
		\begin{itemize}
			\item Die Funktion $f:\RR\setminus\cbc 0\to\RR$, $x\mapsto 1/x$ ist in jedem Punkt $z\in\RR\setminus\cbc 0$ differenzierbar.
			\item Denn 
				\begin{align*}
					\frac{f(x)-f(z)}{x-z}&=\frac{1/x-1/z}{x-z}=\frac{z-x}{zx(x-z)}=-\frac{1}{zx}.
				\end{align*}
			\item Aus den Rechenregeln f\ue r Grenzwerte folgt also
				\begin{align*}
				\lim_{x\to z}\frac{f(x)-f(z)}{x-z}&=-\lim_{x\to z}\frac{1}{zx}=-\frac{1}{z^2}.
				\end{align*}
			\item Also gilt $f'(z)=-1/z^2$.
		\end{itemize}
	\end{block}
\end{frame}

\begin{frame}\frametitle{\mytitle}
	\begin{block}{Lemma}
		Wenn $f:A\to\RR$ im Punkt $a\in A$ differenzierbar ist, dann ist $f$ in diesem Punkt auch stetig.
	\end{block}
	\begin{block}{Beweis}
	\begin{itemize}
		\item Angenommen $(a_n)_n$ konvergiert gegen $a$.
		\item Wir definieren
			\begin{align*}
				e(x)=f(x)-\bc{f(a)+(x-a)f'(a)}.
			\end{align*}
		\item Die Idee ist, da\ss\ $e(x)$ der ``Fehlerterm'' ist, wenn wir $f$ durch ``Position zum Zeitpunkt $a$ plus Geschwindigkeit zum Zeitpunkt $a$ mal Zeitdifferenz $x-a$'' approximieren:
			\begin{align*}
				f(x)\approx f(a)+(x-a)f'(a).
			\end{align*}
	\end{itemize}
	\end{block}
\end{frame}

\begin{frame}\frametitle{\mytitle}
	\begin{block}{Beweis (fortgesetzt)}
	\begin{itemize}
		\item Weil
			\begin{align*}
				\lim_{n\to\infty}f(a)-(a_n-a)f'(a)=f(a)-f'(a)\lim_{n\to\infty}a_n-a=f(a),
			\end{align*}
			reicht es aus zu zeigen, da\ss
			\begin{align*}
				\lim_{x\to a}e(x)=0.
			\end{align*}
		\item Dazu schreiben wir
			\begin{align*}
				\lim_{x\to a}e(x)&=\lim_{x\to a}\bc{f(x)-f(a)-(x-a)f'(a)}\\
								 &=\lim_{x\to a}\underbrace{\bc{\frac{f(x)-f(a)}{x-a}-f'(a)}}_{\to 0}\underbrace{(x-a)}_{\to 0}=0.
			\end{align*}
	\end{itemize}
	\end{block}
\end{frame}

\begin{frame}\frametitle{\mytitle}
	\begin{block}{Definition}
	\begin{itemize}
		\item Angenommen $f:A\to\RR$ ist in allen Punkten $x\in A$ differenzierbar.
		\item Dann nennen wir $f$ \emph{differenzierbar}.
		\item In diesem Fall hei\ss t die Funktion
			\begin{align*}
				f':A\to\RR,\qquad x\mapsto f'(x)
			\end{align*}
			die \emph{Ableitung} von $f$ auf $A$.
	\end{itemize}
	\end{block}
\end{frame}

\begin{frame}\frametitle{\mytitle}
	\begin{block}{Proposition\hfill[``Linearit\ae t  der Ableitung, Teil 1'']}
	\begin{itemize}
		\item Angenommen $f,g:A\to\RR$ sind differenzierbar in $z$.
		\item Dann ist auch $f+g:A\to\RR$ differenzierbar in $z$ und
			\begin{align*}
				(f+g)'(z)=f'(z)+g'(z).
			\end{align*}
	\end{itemize}
	\end{block}
\end{frame}

\begin{frame}\frametitle{\mytitle}
	\begin{block}{Proposition\hfill[``Linearit\ae t  der Ableitung, Teil 2'']}
	\begin{itemize}
		\item Angenommen $f:A\to\RR$ ist differenzierbar in $z$.
		\item Dann ist f\ue r jede Zahl $c\in\RR$ die Funktion
			\begin{align*}
				c\cdot f:A\to\RR,\qquad x\mapsto c\cdot f(x)
			\end{align*}
			differenzierbar in $z$ und
			\begin{align*}
				(c\cdot f)'(z)=c\cdot f'(z).
			\end{align*}
	\end{itemize}
	\end{block}
\end{frame}

\begin{frame}\frametitle{\mytitle}
	\begin{block}{Proposition\hfill[``Produktregel'']}
	\begin{itemize}
		\item Angenommen $f,g:A\to\RR$ sind differenzierbar in $z$.
		\item Dann ist auch $f\cdot g:A\to\RR$ differenzierbar in $z$ und
			\begin{align*}
				(f\cdot g)'(z)=f'(z)\cdot g(z)+f(z)\cdot g'(z).
			\end{align*}
	\end{itemize}
	\end{block}
\end{frame}

\begin{frame}\frametitle{\mytitle}
	\begin{block}{Korollar}
		Jedes Polynom
		\begin{align*}
			f:\RR\to\RR,\qquad x\mapsto \sum_{i=0}^ka_ix^i
		\end{align*}
		ist differenzierbar und
		\begin{align*}
			f'(z)&=\sum_{i=1}^k ia_ix^{i-1}.
		\end{align*}
	\end{block}
\end{frame}

\begin{frame}\frametitle{\mytitle}
	\begin{block}{Beispiel}
	\begin{itemize}
		\item Die konstante Funktion $f(x)=7$ hat die Ableitung
		\begin{align*}
			f'(z)&=0.
		\end{align*}
	\item Die affine Funktion $ g(x)=-5x+7 $ hat die Ableitung
		\begin{align*}
			g'(x)&=-5
		\end{align*}
	\item Die quadratische Funktion $ h(x)=3x^2-5x+7 $ hat die Ableitung
		\begin{align*}
			h'(x)&=6x-5.
		\end{align*}
	\end{itemize}
	\end{block}
\end{frame}

\begin{frame}\frametitle{\mytitle}
	\begin{block}{Kettenregel}
		Angenommen $f:A\to B$ ist differenzierbar im Punkt $z\in A$ und $g:B\to\RR$ ist differenzierbar im Punkt $f(z)$. 
		Dann ist $g\circ f:A\to\RR$ differenzierbar im Punkt $z$ und
		\begin{align*}
			(g\circ f)'(z)=g'(f(z))f'(z).
		\end{align*}
	\end{block}
\end{frame}

\begin{frame}\frametitle{\mytitle}
	\begin{block}{Quotientenregel}
		Angenommen $f,g:A\to\RR$ sind im Punkt $z\in A$ differenzierbar und $g(x)\neq0$ f\ue r alle $x\in A$.
		Dann ist auch die Funktion
		\begin{align*}
			\frac{f}{g}:A\to\RR,\qquad x\mapsto\frac{f(x)}{g(x)}
		\end{align*}
		im Punkt $z$ differenzierbar und
		\begin{align*}
			\bcfr fg'(z)&=\frac{f'(z)g(z)-f(z)g'(z)}{g(z)^2}.
		\end{align*}
	\end{block}
\end{frame}

\begin{frame}\frametitle{\mytitle}
	\begin{block}{Mittelwertsatz der Differentialrechnung}
	\begin{itemize}
		\item Angenommen $a<b$ sind reelle Zahlen.
		\item Sei $f:[a,b]\to\RR$ eine stetige Funktion, die in jedem Punkt $z\in(a,b)$ differenzierbar ist.
		\item Dann gibt es einen Punkt $c\in(a,b)$ mit
			\begin{align*}
				f'(c)&=\frac{f(b)-f(a)}{b-a}.
			\end{align*}
		\item {\itshape Intuition: zwischen dem Beginn $a$ der Reise und ihrem Ende $b$ gibt es irgendeinen Zeitpunkt $c$, zu dem die aktuelle Geschwindigkeit genau gleich der durchschnittlichen Reisegeschwindigkeit $(f(b)-f(a))/(b-a)$ ist.}
	\end{itemize}
	\end{block}
\end{frame}

\begin{frame}\frametitle{\mytitle}
	\begin{block}{Definition}
	\begin{itemize}
		\item Eine Funktion $f:A\to\RR$ hei\ss t \emph{monoton wachsend}, wenn $f(a)\leq f(b)$ f\ue r alle $a,b\in A$ mit $a\leq b$.
		\item Die Funktion hei\ss t \emph{monoton fallend}, wenn $f(a)\geq f(b)$ f\ue r alle $a,b\in A$ mit $a\leq b$.
		\item Die Funktion hei\ss t \emph{konstant}, wenn sie sowohl monoton wachsend als auch monoton fallend ist.
	\end{itemize}
	\end{block}
\end{frame}

\begin{frame}\frametitle{\mytitle}
	\begin{block}{Proposition}
	\begin{itemize}
		\item Sei $f:[a,b]\to\RR$ eine stetige Funktion, die in jedem Punkt $z\in(a,b)$ differenzierbar ist.
		\item $f$ ist genau dann monoton wachsend, wenn $f'(z)\geq0$ f\ue r alle $z\in(a,b)$.
		\item $f$ ist genau dann monoton fallend, wenn $f'(z)\leq0$ f\ue r alle $z\in(a,b)$.
	\end{itemize}
	\end{block}
\end{frame}

\begin{frame}\frametitle{\mytitle}
	\begin{block}{Definition}
		Eine Funktion $f:A\to\RR$ hei\ss t \emph{stetig differenzierbar}, wenn sie differenzierbar ist und die Ableitung $f':A\to\RR$ stetig ist.
	\end{block}
\begin{block}{Beispiel}
		Weil Polynome als Ableitungen Polynome haben, sind Polynome stetig differenzierbar.
	\end{block}
\end{frame}

\begin{frame}\frametitle{\mytitle}
	\begin{block}{Satz von der Umkehrfunktion}
	\begin{itemize}
		\item Angenommen $f:(a,b)\to B$ ist stetig, bijektiv und im Punkt $z\in(a,b)$ differenzierbar mit
			\begin{align*}
				f'(z)\neq0.
			\end{align*}
		\item Dann ist die Umkehrfunktion $f^{-1}:B\to(a,b)$ im Punkt $f(z)$ differenzierbar und
		\begin{align*}
			(f^{-1})'(f(z))&=\frac{1}{f'(z)}.
		\end{align*}
	\end{itemize}
	\end{block}
\end{frame}

\begin{frame}\frametitle{\mytitle}
	\begin{block}{Beispiel}
	\begin{itemize}
		\item Die Funktion $f:(0,\infty)\to(0,\infty)$, $x\mapsto x^2$ ist ein Polynom und daher differenzierbar.
		\item Die Ableitung ist $f'(z)=2z>0$.
		\item Daher ist auch die Umkehrfunktion
			\begin{align*}
				f^{-1}:(0,\infty)\to(0,\infty),\qquad x\mapsto\sqrt x
			\end{align*}
			differenzierbar.
		\item Ihre Ableitung ist
			\begin{align*}
				{f^{-1}}'(z^2)={f^{-1}}'(f(z))=\frac{1}{f'(z)}=\frac{1}{2z}.
			\end{align*}
		\item Setzen wir $z=\sqrt y$, so erhalten wir $ {f^{-1}}'(y)=\frac{1}{2\sqrt y}.  $
	\end{itemize}
	\end{block}
\end{frame}

\begin{frame}\frametitle{\mytitle}
	\begin{block}{Zusammenfassung}
	\begin{itemize}
		\item Die Ableitung einer Funktion $f$ im Punkt $z$ gibt die Geschwindigkeit an, mit der sich $z$ im Punkt $z$ ver\ae ndert.
		\item Wir haben die Rechenregeln f\ue r die Ableitung kennengelernt: Linearit\ae t, Produktregel, Kettenregel, Quotientenregel, Satz von der Umkehrfunktion.
	\end{itemize}
	\end{block}
\end{frame}

\end{document}
