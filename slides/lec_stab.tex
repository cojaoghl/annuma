\documentclass{beamer}
\usepackage{amsmath,graphics}
\usepackage{amssymb}

\usetheme{default}
\usepackage{xcolor}

\definecolor{solarizedBase03}{HTML}{002B36}
\definecolor{solarizedBase02}{HTML}{073642}
\definecolor{solarizedBase01}{HTML}{586e75}
\definecolor{solarizedBase00}{HTML}{657b83}
\definecolor{solarizedBase0}{HTML}{839496}
\definecolor{solarizedBase1}{HTML}{93a1a1}
\definecolor{solarizedBase2}{HTML}{EEE8D5}
\definecolor{solarizedBase3}{HTML}{FDF6E3}
\definecolor{solarizedYellow}{HTML}{B58900}
\definecolor{solarizedOrange}{HTML}{CB4B16}
\definecolor{solarizedRed}{HTML}{DC322F}
\definecolor{solarizedMagenta}{HTML}{D33682}
\definecolor{solarizedViolet}{HTML}{6C71C4}
%\definecolor{solarizedBlue}{HTML}{268BD2}
\definecolor{solarizedBlue}{HTML}{134676}
\definecolor{solarizedCyan}{HTML}{2AA198}
\definecolor{solarizedGreen}{HTML}{859900}
\definecolor{myBlue}{HTML}{162DB0}%{261CA4}
\setbeamercolor*{item}{fg=myBlue}
\setbeamercolor{normal text}{fg=solarizedBase03, bg=solarizedBase3}
\setbeamercolor{alerted text}{fg=myBlue}
\setbeamercolor{example text}{fg=myBlue, bg=solarizedBase3}
\setbeamercolor*{frametitle}{fg=solarizedRed}
\setbeamercolor*{title}{fg=solarizedRed}
\setbeamercolor{block title}{fg=myBlue, bg=solarizedBase3}
\setbeameroption{hide notes}
\setbeamertemplate{note page}[plain]
\beamertemplatenavigationsymbolsempty
\usefonttheme{professionalfonts}
\usefonttheme{serif}

\usepackage{fourier}

\def\vec#1{\mathchoice{\mbox{\boldmath$\displaystyle#1$}}
{\mbox{\boldmath$\textstyle#1$}}
{\mbox{\boldmath$\scriptstyle#1$}}
{\mbox{\boldmath$\scriptscriptstyle#1$}}}
\definecolor{OwnGrey}{rgb}{0.560,0.000,0.000} % #999999
\definecolor{OwnBlue}{rgb}{0.121,0.398,0.711} % #1f64b0
\definecolor{red4}{rgb}{0.5,0,0}
\definecolor{blue4}{rgb}{0,0,0.5}
\definecolor{Blue}{rgb}{0,0,0.66}
\definecolor{LightBlue}{rgb}{0.9,0.9,1}
\definecolor{Green}{rgb}{0,0.5,0}
\definecolor{LightGreen}{rgb}{0.9,1,0.9}
\definecolor{Red}{rgb}{0.9,0,0}
\definecolor{LightRed}{rgb}{1,0.9,0.9}
\definecolor{White}{gray}{1}
\definecolor{Black}{gray}{0}
\definecolor{LightGray}{gray}{0.8}
\definecolor{Orange}{rgb}{0.1,0.2,1}
\setbeamerfont{sidebar right}{size=\scriptsize}
\setbeamercolor{sidebar right}{fg=Black}

\renewcommand{\emph}[1]{{\textcolor{solarizedRed}{\itshape #1}}}

\newcommand\nan{{\tt nan}}
\newcommand\tay{T}
\newcommand\dd{\mathrm d}
\newcommand\eul{\mathrm e}

\newcommand\cA{\mathcal A}
\newcommand\cB{\mathcal B}
\newcommand\cC{\mathcal C}
\newcommand\cD{\mathcal D}
\newcommand\cE{\mathcal E}
\newcommand\cF{\mathcal F}
\newcommand\cG{\mathcal G}
\newcommand\cH{\mathcal H}
\newcommand\cI{\mathcal I}
\newcommand\cJ{\mathcal J}
\newcommand\cK{\mathcal K}
\newcommand\cL{\mathcal L}
\newcommand\cM{\mathcal M}
\newcommand\cN{\mathcal N}
\newcommand\cO{\mathcal O}
\newcommand\cP{\mathcal P}
\newcommand\cQ{\mathcal Q}
\newcommand\cR{\mathcal R}
\newcommand\cS{\mathcal S}
\newcommand\cT{\mathcal T}
\newcommand\cU{\mathcal U}
\newcommand\cV{\mathcal V}
\newcommand\cW{\mathcal W}
\newcommand\cX{\mathcal X}
\newcommand\cY{\mathcal Y}
\newcommand\cZ{\mathcal Z}

\newcommand\fA{\mathfrak A}
\newcommand\fB{\mathfrak B}
\newcommand\fC{\mathfrak C}
\newcommand\fD{\mathfrak D}
\newcommand\fE{\mathfrak E}
\newcommand\fF{\mathfrak F}
\newcommand\fG{\mathfrak G}
\newcommand\fH{\mathfrak H}
\newcommand\fI{\mathfrak I}
\newcommand\fJ{\mathfrak J}
\newcommand\fK{\mathfrak K}
\newcommand\fL{\mathfrak L}
\newcommand\fM{\mathfrak M}
\newcommand\fN{\mathfrak N}
\newcommand\fO{\mathfrak O}
\newcommand\fP{\mathfrak P}
\newcommand\fQ{\mathfrak Q}
\newcommand\fR{\mathfrak R}
\newcommand\fS{\mathfrak S}
\newcommand\fT{\mathfrak T}
\newcommand\fU{\mathfrak U}
\newcommand\fV{\mathfrak V}
\newcommand\fW{\mathfrak W}
\newcommand\fX{\mathfrak X}
\newcommand\fY{\mathfrak Y}
\newcommand\fZ{\mathfrak Z}

\newcommand\fa{\mathfrak a}
\newcommand\fb{\mathfrak b}
\newcommand\fc{\mathfrak c}
\newcommand\fd{\mathfrak d}
\newcommand\fe{\mathfrak e}
\newcommand\ff{\mathfrak f}
\newcommand\fg{\mathfrak g}
\newcommand\fh{\mathfrak h}
%\newcommand\fi{\mathfrak i}
\newcommand\fj{\mathfrak j}
\newcommand\fk{\mathfrak k}
\newcommand\fl{\mathfrak l}
\newcommand\fm{\mathfrak m}
\newcommand\fn{\mathfrak n}
\newcommand\fo{\mathfrak o}
\newcommand\fp{\mathfrak p}
\newcommand\fq{\mathfrak q}
\newcommand\fr{\mathfrak r}
\newcommand\fs{\mathfrak s}
\newcommand\ft{\mathfrak t}
\newcommand\fu{\mathfrak u}
\newcommand\fv{\mathfrak v}
\newcommand\fw{\mathfrak w}
\newcommand\fx{\mathfrak x}
\newcommand\fy{\mathfrak y}
\newcommand\fz{\mathfrak z}

\newcommand\vA{\vec A}
\newcommand\vB{\vec B}
\newcommand\vC{\vec C}
\newcommand\vD{\vec D}
\newcommand\vE{\vec E}
\newcommand\vF{\vec F}
\newcommand\vG{\vec G}
\newcommand\vH{\vec H}
\newcommand\vI{\vec I}
\newcommand\vJ{\vec J}
\newcommand\vK{\vec K}
\newcommand\vL{\vec L}
\newcommand\vM{\vec M}
\newcommand\vN{\vec N}
\newcommand\vO{\vec O}
\newcommand\vP{\vec P}
\newcommand\vQ{\vec Q}
\newcommand\vR{\vec R}
\newcommand\vS{\vec S}
\newcommand\vT{\vec T}
\newcommand\vU{\vec U}
\newcommand\vV{\vec V}
\newcommand\vW{\vec W}
\newcommand\vX{\vec X}
\newcommand\vY{\vec Y}
\newcommand\vZ{\vec Z}

\newcommand\va{\vec a}
\newcommand\vb{\vec b}
\newcommand\vc{\vec c}
\newcommand\vd{\vec d}
\newcommand\ve{\vec e}
\newcommand\vf{\vec f}
\newcommand\vg{\vec g}
\newcommand\vh{\vec h}
\newcommand\vi{\vec i}
\newcommand\vj{\vec j}
\newcommand\vk{\vec k}
\newcommand\vl{\vec l}
\newcommand\vm{\vec m}
\newcommand\vn{\vec n}
\newcommand\vo{\vec o}
\newcommand\vp{\vec p}
\newcommand\vq{\vec q}
\newcommand\vr{\vec r}
\newcommand\vs{\vec s}
\newcommand\vt{\vec t}
\newcommand\vu{\vec u}
\newcommand\vv{\vec v}
\newcommand\vw{\vec w}
\newcommand\vx{\vec x}
\newcommand\vy{\vec y}
\newcommand\vz{\vec z}

\renewcommand\AA{\mathbb A}
\newcommand\NN{\mathbb N}
\newcommand\ZZ{\mathbb Z}
\newcommand\PP{\mathbb P}
\newcommand\QQ{\mathbb Q}
\newcommand\RR{\mathbb R}
\newcommand\RRpos{\mathbb R_{\geq0}}
\renewcommand\SS{\mathbb S}
\newcommand\CC{\mathbb C}

\newcommand{\ord}{\mathrm{ord}}
\newcommand{\id}{\mathrm{id}}
\newcommand{\pr}{\mathrm{P}}
\newcommand{\Vol}{\mathrm{vol}}
\newcommand\norm[1]{\left\|{#1}\right\|} 
\newcommand\sign{\mathrm{sign}}
\newcommand{\eps}{\varepsilon}
\newcommand{\abs}[1]{\left|#1\right|}
\newcommand\bc[1]{\left({#1}\right)} 
\newcommand\cbc[1]{\left\{{#1}\right\}} 
\newcommand\bcfr[2]{\bc{\frac{#1}{#2}}} 
\newcommand{\bck}[1]{\left\langle{#1}\right\rangle} 
\newcommand\brk[1]{\left\lbrack{#1}\right\rbrack} 
\newcommand\scal[2]{\bck{{#1},{#2}}} 
\newcommand{\vecone}{\mathbb{1}}
\newcommand{\tensor}{\otimes}
\newcommand{\diag}{\mathrm{diag}}
\newcommand{\ggt}{\mathrm{ggT}}
\newcommand{\kgv}{\mathrm{kgV}}
\newcommand{\trans}{\top}

\newcommand{\Karonski}{Karo\'nski}
\newcommand{\Erdos}{Erd\H{o}s}
\newcommand{\Renyi}{R\'enyi}
\newcommand{\Lovasz}{Lov\'asz}
\newcommand{\Juhasz}{Juh\'asz}
\newcommand{\Bollobas}{Bollob\'as}
\newcommand{\Furedi}{F\"uredi}
\newcommand{\Komlos}{Koml\'os}
\newcommand{\Luczak}{\L uczak}
\newcommand{\Kucera}{Ku\v{c}era}
\newcommand{\Szemeredi}{Szemer\'edi}

\renewcommand{\ae}{\"a}
\renewcommand{\oe}{\"o}
\newcommand{\ue}{\"u}
\newcommand{\Ae}{\"A}
\newcommand{\Oe}{\"O}
\newcommand{\Ue}{\"U}

\newcommand{\im}{\mathrm{im}}
\newcommand{\rrk}{\mathrm{zrg}}
\newcommand{\crk}{\mathrm{srg}}
\newcommand{\rk}{\mathrm{rg}}
\newcommand{\GL}{\mathrm{GL}}
\newcommand{\SL}{\mathrm{SL}}
\newcommand{\SO}{\mathrm{SO}}
\newcommand{\nul}{\mathrm{nul}}
\newcommand{\eig}{\mathrm{eig}}

\newcommand{\mytitle}{Stabilit\ae tsanalyse}

\title[Annuma]{\mytitle}
\author[Amin Coja-Oghlan]{Amin Coja-Oghlan}
\institute[Frankfurt]{JWGUFFM}
\date{}

\begin{document}

\frame[plain]{\titlepage}

\begin{frame}\frametitle{\mytitle}
	\begin{block}{Worum geht es?}
		\begin{itemize}
			\item Der Stabilit\ae tsanalyse liegt ein vereinfachendes Fehlermodell f\ue r numerische Berechnungen zugrunde.
			\item Die Zielsetzung ist, Fehler, die durch das Rechnen mit begrenzter Genauigkeit entstehen, in den Griff zu bekommen.
		\end{itemize}
	\end{block}
\end{frame}

\begin{frame}\frametitle{\mytitle}
	\begin{block}{Fehlermodell}
		\begin{itemize}
			\item Wir nehmen an, da\ss\
				\begin{align*}
					x\oplus y&=(x+y)\exp(\eps(x,y))&(x,y\in\QQ),
				\end{align*}
				wobei $|\eps(x,y)|<\eps_*$ f\ue r ein $\eps_*>0$, die \emph{Maschinengenauigkeit}.
			\item Analog nehmen wir an, da\ss\
				\begin{align*}
					x\ominus y&=(x-y)\exp(\eps'(x,y))&(x,y\in\QQ)\\
					x\odot y&=(x\cdot y)\exp(\eps''(x,y))&(x,y\in\QQ)\\
					x\oslash y&=\frac xy\exp(\eps'''(x,y))&(x,y\in\QQ)
				\end{align*}
			\item Ein vern\ue nftiger Wert ist $\eps_*=b^{1-p}$.
			\item \itshape Dieses Fehlermodell ist nur eine grobe Ann\ae herung an die Realit\ae t!
		\end{itemize}
	\end{block}
\end{frame}

\begin{frame}\frametitle{\mytitle}
	\begin{block}{Problemstellung}
		\begin{itemize}
			\item Sei $f(x_1,\ldots,x_n)$ ein vollst\"andig geklammerter algebraischer Ausdruck.
			\item D.h.\ $f$ wendet die Operationen $+$, $\cdot$, $-$, $/$ und Klammern auf $x_1,\ldots,x_n$ an, so da\ss\ durch die Klammern die Reihenfolge der Operationen eindeutig vorgegeben wird.
			\item Wir k\"onnen einen entsprechenden Ausdruck $\hat f(x_1,\ldots,x_n)$ hinschreiben, der die Flie\ss kommaoperationen $\oplus$, $\odot$, $\ominus$, $\oslash$ verwendet.
			\item Ist $\hat f$ eine gute Approximation von $f$?
		\end{itemize}
	\end{block}
\end{frame}

\begin{frame}\frametitle{\mytitle}
	\begin{block}{Definition}
		Der \emph{R\"uckw\"artsfehler} von $(f,x)$ ist definiert als
		\begin{align*}
		\rho(f,x)&=\inf\cbc{\eta>0:\exists \hat x:f(\hat x)=\hat f( x)\wedge\frac{\|\hat x-x\|}{\|x\|}\lesssim\eta\eps_*\mbox{ f\ue r }\eps_*\to0 }.
		\end{align*}
		Der \emph{Vorw\"artsfehler} von $(f,x)$ ist definiert als
		\begin{align*}
			\hat\rho(f,x)&=\inf\cbc{\sigma>0:\frac{\|\hat f(x)-f(x)\|}{\|f(x)\|}\lesssim\sigma\hat\kappa(f,x)\eps_*\mbox{ f\ue r }\eps_*\to0}.
		\end{align*}
	\end{block}
\end{frame}

\begin{frame}\frametitle{\mytitle}
	\begin{block}{Bemerkung}
		\begin{itemize}
			\item In der Regel betrachtet man statt $x$ ferner eine ``approximative'' Eingabe $\tilde x$ und setzt eine Schranke f\"ur die Abweichung $\|x-\tilde x\|$ voraus.
			\item In diesem Fall verlangt man, da\ss\ die obigen Absch\"atzungen f\"ur alle m\"oglichen $\tilde x$ gelten.
			\item Der Einfachheit halber lassen wir das hier bleiben und betrachten nur die punktweisen Fehler.
		\end{itemize}
	\end{block}
\end{frame}

\begin{frame}\frametitle{\mytitle}
	\begin{block}{Beispiel: Skalarprodukt}
		\begin{itemize}
			\item Angenommen wir berechnen f\"ur ein festes $y\in\RR^n$ das Skalarprodukt
				$$f:x\in\RR^n\mapsto\scal xy=\sum_{i=1}^nx_iy_i.$$
			\item Wir verwenden die Rekursionsformel
				\begin{align*}
					\scal xy=x_ny_n+\scal{(x_1,\ldots,x_{n-1})^T}{(y_1,\ldots,y_{n-1})^T}.
				\end{align*}
			\item Was k\"onnen wir \"uber den R\"uckw\"artsfehler aussagen?
		\end{itemize}
	\end{block}
\end{frame}

\begin{frame}\frametitle{\mytitle}
	\begin{block}{Proposition}
Bei obiger Berechnung des Sklarproduktes gilt $\rho(f,x)\leq 4n$.
	\end{block}
	\begin{block}{Beweis}
	\begin{itemize}
	\item Wir f\"uhren Induktion nach $n$.
\item Der Fall $n=1$ folgt unmittelbar aus der Annahme an die Grundoperationen.
\item Sei $n>1$.
\item Sei $x_*$ der Vektor, der aus $x$ entsteht, indem die $n$-te Koordinate weggelassen wird, und definiere $y_*$, $z_*$ entsprechend.
	\end{itemize}	
	\end{block}
\end{frame}

\begin{frame}\frametitle{\mytitle}
	\begin{block}{Beweis (fortgesetzt)}
		\begin{itemize}
			\item Sei nun $n>1$.
			\item Sei $x_*$ der Vektor, der aus $x$ entsteht, indem die $n$-te Koordinate weggelassen wird, und definiere $y_*$, $z_*$ entsprechend.
			\item Nach Induktion gibt es ein $z\in\RR^{n-1}$, so da\ss
				\begin{align*}
					\widehat{\scal{x_*}{y_*} }&=\scal zy\quad\mbox{und}\quad\|z-x_*\|\lesssim4(n-1)\eps_*\|x_*\|.
				\end{align*}
			\item Ferner 
				\begin{align*}
					\widehat{\scal{x}{y} }&=\exp(\eps)(x_ny_n\exp(\eps')+\widehat{\scal{x_*}{y_*} })
				\end{align*}
				mit $|\eps|,|\eps'|\leq\eps_*$.
		\end{itemize}
	\end{block}
\end{frame}

\begin{frame}\frametitle{\mytitle}
	\begin{block}{Beweis (fortgesetzt)}
		\begin{itemize}
			\item Sei
				\begin{align*}
					\hat x_n&=x_n\exp(\eps+\eps'),&\hat x_i&=z_i\exp(\eps)\quad(i=1,\ldots,n-1).
				\end{align*}
			\item Dann erhalten wir
				\begin{align*}
					\widehat{\scal xy}&=\scal{\hat x}y.
				\end{align*}
			\item 
				Ferner folgt aus der Dreiecksungleichung
				\begin{align*}
					\|\hat x-x\|&\leq|\hat x_n-x_n|+\|\exp(\eps)z-x_*\|\\&\leq|\hat x_n-x_n|+\|z-x_*\|+|\exp(\eps)-1|\|z\|\\
								&\lesssim2\eps_*\|x\|+4(n-1)\eps_*\|x\|+\eps_*\|z-x\|+\eps_*\|x\|\\
								&\lesssim2\eps_*\|x\|+4(n-1)\eps_*\|x\|+\eps_*\|x\|\leq4n\eps_*\|x\|,
				\end{align*}
				wie behauptet.
		\end{itemize}
	\end{block}
\end{frame}

\begin{frame}\frametitle{\mytitle}
	\begin{block}{Zusammenfassung}
		\begin{itemize}
			\item Ziel der Stabilit\ae tsanalyse ist die Kontrolle von Rechenfehlern, die durch begrenzte Genauigkeit entstehen.
			\item Es geht also um die Fehler, die der Algorithmus selbst ``produziert''.
			\item Wir haben dazu die Begriffe des Vorw\ae rts- und R\ue ckw\ae rtsfehlers kennengelernt.
		\end{itemize}
	\end{block}
\end{frame}
\end{document}
