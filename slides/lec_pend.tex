\documentclass{beamer}
\usepackage{amsmath,graphics}
\usepackage{amssymb}

\usetheme{default}
\usepackage{xcolor}

\definecolor{solarizedBase03}{HTML}{002B36}
\definecolor{solarizedBase02}{HTML}{073642}
\definecolor{solarizedBase01}{HTML}{586e75}
\definecolor{solarizedBase00}{HTML}{657b83}
\definecolor{solarizedBase0}{HTML}{839496}
\definecolor{solarizedBase1}{HTML}{93a1a1}
\definecolor{solarizedBase2}{HTML}{EEE8D5}
\definecolor{solarizedBase3}{HTML}{FDF6E3}
\definecolor{solarizedYellow}{HTML}{B58900}
\definecolor{solarizedOrange}{HTML}{CB4B16}
\definecolor{solarizedRed}{HTML}{DC322F}
\definecolor{solarizedMagenta}{HTML}{D33682}
\definecolor{solarizedViolet}{HTML}{6C71C4}
%\definecolor{solarizedBlue}{HTML}{268BD2}
\definecolor{solarizedBlue}{HTML}{134676}
\definecolor{solarizedCyan}{HTML}{2AA198}
\definecolor{solarizedGreen}{HTML}{859900}
\definecolor{myBlue}{HTML}{162DB0}%{261CA4}
\setbeamercolor*{item}{fg=myBlue}
\setbeamercolor{normal text}{fg=solarizedBase03, bg=solarizedBase3}
\setbeamercolor{alerted text}{fg=myBlue}
\setbeamercolor{example text}{fg=myBlue, bg=solarizedBase3}
\setbeamercolor*{frametitle}{fg=solarizedRed}
\setbeamercolor*{title}{fg=solarizedRed}
\setbeamercolor{block title}{fg=myBlue, bg=solarizedBase3}
\setbeameroption{hide notes}
\setbeamertemplate{note page}[plain]
\beamertemplatenavigationsymbolsempty
\usefonttheme{professionalfonts}
\usefonttheme{serif}

\usepackage{fourier}

\def\vec#1{\mathchoice{\mbox{\boldmath$\displaystyle#1$}}
{\mbox{\boldmath$\textstyle#1$}}
{\mbox{\boldmath$\scriptstyle#1$}}
{\mbox{\boldmath$\scriptscriptstyle#1$}}}
\definecolor{OwnGrey}{rgb}{0.560,0.000,0.000} % #999999
\definecolor{OwnBlue}{rgb}{0.121,0.398,0.711} % #1f64b0
\definecolor{red4}{rgb}{0.5,0,0}
\definecolor{blue4}{rgb}{0,0,0.5}
\definecolor{Blue}{rgb}{0,0,0.66}
\definecolor{LightBlue}{rgb}{0.9,0.9,1}
\definecolor{Green}{rgb}{0,0.5,0}
\definecolor{LightGreen}{rgb}{0.9,1,0.9}
\definecolor{Red}{rgb}{0.9,0,0}
\definecolor{LightRed}{rgb}{1,0.9,0.9}
\definecolor{White}{gray}{1}
\definecolor{Black}{gray}{0}
\definecolor{LightGray}{gray}{0.8}
\definecolor{Orange}{rgb}{0.1,0.2,1}
\setbeamerfont{sidebar right}{size=\scriptsize}
\setbeamercolor{sidebar right}{fg=Black}

\renewcommand{\emph}[1]{{\textcolor{solarizedRed}{\itshape #1}}}

\newcommand\tay{T}
\newcommand\dd{\mathrm d}
\newcommand\eul{\mathrm e}

\newcommand\cA{\mathcal A}
\newcommand\cB{\mathcal B}
\newcommand\cC{\mathcal C}
\newcommand\cD{\mathcal D}
\newcommand\cE{\mathcal E}
\newcommand\cF{\mathcal F}
\newcommand\cG{\mathcal G}
\newcommand\cH{\mathcal H}
\newcommand\cI{\mathcal I}
\newcommand\cJ{\mathcal J}
\newcommand\cK{\mathcal K}
\newcommand\cL{\mathcal L}
\newcommand\cM{\mathcal M}
\newcommand\cN{\mathcal N}
\newcommand\cO{\mathcal O}
\newcommand\cP{\mathcal P}
\newcommand\cQ{\mathcal Q}
\newcommand\cR{\mathcal R}
\newcommand\cS{\mathcal S}
\newcommand\cT{\mathcal T}
\newcommand\cU{\mathcal U}
\newcommand\cV{\mathcal V}
\newcommand\cW{\mathcal W}
\newcommand\cX{\mathcal X}
\newcommand\cY{\mathcal Y}
\newcommand\cZ{\mathcal Z}

\newcommand\fA{\mathfrak A}
\newcommand\fB{\mathfrak B}
\newcommand\fC{\mathfrak C}
\newcommand\fD{\mathfrak D}
\newcommand\fE{\mathfrak E}
\newcommand\fF{\mathfrak F}
\newcommand\fG{\mathfrak G}
\newcommand\fH{\mathfrak H}
\newcommand\fI{\mathfrak I}
\newcommand\fJ{\mathfrak J}
\newcommand\fK{\mathfrak K}
\newcommand\fL{\mathfrak L}
\newcommand\fM{\mathfrak M}
\newcommand\fN{\mathfrak N}
\newcommand\fO{\mathfrak O}
\newcommand\fP{\mathfrak P}
\newcommand\fQ{\mathfrak Q}
\newcommand\fR{\mathfrak R}
\newcommand\fS{\mathfrak S}
\newcommand\fT{\mathfrak T}
\newcommand\fU{\mathfrak U}
\newcommand\fV{\mathfrak V}
\newcommand\fW{\mathfrak W}
\newcommand\fX{\mathfrak X}
\newcommand\fY{\mathfrak Y}
\newcommand\fZ{\mathfrak Z}

\newcommand\fa{\mathfrak a}
\newcommand\fb{\mathfrak b}
\newcommand\fc{\mathfrak c}
\newcommand\fd{\mathfrak d}
\newcommand\fe{\mathfrak e}
\newcommand\ff{\mathfrak f}
\newcommand\fg{\mathfrak g}
\newcommand\fh{\mathfrak h}
%\newcommand\fi{\mathfrak i}
\newcommand\fj{\mathfrak j}
\newcommand\fk{\mathfrak k}
\newcommand\fl{\mathfrak l}
\newcommand\fm{\mathfrak m}
\newcommand\fn{\mathfrak n}
\newcommand\fo{\mathfrak o}
\newcommand\fp{\mathfrak p}
\newcommand\fq{\mathfrak q}
\newcommand\fr{\mathfrak r}
\newcommand\fs{\mathfrak s}
\newcommand\ft{\mathfrak t}
\newcommand\fu{\mathfrak u}
\newcommand\fv{\mathfrak v}
\newcommand\fw{\mathfrak w}
\newcommand\fx{\mathfrak x}
\newcommand\fy{\mathfrak y}
\newcommand\fz{\mathfrak z}

\newcommand\vA{\vec A}
\newcommand\vB{\vec B}
\newcommand\vC{\vec C}
\newcommand\vD{\vec D}
\newcommand\vE{\vec E}
\newcommand\vF{\vec F}
\newcommand\vG{\vec G}
\newcommand\vH{\vec H}
\newcommand\vI{\vec I}
\newcommand\vJ{\vec J}
\newcommand\vK{\vec K}
\newcommand\vL{\vec L}
\newcommand\vM{\vec M}
\newcommand\vN{\vec N}
\newcommand\vO{\vec O}
\newcommand\vP{\vec P}
\newcommand\vQ{\vec Q}
\newcommand\vR{\vec R}
\newcommand\vS{\vec S}
\newcommand\vT{\vec T}
\newcommand\vU{\vec U}
\newcommand\vV{\vec V}
\newcommand\vW{\vec W}
\newcommand\vX{\vec X}
\newcommand\vY{\vec Y}
\newcommand\vZ{\vec Z}

\newcommand\va{\vec a}
\newcommand\vb{\vec b}
\newcommand\vc{\vec c}
\newcommand\vd{\vec d}
\newcommand\ve{\vec e}
\newcommand\vf{\vec f}
\newcommand\vg{\vec g}
\newcommand\vh{\vec h}
\newcommand\vi{\vec i}
\newcommand\vj{\vec j}
\newcommand\vk{\vec k}
\newcommand\vl{\vec l}
\newcommand\vm{\vec m}
\newcommand\vn{\vec n}
\newcommand\vo{\vec o}
\newcommand\vp{\vec p}
\newcommand\vq{\vec q}
\newcommand\vr{\vec r}
\newcommand\vs{\vec s}
\newcommand\vt{\vec t}
\newcommand\vu{\vec u}
\newcommand\vv{\vec v}
\newcommand\vw{\vec w}
\newcommand\vx{\vec x}
\newcommand\vy{\vec y}
\newcommand\vz{\vec z}

\renewcommand\AA{\mathbb A}
\newcommand\NN{\mathbb N}
\newcommand\ZZ{\mathbb Z}
\newcommand\PP{\mathbb P}
\newcommand\QQ{\mathbb Q}
\newcommand\RR{\mathbb R}
\newcommand\RRpos{\mathbb R_{\geq0}}
\renewcommand\SS{\mathbb S}
\newcommand\CC{\mathbb C}

\newcommand{\ord}{\mathrm{ord}}
\newcommand{\id}{\mathrm{id}}
\newcommand{\pr}{\mathrm{P}}
\newcommand{\Vol}{\mathrm{vol}}
\newcommand\norm[1]{\left\|{#1}\right\|} 
\newcommand\sign{\mathrm{sign}}
\newcommand{\eps}{\varepsilon}
\newcommand{\abs}[1]{\left|#1\right|}
\newcommand\bc[1]{\left({#1}\right)} 
\newcommand\cbc[1]{\left\{{#1}\right\}} 
\newcommand\bcfr[2]{\bc{\frac{#1}{#2}}} 
\newcommand{\bck}[1]{\left\langle{#1}\right\rangle} 
\newcommand\brk[1]{\left\lbrack{#1}\right\rbrack} 
\newcommand\scal[2]{\bck{{#1},{#2}}} 
\newcommand{\vecone}{\mathbb{1}}
\newcommand{\tensor}{\otimes}
\newcommand{\diag}{\mathrm{diag}}
\newcommand{\ggt}{\mathrm{ggT}}
\newcommand{\kgv}{\mathrm{kgV}}
\newcommand{\trans}{\top}

\newcommand{\Karonski}{Karo\'nski}
\newcommand{\Erdos}{Erd\H{o}s}
\newcommand{\Renyi}{R\'enyi}
\newcommand{\Lovasz}{Lov\'asz}
\newcommand{\Juhasz}{Juh\'asz}
\newcommand{\Bollobas}{Bollob\'as}
\newcommand{\Furedi}{F\"uredi}
\newcommand{\Komlos}{Koml\'os}
\newcommand{\Luczak}{\L uczak}
\newcommand{\Kucera}{Ku\v{c}era}
\newcommand{\Szemeredi}{Szemer\'edi}

\renewcommand{\ae}{\"a}
\renewcommand{\oe}{\"o}
\newcommand{\ue}{\"u}
\newcommand{\Ae}{\"A}
\newcommand{\Oe}{\"O}
\newcommand{\Ue}{\"U}

\newcommand{\im}{\mathrm{im}}
\newcommand{\rrk}{\mathrm{zrg}}
\newcommand{\crk}{\mathrm{srg}}
\newcommand{\rk}{\mathrm{rg}}
\newcommand{\GL}{\mathrm{GL}}
\newcommand{\SL}{\mathrm{SL}}
\newcommand{\SO}{\mathrm{SO}}
\newcommand{\nul}{\mathrm{nul}}
\newcommand{\eig}{\mathrm{eig}}

\newcommand{\mytitle}{Anwendung: das Pendel}

\title[Annuma]{\mytitle}
\author[Amin Coja-Oghlan]{Amin Coja-Oghlan}
\institute[Frankfurt]{JWGUFFM}
\date{}

\begin{document}

\frame[plain]{\titlepage}

\begin{frame}\frametitle{\mytitle}
	\begin{block}{Problemstellung}
		\begin{itemize}
			\item Das Pendel ist ein Problem der klassischen Mechanik.
			\item Ein Gegenstand ist an einem masselosen, nicht biegsamen Stab befestigt.
			\item Von einem bestimmten Winkel aus losgelassen ist der Gegenstand der Graviation ausgesetzt.
			\item Der Gegenstand ist eine Punktmasse und bewegt sich in einer zweidimensionalen Ebene.
			\item Wir sehen von Luftwiderstand und Reibung ab.
		\end{itemize}
	\end{block}
\end{frame}

\begin{frame}\frametitle{\mytitle}
	\begin{block}{Differentialgleichung}
		\begin{itemize}
			\item Die Bewegung ist beschrieben durch die Differentialgleichung
				\begin{align*}
					\frac{\dd^2\theta}{\dd t^2}+g\sin\theta&=0
				\end{align*}
			\item Dabei ist $\theta$ der Winkel, den das Pendel mit der Richtung einschlie\ss t, in die die Graviation wirkt.
			\item $t$ ist die Zeit.
			\item $g$ ist die Graviationsbeschleunigung.
			\item \emph{Ziel:} Berechnung der Periode $T$, bis das Pendel nach einer Schwingung wieder zum Ausangspunkt zur\"uckkehrt.
		\end{itemize}
	\end{block}
\end{frame}

\begin{frame}\frametitle{\mytitle}
	\begin{block}{Herleitung der Differentialgleichung}
		\begin{itemize}
			\item Wir verwenden den \emph{Energieerhaltungssatz}.
			\item Gegeben die Masse $m$ des Gegenstands und die Fallh\"ohe $h$ \"andert sich die \alert{potentielle Energie} um $ghm$.
			\item Die \alert{Bewegungsenergie} ver\"andert sich um $mv^2/2$, wenn $v=\frac{\dd\theta}{\dd t}$ die aktuelle Geschwindigkeit ist.
			\item Wir erhalten also 
				\begin{align*}
					ghm=\frac{mv^2}{2}&&\Rightarrow&&\frac{\dd\theta}{\dd t}=\sqrt{2gh}.
				\end{align*}
			\item Sei $\theta_0$ der Anfangswinkel.
			\item Dann gilt f\"ur die Fallh\"ohe
				\begin{align*}
					h&=\cos\theta-\cos\theta_0.
				\end{align*}
		\end{itemize}
	\end{block}
\end{frame}

\begin{frame}\frametitle{\mytitle}
	\begin{block}{Herleitung der Differentialgleichung (fortgesetzt)}
		\begin{itemize}
			\item Wir erhalten also die Gleichung
				\begin{align*}
					\frac{\dd\theta}{\dd t}&=\sqrt{2g\bc{\cos\theta-\cos\theta_0}}.&&(\star)
				\end{align*}
			\item Ableiten nach $t$ ergibt mit der Kettenregel
				\begin{align*}
					\frac{\dd^2\theta}{\dd t^2}&=\frac{\dd}{\dd t}\sqrt{2g\bc{\cos\theta-\cos\theta_0}}
					=-\sqrt{\frac g2}\frac{\sin\theta}{\sqrt{\cos\theta-\cos\theta_0}}\frac{\dd\theta}{\dd t}\\
											   &=-g\sin\theta.
				\end{align*}
		\end{itemize}
	\end{block}
\end{frame}

\begin{frame}\frametitle{\mytitle}
\begin{block}{Die Periode}
	\begin{itemize}
		\item Aus der Gleichung $(\star)$ erhalten wir
			\begin{align*}
				\frac{\dd t}{\dd\theta}=(2g(\cos\theta-\cos\theta_0))^{-\frac{1}{2}}.
			\end{align*}
		\item Wenn wir diese Gleichung \"uber eine komplette Schwingung integrieren, ergibt sich
			\begin{align*}
				T&=4(2g)^{-\frac{1}{2}}\int_0^{\theta_0}\frac{\dd\theta}{\sqrt{\cos\theta-\cos\theta_0}}=\sqrt{\frac{8}{g}}\int_0^{\theta_0}\frac{\dd\theta}{\sqrt{\cos\theta-\cos\theta_0}}
			\end{align*}
		\item Dieses Integral besitzt keine ``einfache'' explizite L\"osung.
	\end{itemize}
\end{block}
\end{frame}

\begin{frame}\frametitle{\mytitle}
	\begin{block}{Die Periode (fortgesetzt)}
	\begin{itemize}
		\item Jedoch ist es mit Hilfe der Legendrepolynome m\"oglich, die L\"osung als Reihe zu formulieren.
		\item Wir erhalten
			\begin{align*}
				T&=2\pi g^{-1/2}\sum_{n=0}^\infty\frac{((2n)!)^2}{16^n(n!)^4}\sin^{2n}\frac{\theta_0}{2}
			\end{align*}
	\end{itemize}
\end{block}
\end{frame}

\begin{frame}\frametitle{\mytitle}
	\begin{block}{Zusammenfassung}
	\begin{itemize}
		\item Das Pendel ist eine scheinbar einfache Fragestellung der klassischen Mechanik.
		\item Wir haben gesehen, wie die L\"osung auf eine Differentialgleichung f\"uhrt.
		\item Die Periode kann durch numerische Integration angen\"ahert werden.
	\end{itemize}
\end{block}
\end{frame}
\end{document}
