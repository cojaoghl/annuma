\documentclass{beamer}
\usepackage{amsmath,graphics}
\usepackage{amssymb}

\usetheme{default}
\usepackage{xcolor}

\definecolor{solarizedBase03}{HTML}{002B36}
\definecolor{solarizedBase02}{HTML}{073642}
\definecolor{solarizedBase01}{HTML}{586e75}
\definecolor{solarizedBase00}{HTML}{657b83}
\definecolor{solarizedBase0}{HTML}{839496}
\definecolor{solarizedBase1}{HTML}{93a1a1}
\definecolor{solarizedBase2}{HTML}{EEE8D5}
\definecolor{solarizedBase3}{HTML}{FDF6E3}
\definecolor{solarizedYellow}{HTML}{B58900}
\definecolor{solarizedOrange}{HTML}{CB4B16}
\definecolor{solarizedRed}{HTML}{DC322F}
\definecolor{solarizedMagenta}{HTML}{D33682}
\definecolor{solarizedViolet}{HTML}{6C71C4}
%\definecolor{solarizedBlue}{HTML}{268BD2}
\definecolor{solarizedBlue}{HTML}{134676}
\definecolor{solarizedCyan}{HTML}{2AA198}
\definecolor{solarizedGreen}{HTML}{859900}
\definecolor{myBlue}{HTML}{162DB0}%{261CA4}
\setbeamercolor*{item}{fg=myBlue}
\setbeamercolor{normal text}{fg=solarizedBase03, bg=solarizedBase3}
\setbeamercolor{alerted text}{fg=myBlue}
\setbeamercolor{example text}{fg=myBlue, bg=solarizedBase3}
\setbeamercolor*{frametitle}{fg=solarizedRed}
\setbeamercolor*{title}{fg=solarizedRed}
\setbeamercolor{block title}{fg=myBlue, bg=solarizedBase3}
\setbeameroption{hide notes}
\setbeamertemplate{note page}[plain]
\beamertemplatenavigationsymbolsempty
\usefonttheme{professionalfonts}
\usefonttheme{serif}

\usepackage{fourier}

\def\vec#1{\mathchoice{\mbox{\boldmath$\displaystyle#1$}}
{\mbox{\boldmath$\textstyle#1$}}
{\mbox{\boldmath$\scriptstyle#1$}}
{\mbox{\boldmath$\scriptscriptstyle#1$}}}
\definecolor{OwnGrey}{rgb}{0.560,0.000,0.000} % #999999
\definecolor{OwnBlue}{rgb}{0.121,0.398,0.711} % #1f64b0
\definecolor{red4}{rgb}{0.5,0,0}
\definecolor{blue4}{rgb}{0,0,0.5}
\definecolor{Blue}{rgb}{0,0,0.66}
\definecolor{LightBlue}{rgb}{0.9,0.9,1}
\definecolor{Green}{rgb}{0,0.5,0}
\definecolor{LightGreen}{rgb}{0.9,1,0.9}
\definecolor{Red}{rgb}{0.9,0,0}
\definecolor{LightRed}{rgb}{1,0.9,0.9}
\definecolor{White}{gray}{1}
\definecolor{Black}{gray}{0}
\definecolor{LightGray}{gray}{0.8}
\definecolor{Orange}{rgb}{0.1,0.2,1}
\setbeamerfont{sidebar right}{size=\scriptsize}
\setbeamercolor{sidebar right}{fg=Black}

\renewcommand{\emph}[1]{{\textcolor{solarizedRed}{\itshape #1}}}

\newcommand\tay{T}
\newcommand\dd{\mathrm d}
\newcommand\eul{\mathrm e}

\newcommand\cA{\mathcal A}
\newcommand\cB{\mathcal B}
\newcommand\cC{\mathcal C}
\newcommand\cD{\mathcal D}
\newcommand\cE{\mathcal E}
\newcommand\cF{\mathcal F}
\newcommand\cG{\mathcal G}
\newcommand\cH{\mathcal H}
\newcommand\cI{\mathcal I}
\newcommand\cJ{\mathcal J}
\newcommand\cK{\mathcal K}
\newcommand\cL{\mathcal L}
\newcommand\cM{\mathcal M}
\newcommand\cN{\mathcal N}
\newcommand\cO{\mathcal O}
\newcommand\cP{\mathcal P}
\newcommand\cQ{\mathcal Q}
\newcommand\cR{\mathcal R}
\newcommand\cS{\mathcal S}
\newcommand\cT{\mathcal T}
\newcommand\cU{\mathcal U}
\newcommand\cV{\mathcal V}
\newcommand\cW{\mathcal W}
\newcommand\cX{\mathcal X}
\newcommand\cY{\mathcal Y}
\newcommand\cZ{\mathcal Z}

\newcommand\fA{\mathfrak A}
\newcommand\fB{\mathfrak B}
\newcommand\fC{\mathfrak C}
\newcommand\fD{\mathfrak D}
\newcommand\fE{\mathfrak E}
\newcommand\fF{\mathfrak F}
\newcommand\fG{\mathfrak G}
\newcommand\fH{\mathfrak H}
\newcommand\fI{\mathfrak I}
\newcommand\fJ{\mathfrak J}
\newcommand\fK{\mathfrak K}
\newcommand\fL{\mathfrak L}
\newcommand\fM{\mathfrak M}
\newcommand\fN{\mathfrak N}
\newcommand\fO{\mathfrak O}
\newcommand\fP{\mathfrak P}
\newcommand\fQ{\mathfrak Q}
\newcommand\fR{\mathfrak R}
\newcommand\fS{\mathfrak S}
\newcommand\fT{\mathfrak T}
\newcommand\fU{\mathfrak U}
\newcommand\fV{\mathfrak V}
\newcommand\fW{\mathfrak W}
\newcommand\fX{\mathfrak X}
\newcommand\fY{\mathfrak Y}
\newcommand\fZ{\mathfrak Z}

\newcommand\fa{\mathfrak a}
\newcommand\fb{\mathfrak b}
\newcommand\fc{\mathfrak c}
\newcommand\fd{\mathfrak d}
\newcommand\fe{\mathfrak e}
\newcommand\ff{\mathfrak f}
\newcommand\fg{\mathfrak g}
\newcommand\fh{\mathfrak h}
%\newcommand\fi{\mathfrak i}
\newcommand\fj{\mathfrak j}
\newcommand\fk{\mathfrak k}
\newcommand\fl{\mathfrak l}
\newcommand\fm{\mathfrak m}
\newcommand\fn{\mathfrak n}
\newcommand\fo{\mathfrak o}
\newcommand\fp{\mathfrak p}
\newcommand\fq{\mathfrak q}
\newcommand\fr{\mathfrak r}
\newcommand\fs{\mathfrak s}
\newcommand\ft{\mathfrak t}
\newcommand\fu{\mathfrak u}
\newcommand\fv{\mathfrak v}
\newcommand\fw{\mathfrak w}
\newcommand\fx{\mathfrak x}
\newcommand\fy{\mathfrak y}
\newcommand\fz{\mathfrak z}

\newcommand\vA{\vec A}
\newcommand\vB{\vec B}
\newcommand\vC{\vec C}
\newcommand\vD{\vec D}
\newcommand\vE{\vec E}
\newcommand\vF{\vec F}
\newcommand\vG{\vec G}
\newcommand\vH{\vec H}
\newcommand\vI{\vec I}
\newcommand\vJ{\vec J}
\newcommand\vK{\vec K}
\newcommand\vL{\vec L}
\newcommand\vM{\vec M}
\newcommand\vN{\vec N}
\newcommand\vO{\vec O}
\newcommand\vP{\vec P}
\newcommand\vQ{\vec Q}
\newcommand\vR{\vec R}
\newcommand\vS{\vec S}
\newcommand\vT{\vec T}
\newcommand\vU{\vec U}
\newcommand\vV{\vec V}
\newcommand\vW{\vec W}
\newcommand\vX{\vec X}
\newcommand\vY{\vec Y}
\newcommand\vZ{\vec Z}

\newcommand\va{\vec a}
\newcommand\vb{\vec b}
\newcommand\vc{\vec c}
\newcommand\vd{\vec d}
\newcommand\ve{\vec e}
\newcommand\vf{\vec f}
\newcommand\vg{\vec g}
\newcommand\vh{\vec h}
\newcommand\vi{\vec i}
\newcommand\vj{\vec j}
\newcommand\vk{\vec k}
\newcommand\vl{\vec l}
\newcommand\vm{\vec m}
\newcommand\vn{\vec n}
\newcommand\vo{\vec o}
\newcommand\vp{\vec p}
\newcommand\vq{\vec q}
\newcommand\vr{\vec r}
\newcommand\vs{\vec s}
\newcommand\vt{\vec t}
\newcommand\vu{\vec u}
\newcommand\vv{\vec v}
\newcommand\vw{\vec w}
\newcommand\vx{\vec x}
\newcommand\vy{\vec y}
\newcommand\vz{\vec z}

\renewcommand\AA{\mathbb A}
\newcommand\NN{\mathbb N}
\newcommand\ZZ{\mathbb Z}
\newcommand\PP{\mathbb P}
\newcommand\QQ{\mathbb Q}
\newcommand\RR{\mathbb R}
\newcommand\RRpos{\mathbb R_{\geq0}}
\renewcommand\SS{\mathbb S}
\newcommand\CC{\mathbb C}

\newcommand{\ord}{\mathrm{ord}}
\newcommand{\id}{\mathrm{id}}
\newcommand{\pr}{\mathrm{P}}
\newcommand{\Vol}{\mathrm{vol}}
\newcommand\norm[1]{\left\|{#1}\right\|} 
\newcommand\sign{\mathrm{sign}}
\newcommand{\eps}{\varepsilon}
\newcommand{\abs}[1]{\left|#1\right|}
\newcommand\bc[1]{\left({#1}\right)} 
\newcommand\cbc[1]{\left\{{#1}\right\}} 
\newcommand\bcfr[2]{\bc{\frac{#1}{#2}}} 
\newcommand{\bck}[1]{\left\langle{#1}\right\rangle} 
\newcommand\brk[1]{\left\lbrack{#1}\right\rbrack} 
\newcommand\scal[2]{\bck{{#1},{#2}}} 
\newcommand{\vecone}{\mathbb{1}}
\newcommand{\tensor}{\otimes}
\newcommand{\diag}{\mathrm{diag}}
\newcommand{\ggt}{\mathrm{ggT}}
\newcommand{\kgv}{\mathrm{kgV}}
\newcommand{\trans}{\top}

\newcommand{\Karonski}{Karo\'nski}
\newcommand{\Erdos}{Erd\H{o}s}
\newcommand{\Renyi}{R\'enyi}
\newcommand{\Lovasz}{Lov\'asz}
\newcommand{\Juhasz}{Juh\'asz}
\newcommand{\Bollobas}{Bollob\'as}
\newcommand{\Furedi}{F\"uredi}
\newcommand{\Komlos}{Koml\'os}
\newcommand{\Luczak}{\L uczak}
\newcommand{\Kucera}{Ku\v{c}era}
\newcommand{\Szemeredi}{Szemer\'edi}

\renewcommand{\ae}{\"a}
\renewcommand{\oe}{\"o}
\newcommand{\ue}{\"u}
\newcommand{\Ae}{\"A}
\newcommand{\Oe}{\"O}
\newcommand{\Ue}{\"U}

\newcommand{\im}{\mathrm{im}}
\newcommand{\rrk}{\mathrm{zrg}}
\newcommand{\crk}{\mathrm{srg}}
\newcommand{\rk}{\mathrm{rg}}
\newcommand{\GL}{\mathrm{GL}}
\newcommand{\SL}{\mathrm{SL}}
\newcommand{\SO}{\mathrm{SO}}
\newcommand{\nul}{\mathrm{nul}}
\newcommand{\eig}{\mathrm{eig}}

\newcommand{\mytitle}{Kontinuierliche Optimierungsprobleme}

\title[Annuma]{\mytitle}
\author[Amin Coja-Oghlan]{Amin Coja-Oghlan}
\institute[Frankfurt]{JWGUFFM}
\date{}

\begin{document}

\frame[plain]{\titlepage}

\begin{frame}\frametitle{\mytitle}
	\begin{block}{Worum geht es?}
		\begin{itemize}
			\item In vielen Anwendungen gilt es, eine Funktion mehrere Ver\ae nderlicher unter Nebenbedingungen zu optimiern.
			\item Dazu lernen wir die Methode der Lagrangemultiplikatoren kennen.
		\end{itemize}
	\end{block}
\end{frame}

\begin{frame}\frametitle{\mytitle}
	\begin{block}{Satz}
	\begin{itemize}
	\item Sei $f:\RR^n\to\RR$ eine Funktion.
	\item Sei $g:\RR^n\to\RR^m$ eine weitere Funktion mit $m<n$.
	\item Angenommen die partiellen Ableitungen
		\begin{align*}
		\frac{\partial f}{\partial x_i},\quad\frac{\partial g_j}{\partial x_i}
		\end{align*}
		existieren und sind stetig.
	\item Sei ferner $\hat x\in\RR^n$ so, da\ss
		\begin{align*}
			f(\hat x)&=\max\cbc{f(x):x\in\RR^n,\quad g(x)=0}.
		\end{align*}
	\item Wenn $\rk Dg(\hat x)=m$, dann existiert ein Vektor $\hat\lambda\in\RR^m$ mit
		\begin{align*}
			Df(\hat x)+\hat\lambda Dg(\hat x)=0.
		\end{align*}
	\end{itemize}
	\end{block}
\end{frame}

\begin{frame}\frametitle{\mytitle}
	\begin{block}{Praktische Umsetzung: eine Nebenbedingung}
	\begin{itemize}
		\item Sei $f:\RR^n\to\RR$ eine Funktion.
		\item Sei $g:\RR^n\to\RR$ eine weitere Funktion
		\item Wir nehmen an, da\ss\ die partiellen Ableitungen existieren und stetig sind.
		\item Wir wollen $f$ maximieren (oder minimieren) unter der \alert{Nebenbedingung} $g(x)=0$.
		\item Dazu stellen wir die \alert{Langrangefunktion} auf:
			\begin{align*}
				L(x_1,\ldots,x_n,\lambda)&=f(x)+\lambda g(x)
			\end{align*}
		\item Diese leiten wir nach $x_1,\ldots,x_n$ und $\lambda$ ab.
		\item Dann inspizieren wir die Nullstellen der Ableitungen.
	\end{itemize}
	\end{block}
\end{frame}

\begin{frame}\frametitle{\mytitle}
	\begin{block}{Beispiel}
	\begin{itemize}
		\item Betrachte die lineare Funktion $f(x,y)=3x+2y$.
		\item Die Nebenbedingung ist gegeben durch die Funktion
			\begin{align*}
				g(x,y)&=x^2+y^2-1.
			\end{align*}
		\item Wir m\oe chten $f(x,y)$ unter der Nebenbedingung $g(x,y)=0$ maximieren:
			\begin{align*}
				\max\cbc{f(x,y):x,y\in\RR\mbox{ und }g(x,y)=0}.
			\end{align*}
		\item \itshape Geometrisch gesprochen ist die Menge aller $x,y$ mit $g(x,y)=0$ der Einheitskreis.
	\end{itemize}
	\end{block}
\end{frame}

\begin{frame}\frametitle{\mytitle}
	\begin{block}{Beispiel}
	\begin{itemize}
		\item Wir stellen dazu die Hilfsfunktion auf:
			\begin{align*}
				L(x,y,\lambda)=f(x,y)+\lambda g(x,y).
			\end{align*}
		\item Ihre Ableitungen sind
			\begin{align*}
				\frac{\partial L}{\partial x}&=\frac{\partial f}{\partial x}+\lambda\frac{\partial g}{\partial x}=3+2\lambda x\\
				\frac{\partial L}{\partial y}&=\frac{\partial f}{\partial y}+\lambda\frac{\partial g}{\partial y}=2+2\lambda y\\
				\frac{\partial L}{\partial \lambda}&=g(x,y)=x^2+y^2-1.
			\end{align*}
	\end{itemize}
	\end{block}
\end{frame}

\begin{frame}\frametitle{\mytitle}
	\begin{block}{Beispiel}
	\begin{itemize}
		\item Wir suchen nach (gemeinsamen) Nullstellen der Ableitungen, d.h.\ nach L\oe sungen des nichtlinearen Gleichungssystems
			\begin{align*}
				3+2\lambda x&=0&2+2\lambda y&=0&x^2+y^2-1=0
			\end{align*}
		\item Wir formen die ersten beiden Gleichungen um zu
			\begin{align*}
				x&=-\frac{3}{2\lambda}&y&=-\frac{1}{\lambda}
			\end{align*}
		\item Einsetzen in die dritte Gleichung liefert
			\begin{align*}
			\frac{9}{4\lambda^2}+\frac{1}{\lambda^2}-1=0
			\end{align*}
		\item Diese Gleichung l\oe sen wir nach $\lambda$ auf.
	\end{itemize}
	\end{block}
\end{frame}

\begin{frame}\frametitle{\mytitle}
	\begin{block}{Beispiel}
	\begin{itemize}
		\item Diese Gleichung l\oe sen wir nach $\lambda$ auf:
			\begin{align*}
				\lambda^2&=\frac{13}{4}&\Rightarrow&&\lambda=\pm\frac{\sqrt{13}}2.
			\end{align*}
		\item Einsetzen f\ue r $x,y$ ergibt
			\begin{align*}
				x&=\frac{3}{\sqrt{13}}&
				y&=\frac{2}{\sqrt{13}}.
			\end{align*}
	\end{itemize}
	\end{block}
\end{frame}

\begin{frame}\frametitle{\mytitle}
	\begin{block}{Praktische Umsetzung: mehrere Nebenbedingungen}
	\begin{itemize}
		\item Wenn $g:\RR^n\to\RR^m$ f\ue r $m>1$, ben\oe tigen wir mehrere Hilfsvariablen $\lambda_1,\ldots,\lambda_m$, sog.\ \alert{Lagrangemultiplikatoren}.
		\item Wir stellen die reellwertige Hilfsfunktion
			\begin{align*}
				L(x_1,\ldots,x_n,\lambda_1,\ldots,\lambda_m)&=f(x)+\sum_{i=1}^m\lambda_ig_i(x)
			\end{align*}
			auf.
		\item Dann bilden wir die Ableitungen
			\begin{align*}
				\frac{\partial L}{\partial x_1},\ldots,\frac{\partial L}{\partial x_n}, 
				\frac{\partial L}{\partial \lambda_1},\ldots,\frac{\partial L}{\partial \lambda_m}
			\end{align*}
			und suchen nach gemeinsamen Nullstellen.
	\end{itemize}
	\end{block}
\end{frame}

\begin{frame}\frametitle{\mytitle}
	\begin{block}{Beispiel}
		\begin{itemize}
			\item Wir maximieren die Funktion 
				\begin{align*}
					f(x,y,z)&=-x\ln x-y\ln y-z\ln z
				\end{align*}
				unter den Nebenbedingungen
				\begin{align*}
					g_1(x,y,z)&=x+y+z-1\\
					g_2(x,y,z)&=x+2y-1
				\end{align*}
			\item Die Hilfsfunktion lautet
				\begin{align*}
					L(x,y,z,\lambda_1,\lambda_2)&=f(x,y,z)+\lambda_1g_1(x,y,z)+\lambda_2g_2(x,y,z)\\
												&=-x\ln x-y\ln y-z\ln z\\&\qquad+\lambda_1(x+y+z-1)+\lambda_2(x+2y-1)
				\end{align*}
		\end{itemize}
	\end{block}
\end{frame}

\begin{frame}\frametitle{\mytitle}
	\begin{block}{Beispiel}
		\begin{itemize}
			\item Ihre Ableitungen lauten
				\begin{align*}
					\frac{\partial L}{\partial x}&=-1-\ln x+\lambda_1+\lambda_2\\
					\frac{\partial L}{\partial y}&=-1-\ln y+\lambda_1+2\lambda_2\\
					\frac{\partial L}{\partial z}&=-1-\ln z+\lambda_1\\
					\frac{\partial L}{\partial \lambda_1}&=x+y+z-1\\
					\frac{\partial L}{\partial \lambda_2}&=x+2y-1
				\end{align*}
			\item Wir suchen nach gemeinsamen Nullstellen der Ableitungen.
		\end{itemize}
	\end{block}
\end{frame}

\begin{frame}\frametitle{\mytitle}
	\begin{block}{Beispiel}
		\begin{itemize}
			\item Dazu formen wir die ersten drei Gleichungen um zu
				\begin{align*}
					x&=\exp(1+\lambda_1+\lambda_2)\\
					y&=\exp(1+\lambda_1+2\lambda_2)\\
					z&=\exp(1+\lambda_1)
				\end{align*}
			\item Einsetzen in die letzten beiden Gleichungen gibt
				\begin{align*}
					\exp(1+\lambda_1+\lambda_2)+\exp(1+\lambda_1+2\lambda_2)+\exp(1+\lambda_1)-1&=0\\
					\exp(1+\lambda_1+\lambda_2)+2\exp(1+\lambda_1+2\lambda_2)-1&=0\\
				\end{align*}
		\end{itemize}
	\end{block}
\end{frame}

\begin{frame}\frametitle{\mytitle}
	\begin{block}{Beispiel}
		\begin{itemize}
			\item Abziehen der beiden Gleichungen voneinander f\ue hrt auf
				\begin{align*}
					\exp(1+\lambda_1)(\exp(2\lambda_2)-1)&=0
				\end{align*}
				also $\lambda_2=0$.
			\item Die erste Gleichung zeigt dann
				\begin{align*}
					\exp(1+\lambda_1)=1/3.
				\end{align*}
			\item Einsetzen f\ue r $x,y,z$ liefert die L\oe sung
				\begin{align*}
				x=y=z=\frac{1}{3}.
				\end{align*}
		\end{itemize}
	\end{block}
\end{frame}

\begin{frame}\frametitle{\mytitle}
	\begin{block}{Zusammenfassung}
		\begin{itemize}
			\item Mit Hilfe von Lagrangemultiplikatoren k\oe nnen wir Optimierungsprobleme unter Nebenbedingungen l\oe sen.
			\item Zum Auffinden der L\oe sung gibt es kein mechanisches Rechenschema.
		\end{itemize}
	\end{block}
\end{frame}

\end{document}
