\documentclass{beamer}
\usepackage{amsmath,graphics}
\usepackage{amssymb}

\usetheme{default}
\usepackage{xcolor}

\definecolor{solarizedBase03}{HTML}{002B36}
\definecolor{solarizedBase02}{HTML}{073642}
\definecolor{solarizedBase01}{HTML}{586e75}
\definecolor{solarizedBase00}{HTML}{657b83}
\definecolor{solarizedBase0}{HTML}{839496}
\definecolor{solarizedBase1}{HTML}{93a1a1}
\definecolor{solarizedBase2}{HTML}{EEE8D5}
\definecolor{solarizedBase3}{HTML}{FDF6E3}
\definecolor{solarizedYellow}{HTML}{B58900}
\definecolor{solarizedOrange}{HTML}{CB4B16}
\definecolor{solarizedRed}{HTML}{DC322F}
\definecolor{solarizedMagenta}{HTML}{D33682}
\definecolor{solarizedViolet}{HTML}{6C71C4}
%\definecolor{solarizedBlue}{HTML}{268BD2}
\definecolor{solarizedBlue}{HTML}{134676}
\definecolor{solarizedCyan}{HTML}{2AA198}
\definecolor{solarizedGreen}{HTML}{859900}
\definecolor{myBlue}{HTML}{162DB0}%{261CA4}
\setbeamercolor*{item}{fg=myBlue}
\setbeamercolor{normal text}{fg=solarizedBase03, bg=solarizedBase3}
\setbeamercolor{alerted text}{fg=myBlue}
\setbeamercolor{example text}{fg=myBlue, bg=solarizedBase3}
\setbeamercolor*{frametitle}{fg=solarizedRed}
\setbeamercolor*{title}{fg=solarizedRed}
\setbeamercolor{block title}{fg=myBlue, bg=solarizedBase3}
\setbeameroption{hide notes}
\setbeamertemplate{note page}[plain]
\beamertemplatenavigationsymbolsempty
\usefonttheme{professionalfonts}
\usefonttheme{serif}

\usepackage{fourier}

\def\vec#1{\mathchoice{\mbox{\boldmath$\displaystyle#1$}}
{\mbox{\boldmath$\textstyle#1$}}
{\mbox{\boldmath$\scriptstyle#1$}}
{\mbox{\boldmath$\scriptscriptstyle#1$}}}
\definecolor{OwnGrey}{rgb}{0.560,0.000,0.000} % #999999
\definecolor{OwnBlue}{rgb}{0.121,0.398,0.711} % #1f64b0
\definecolor{red4}{rgb}{0.5,0,0}
\definecolor{blue4}{rgb}{0,0,0.5}
\definecolor{Blue}{rgb}{0,0,0.66}
\definecolor{LightBlue}{rgb}{0.9,0.9,1}
\definecolor{Green}{rgb}{0,0.5,0}
\definecolor{LightGreen}{rgb}{0.9,1,0.9}
\definecolor{Red}{rgb}{0.9,0,0}
\definecolor{LightRed}{rgb}{1,0.9,0.9}
\definecolor{White}{gray}{1}
\definecolor{Black}{gray}{0}
\definecolor{LightGray}{gray}{0.8}
\definecolor{Orange}{rgb}{0.1,0.2,1}
\setbeamerfont{sidebar right}{size=\scriptsize}
\setbeamercolor{sidebar right}{fg=Black}

\renewcommand{\emph}[1]{{\textcolor{solarizedRed}{\itshape #1}}}

\newcommand\tay{T}
\newcommand\dd{\mathrm d}
\newcommand\eul{\mathrm e}

\newcommand\cA{\mathcal A}
\newcommand\cB{\mathcal B}
\newcommand\cC{\mathcal C}
\newcommand\cD{\mathcal D}
\newcommand\cE{\mathcal E}
\newcommand\cF{\mathcal F}
\newcommand\cG{\mathcal G}
\newcommand\cH{\mathcal H}
\newcommand\cI{\mathcal I}
\newcommand\cJ{\mathcal J}
\newcommand\cK{\mathcal K}
\newcommand\cL{\mathcal L}
\newcommand\cM{\mathcal M}
\newcommand\cN{\mathcal N}
\newcommand\cO{\mathcal O}
\newcommand\cP{\mathcal P}
\newcommand\cQ{\mathcal Q}
\newcommand\cR{\mathcal R}
\newcommand\cS{\mathcal S}
\newcommand\cT{\mathcal T}
\newcommand\cU{\mathcal U}
\newcommand\cV{\mathcal V}
\newcommand\cW{\mathcal W}
\newcommand\cX{\mathcal X}
\newcommand\cY{\mathcal Y}
\newcommand\cZ{\mathcal Z}

\newcommand\fA{\mathfrak A}
\newcommand\fB{\mathfrak B}
\newcommand\fC{\mathfrak C}
\newcommand\fD{\mathfrak D}
\newcommand\fE{\mathfrak E}
\newcommand\fF{\mathfrak F}
\newcommand\fG{\mathfrak G}
\newcommand\fH{\mathfrak H}
\newcommand\fI{\mathfrak I}
\newcommand\fJ{\mathfrak J}
\newcommand\fK{\mathfrak K}
\newcommand\fL{\mathfrak L}
\newcommand\fM{\mathfrak M}
\newcommand\fN{\mathfrak N}
\newcommand\fO{\mathfrak O}
\newcommand\fP{\mathfrak P}
\newcommand\fQ{\mathfrak Q}
\newcommand\fR{\mathfrak R}
\newcommand\fS{\mathfrak S}
\newcommand\fT{\mathfrak T}
\newcommand\fU{\mathfrak U}
\newcommand\fV{\mathfrak V}
\newcommand\fW{\mathfrak W}
\newcommand\fX{\mathfrak X}
\newcommand\fY{\mathfrak Y}
\newcommand\fZ{\mathfrak Z}

\newcommand\fa{\mathfrak a}
\newcommand\fb{\mathfrak b}
\newcommand\fc{\mathfrak c}
\newcommand\fd{\mathfrak d}
\newcommand\fe{\mathfrak e}
\newcommand\ff{\mathfrak f}
\newcommand\fg{\mathfrak g}
\newcommand\fh{\mathfrak h}
%\newcommand\fi{\mathfrak i}
\newcommand\fj{\mathfrak j}
\newcommand\fk{\mathfrak k}
\newcommand\fl{\mathfrak l}
\newcommand\fm{\mathfrak m}
\newcommand\fn{\mathfrak n}
\newcommand\fo{\mathfrak o}
\newcommand\fp{\mathfrak p}
\newcommand\fq{\mathfrak q}
\newcommand\fr{\mathfrak r}
\newcommand\fs{\mathfrak s}
\newcommand\ft{\mathfrak t}
\newcommand\fu{\mathfrak u}
\newcommand\fv{\mathfrak v}
\newcommand\fw{\mathfrak w}
\newcommand\fx{\mathfrak x}
\newcommand\fy{\mathfrak y}
\newcommand\fz{\mathfrak z}

\newcommand\vA{\vec A}
\newcommand\vB{\vec B}
\newcommand\vC{\vec C}
\newcommand\vD{\vec D}
\newcommand\vE{\vec E}
\newcommand\vF{\vec F}
\newcommand\vG{\vec G}
\newcommand\vH{\vec H}
\newcommand\vI{\vec I}
\newcommand\vJ{\vec J}
\newcommand\vK{\vec K}
\newcommand\vL{\vec L}
\newcommand\vM{\vec M}
\newcommand\vN{\vec N}
\newcommand\vO{\vec O}
\newcommand\vP{\vec P}
\newcommand\vQ{\vec Q}
\newcommand\vR{\vec R}
\newcommand\vS{\vec S}
\newcommand\vT{\vec T}
\newcommand\vU{\vec U}
\newcommand\vV{\vec V}
\newcommand\vW{\vec W}
\newcommand\vX{\vec X}
\newcommand\vY{\vec Y}
\newcommand\vZ{\vec Z}

\newcommand\va{\vec a}
\newcommand\vb{\vec b}
\newcommand\vc{\vec c}
\newcommand\vd{\vec d}
\newcommand\ve{\vec e}
\newcommand\vf{\vec f}
\newcommand\vg{\vec g}
\newcommand\vh{\vec h}
\newcommand\vi{\vec i}
\newcommand\vj{\vec j}
\newcommand\vk{\vec k}
\newcommand\vl{\vec l}
\newcommand\vm{\vec m}
\newcommand\vn{\vec n}
\newcommand\vo{\vec o}
\newcommand\vp{\vec p}
\newcommand\vq{\vec q}
\newcommand\vr{\vec r}
\newcommand\vs{\vec s}
\newcommand\vt{\vec t}
\newcommand\vu{\vec u}
\newcommand\vv{\vec v}
\newcommand\vw{\vec w}
\newcommand\vx{\vec x}
\newcommand\vy{\vec y}
\newcommand\vz{\vec z}

\renewcommand\AA{\mathbb A}
\newcommand\NN{\mathbb N}
\newcommand\ZZ{\mathbb Z}
\newcommand\PP{\mathbb P}
\newcommand\QQ{\mathbb Q}
\newcommand\RR{\mathbb R}
\newcommand\RRpos{\mathbb R_{\geq0}}
\renewcommand\SS{\mathbb S}
\newcommand\CC{\mathbb C}

\newcommand{\ord}{\mathrm{ord}}
\newcommand{\id}{\mathrm{id}}
\newcommand{\pr}{\mathrm{P}}
\newcommand{\Vol}{\mathrm{vol}}
\newcommand\norm[1]{\left\|{#1}\right\|} 
\newcommand\sign{\mathrm{sign}}
\newcommand{\eps}{\varepsilon}
\newcommand{\abs}[1]{\left|#1\right|}
\newcommand\bc[1]{\left({#1}\right)} 
\newcommand\cbc[1]{\left\{{#1}\right\}} 
\newcommand\bcfr[2]{\bc{\frac{#1}{#2}}} 
\newcommand{\bck}[1]{\left\langle{#1}\right\rangle} 
\newcommand\brk[1]{\left\lbrack{#1}\right\rbrack} 
\newcommand\scal[2]{\bck{{#1},{#2}}} 
\newcommand{\vecone}{\mathbb{1}}
\newcommand{\tensor}{\otimes}
\newcommand{\diag}{\mathrm{diag}}
\newcommand{\ggt}{\mathrm{ggT}}
\newcommand{\kgv}{\mathrm{kgV}}
\newcommand{\trans}{\top}

\newcommand{\Karonski}{Karo\'nski}
\newcommand{\Erdos}{Erd\H{o}s}
\newcommand{\Renyi}{R\'enyi}
\newcommand{\Lovasz}{Lov\'asz}
\newcommand{\Juhasz}{Juh\'asz}
\newcommand{\Bollobas}{Bollob\'as}
\newcommand{\Furedi}{F\"uredi}
\newcommand{\Komlos}{Koml\'os}
\newcommand{\Luczak}{\L uczak}
\newcommand{\Kucera}{Ku\v{c}era}
\newcommand{\Szemeredi}{Szemer\'edi}

\renewcommand{\ae}{\"a}
\renewcommand{\oe}{\"o}
\newcommand{\ue}{\"u}
\newcommand{\Ae}{\"A}
\newcommand{\Oe}{\"O}
\newcommand{\Ue}{\"U}

\newcommand{\im}{\mathrm{im}}
\newcommand{\rrk}{\mathrm{zrg}}
\newcommand{\crk}{\mathrm{srg}}
\newcommand{\rk}{\mathrm{rg}}
\newcommand{\GL}{\mathrm{GL}}
\newcommand{\SL}{\mathrm{SL}}
\newcommand{\SO}{\mathrm{SO}}
\newcommand{\nul}{\mathrm{nul}}
\newcommand{\eig}{\mathrm{eig}}

\newcommand{\mytitle}{Differentialgleichungen}

\title[Annuma]{\mytitle}
\author[Amin Coja-Oghlan]{Amin Coja-Oghlan}
\institute[Frankfurt]{JWGUFFM}
\date{}

\begin{document}

\frame[plain]{\titlepage}

\begin{frame}\frametitle{\mytitle}
	\begin{block}{Worum geht es?}
		\begin{itemize}
			\item Differentialgleichungen dr\ue cken Zusammenh\ae nge zwischen den Ableitungen einer Funktion aus.
			\item Zum L\oe sen von Differentialgleichungen sind Funktionen zu finden, die diesen Bedingungen gen\ue gen.
			\item Wir befassen uns hier mit der Frage, unter welchen Annahmen L\oe sungen existieren.
			\item Anschlie\ss end lernen wir einige Spezialf\ae lle kennen.
		\end{itemize}
	\end{block}
\end{frame}

\begin{frame}\frametitle{\mytitle}
	\begin{block}{Einfache Differentialgleichungen}
		\begin{itemize}
			\item Sei $A\subseteq\RR^2$ eine Menge und $f:A\to\RR$, $(x,y)\mapsto f(x,y)$ eine stetige Funktion.
			\item Eine \alert{Differentialgleichung} hat die Form
				\begin{align*}
					\phi'(x)&=f(x,\phi(x)).
				\end{align*}
			\item Dabei ist $\phi:I\to\RR$ eine differenzierbare Funktion auf einem Intervall $I$ mit der Eigenschaft, da\ss\ der Graph von $\phi$, also die Menge
				\begin{align*}
					G(\phi)&=\cbc{(x,\phi(x)):x\in I}
				\end{align*}
				eine Teilmenge von $A$ ist.
		\end{itemize}
	\end{block}
\end{frame}

\begin{frame}\frametitle{\mytitle}
	\begin{block}{Einfache Differentialgleichungen}
		\begin{itemize}
			\item Eine \ue bliche wenn auch saloppe Schreibweise ist
				\begin{align*}
					y'&=f(x,y).
				\end{align*}
			\item {\itshape Intuition: die Funktion $f$ gibt an jedem Punkt $\binom xy\in A$ eine Geschwindigkeit vor. Wir suchen eine Bewegung $\phi(x)$, deren Geschwindigkeit zum Zeitpunkt $x$ genau den vorgegebenen Wert $f(x,\phi(x))$ hat.}
		\end{itemize}
	\end{block}
\end{frame}

\begin{frame}\frametitle{\mytitle}
	\begin{block}{Beispiel}
		\begin{itemize}
			\item Wir betrachten die Funktion $f(x,y)=xy$.
			\item Die L\oe sung der Differentialgleichung
				\begin{align*}
				y'=xy
				\end{align*}
				ist die Funktion $y=\exp(x^2/2)$.
			\item Denn nach der Kettenregel gilt
				$$y'=\exp(x^2/2)'=\frac{2x}{2}\exp(x^2/2)=xy.$$
		\end{itemize}
	\end{block}
\end{frame}

\begin{frame}\frametitle{\mytitle}
	\begin{block}{Beispiel}
		\begin{itemize}
			\item Wir betrachten die Funktion $f(x,y)=xy$.
			\item Die L\oe sung der Differentialgleichung
				\begin{align*}
				y'=xy
				\end{align*}
				ist die Funktion $y=\exp(x^2/2)$.
			\item Denn nach der Kettenregel gilt
				$$y'=\exp(x^2/2)'=\frac{2x}{2}\exp(x^2/2)=xy.$$
		\end{itemize}
	\end{block}
\end{frame}

\begin{frame}\frametitle{\mytitle}
	\begin{block}{Systeme von Differentialgleichungen}
		\begin{itemize}
			\item Sei $A\subseteq\RR^{n+1}$ und sei
				\begin{align*}
					f:A\to\RR^n,\quad (x,y_1,\ldots,y_n)\mapsto f(x,y_1,\ldots,y_n)=\begin{pmatrix}f_1(x,y_1,\ldots,y_n)\\\vdots\\f_n(x,y_1,\ldots,y_n)
					\end{pmatrix}
				\end{align*}
				eine Funktion.
			\item Wir fragen nach einer L\oe sung des Systems von Differentialgleichungen
				\begin{align*}
					y_1'&=f_1(x,y_1,\ldots,y_n)\\&\vdots\\y_n'&=f_n(x,y_1,\ldots,y_n)
				\end{align*}
			\item \alert{Kurzschreibweise:} $y'=f(x,y)$.
		\end{itemize}
	\end{block}
\end{frame}

\begin{frame}\frametitle{\mytitle}
	\begin{block}{Systeme von Differentialgleichungen}
		\begin{itemize}
			\item Das bedeutet, wir such eine Funktion $$\phi:I\to\RR^n,\quad x\mapsto\phi(x)=\begin{pmatrix}\phi_1(x)\\\vdots\\\phi_n(x)\end{pmatrix}$$ auf einem Intervall $I$, so da\ss
				\begin{align*}
					G(\phi)&=\cbc{(x,\phi(x)):x\in I}\subseteq A,
				\end{align*}
			\item und so da\ss die einzelnen Funktionen $\phi_1,\ldots,\phi_n$ differenzierbar sind mit Ableitungen
			\begin{align*}
				\phi_i'(x)=f(x,\phi_i(x)).
			\end{align*}
		\end{itemize}
	\end{block}
\end{frame}

\begin{frame}\frametitle{\mytitle}
	\begin{block}{Differentialgleichungen h\oe herer Ordnung}
		\begin{itemize}
			\item Eine \emph{Differentialgleichung $n$-ter Ordnung} ist eine Gleichung der Form
				\begin{align*}
					y^{[n]}=f(x,y^{[1]},\ldots,y^{[n-1]}),
				\end{align*}
				wobei $y=y(x)\in\RR$.
			\item In der Gleichung treten also die ersten $n$ Ableitungen von $y$ auf!
			\item Differentialgleichungen $n$-ter Ordnung k\oe nnen auf Systeme von Differentialgleichungen zur\ue ckgef\ue hrt werden.
		\end{itemize}
	\end{block}
\end{frame}

\begin{frame}\frametitle{\mytitle}
	\begin{block}{Definition}
		\begin{itemize}
			\item Eine Menge $A\subseteq\RR^n$ hei\ss t \emph{offen}, wenn es zu jedem Punkt $a\in A$ eine Zahl $\eps>0$ gibt, so da\ss\
				\begin{align*}
					\cbc{y\in\RR^n:\|x-y\|<\eps}\subseteq A.
				\end{align*}
			\item \alert{Erinnerung:} $$\|z\|=\sqrt{\sum_{i=1}^nz_i^2}.$$
		\end{itemize}
	\end{block}
\end{frame}

\begin{frame}\frametitle{\mytitle}
	\begin{block}{Definition}
		\begin{itemize}
			\item Sei $A\subseteq\RR^n$.
			\item Eine Funktion $f:A\to\RR^m$ hei\ss t \emph{lokal Lipschitz-stetig}, wenn es zu jedem $a\in A$ Zahlen $\eps>0$ und $L>0$ gibt, so da\ss\ f\ue r alle $b\in A$ mit $\|a-b\|<\eps$ gilt
				\begin{align*}
					\|f(a)-f(b)\|\leq L\|a-b\|.
				\end{align*}
		\end{itemize}
	\end{block}
\end{frame}

\begin{frame}\frametitle{\mytitle}
	\begin{block}{Proposition}
		\begin{itemize}
			\item Sei $A\subseteq\RR^n$ offen und $f:A\to\RR^m$ eine Funktion, deren partielle Ableitungen existieren und stetig sind.
			\item Dann ist $f$ lokal Lipschitz-stetig.
		\end{itemize}
	\end{block}
\end{frame}

\begin{frame}\frametitle{\mytitle}
	\begin{block}{Eindeutigkeitssatz}
		\begin{itemize}
			\item Sei $A\subseteq\RR^{n+1}$ offen.
			\item Sei $f:A\to\RR^n$ lokal Lipschitz-stetig.
			\item Wenn $\phi,\psi:I\to\RR^n$ zwei L\oe sungen des Differentialgleichungssystems
				\begin{align*}
					y'=f(x,y)
				\end{align*}
				sind,
			\item und wenn es einen Punkt $z\in I$ mit $\phi(z)=\psi(z)$ gibt,
			\item dann gilt $\phi(x)=\psi(x)$ f\ue r alle $x\in I$.
		\end{itemize}
	\end{block}
\end{frame}

\begin{frame}\frametitle{\mytitle}
	\begin{block}{Existenzsatz}
		\begin{itemize}
			\item Sei $A\subset\RR^{n+1}$ offen.
			\item Sei $f:A\to\RR^n$ lokal Lipschitz-stetig.
			\item Dann existiert zu jedem Punkt $(x_0,y_0)\in A$ eine Zahl $\eps>0$ und eine Funktion $\phi:(x_0-\eps,x_0+eps)\to\RR^n$, so da\ss
				\begin{align*}
					\phi(x_0)&=y_0\\
					\phi_i'(x)&=f_i(x,\phi(x))&&\mbox{f\ue r alle }x\in(x_0-\eps,x_0+\eps).
				\end{align*}
		\end{itemize}
	\end{block}
\end{frame}

\begin{frame}\frametitle{\mytitle}
	\begin{block}{Existenzsatz}
		\begin{itemize}
			\item Sei $A\subset\RR^{n+1}$ offen.
			\item Sei $f:A\to\RR^n$ lokal Lipschitz-stetig.
			\item Dann existiert zu jedem Punkt $(x_0,y_0)\in A$ eine Zahl $\eps>0$ und eine Funktion $\phi:(x_0-\eps,x_0+\eps)\to\RR^n$, so da\ss
				\begin{align*}
					\phi(x_0)&=y_0\\
					\phi_i'(x)&=f_i(x,\phi(x))&&\mbox{f\ue r alle }x\in(x_0-\eps,x_0+\eps).
				\end{align*}
		\end{itemize}
	\end{block}
\end{frame}

\begin{frame}\frametitle{\mytitle}
	\begin{block}{Beispiel: L\oe sung durch Integrieren}
		\begin{itemize}
			\item Wir kommen zur\ue ck auf unsere erste Differentialgleichung
				\begin{align*}
				y'=2xy
				\end{align*}
			\item Wir geben eine \alert{Anfangsbedingung} $\phi_0(0)=y_0$ vor.
			\item Davon aus gehend konstruieren wir eine Folge von Funktion
				\begin{align*}
					\phi_{\ell+1}(x)&=y_0+2\int_0^xz\phi_\ell(z)\dd z
				\end{align*}
			\item Ausgehend von $\phi_0$ berechnen wir
				\begin{align*}
					\phi_1(x)&=y_0+2\int_0^x\phi_0(z)z\dd z=y_0+y_0\int_0^xz\dd z=y_0(1+x^2)\\
					\phi_2(x)&=y_0+2\int_0^x\phi_1(z)z\dd z=y_0(1+x^2+x^4/2)
				\end{align*}
		\end{itemize}
	\end{block}
\end{frame}

\begin{frame}\frametitle{\mytitle}
	\begin{block}{Beispiel: L\oe sung durch Integrieren}
		\begin{itemize}
			\item Allgemein erhalten wir
				\begin{align*}
					\phi_\ell(x)&=y_0\sum_{i=0}^\ell\frac{x^{2i}}{i!}
				\end{align*}
			\item Bilden wir den Grenzwert $\ell\to\infty$, so erhalten wir
				\begin{align*}
					\phi(x)&=\lim_{\ell\to\infty}\phi_\ell(x)=y_0\exp(x^2).
				\end{align*}
		\end{itemize}
	\end{block}
\end{frame}

\begin{frame}\frametitle{\mytitle}
	\begin{block}{Beispiel: Lineare Differentialgleichungen}
		\begin{itemize}
			\item Wir betrachten die \emph{homogene lineare Differentialgleichung}
				\begin{align*}
					y'=\alpha(x)y.
				\end{align*}
			\item Auch diese Differentialgleichung k\oe nnen wir durch Integration l\oe sen:
				\begin{align*}
					\phi(x)&=y_0\exp\bc{\int_{x_0}^x\alpha(z)\dd z}.
				\end{align*}
		\end{itemize}
	\end{block}
\end{frame}

\begin{frame}\frametitle{\mytitle}
	\begin{block}{Beispiel: inhomogene lineare Differentialgleichungen}
		\begin{itemize}
			\item Etwas komplizierter ist der inhomogene Fall
				\begin{align*}
					y'=\alpha(x)y+\beta(x).
				\end{align*}
			\item Wir betrachten dazu die homogene L\oe sung
				\begin{align*}
					\phi(x)&=y_0\exp\bc{\int_{x_0}^x\alpha(z)\dd z}
				\end{align*}
				und nehmen an, da\ss\ $\phi(x_0)\neq0$.
			\item Wir definieren nun die Funktion
				\begin{align*}
					\psi(x)=\int_{x_0}^x\frac{\beta(z)}{\phi(z)}\dd z+c
				\end{align*}
				f\ue r eine feste Zahl $c\in\RR$.
		\end{itemize}
	\end{block}
\end{frame}

\begin{frame}\frametitle{\mytitle}
	\begin{block}{Beispiel: inhomogene lineare Differentialgleichungen}
		\begin{itemize}
			\item Dann erhalten wir die L\oe sung
				\begin{align*}
					\varphi(x)=\phi(x)\psi(x).
				\end{align*}
			\item Durch geeignete Wahl von $c$ l\ae\ss t sich die gew\ue nschte Anfangsbedingung einstellen.
		\end{itemize}
	\end{block}
\end{frame}

\begin{frame}\frametitle{\mytitle}
	\begin{block}{Zusammenfassung}
		\begin{itemize}
			\item Gew\oe hnliche Differentialgleichungen treten in Anwendungen in der Physik, Elektrotechnik und Informatik auf.
			\item Es gibt kein allgemeines Rechenschema zum L\oe sen solcher Differentialgleichungen.
			\item Viele Spezialf\ae lle k\oe nnen aber durch Integration gel\oe st werden.
		\end{itemize}
	\end{block}
\end{frame}

\end{document}
