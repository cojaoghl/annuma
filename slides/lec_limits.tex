\documentclass{beamer}
\usepackage{amsmath,graphics}
\usepackage{amssymb}

\usetheme{default}
\usepackage{xcolor}

\definecolor{solarizedBase03}{HTML}{002B36}
\definecolor{solarizedBase02}{HTML}{073642}
\definecolor{solarizedBase01}{HTML}{586e75}
\definecolor{solarizedBase00}{HTML}{657b83}
\definecolor{solarizedBase0}{HTML}{839496}
\definecolor{solarizedBase1}{HTML}{93a1a1}
\definecolor{solarizedBase2}{HTML}{EEE8D5}
\definecolor{solarizedBase3}{HTML}{FDF6E3}
\definecolor{solarizedYellow}{HTML}{B58900}
\definecolor{solarizedOrange}{HTML}{CB4B16}
\definecolor{solarizedRed}{HTML}{DC322F}
\definecolor{solarizedMagenta}{HTML}{D33682}
\definecolor{solarizedViolet}{HTML}{6C71C4}
%\definecolor{solarizedBlue}{HTML}{268BD2}
\definecolor{solarizedBlue}{HTML}{134676}
\definecolor{solarizedCyan}{HTML}{2AA198}
\definecolor{solarizedGreen}{HTML}{859900}
\definecolor{myBlue}{HTML}{162DB0}%{261CA4}
\setbeamercolor*{item}{fg=myBlue}
\setbeamercolor{normal text}{fg=solarizedBase03, bg=solarizedBase3}
\setbeamercolor{alerted text}{fg=myBlue}
\setbeamercolor{example text}{fg=myBlue, bg=solarizedBase3}
\setbeamercolor*{frametitle}{fg=solarizedRed}
\setbeamercolor*{title}{fg=solarizedRed}
\setbeamercolor{block title}{fg=myBlue, bg=solarizedBase3}
\setbeameroption{hide notes}
\setbeamertemplate{note page}[plain]
\beamertemplatenavigationsymbolsempty
\usefonttheme{professionalfonts}
\usefonttheme{serif}

\usepackage{fourier}

\def\vec#1{\mathchoice{\mbox{\boldmath$\displaystyle#1$}}
{\mbox{\boldmath$\textstyle#1$}}
{\mbox{\boldmath$\scriptstyle#1$}}
{\mbox{\boldmath$\scriptscriptstyle#1$}}}
\definecolor{OwnGrey}{rgb}{0.560,0.000,0.000} % #999999
\definecolor{OwnBlue}{rgb}{0.121,0.398,0.711} % #1f64b0
\definecolor{red4}{rgb}{0.5,0,0}
\definecolor{blue4}{rgb}{0,0,0.5}
\definecolor{Blue}{rgb}{0,0,0.66}
\definecolor{LightBlue}{rgb}{0.9,0.9,1}
\definecolor{Green}{rgb}{0,0.5,0}
\definecolor{LightGreen}{rgb}{0.9,1,0.9}
\definecolor{Red}{rgb}{0.9,0,0}
\definecolor{LightRed}{rgb}{1,0.9,0.9}
\definecolor{White}{gray}{1}
\definecolor{Black}{gray}{0}
\definecolor{LightGray}{gray}{0.8}
\definecolor{Orange}{rgb}{0.1,0.2,1}
\setbeamerfont{sidebar right}{size=\scriptsize}
\setbeamercolor{sidebar right}{fg=Black}

\renewcommand{\emph}[1]{{\textcolor{solarizedRed}{\itshape #1}}}

\newcommand\cA{\mathcal A}
\newcommand\cB{\mathcal B}
\newcommand\cC{\mathcal C}
\newcommand\cD{\mathcal D}
\newcommand\cE{\mathcal E}
\newcommand\cF{\mathcal F}
\newcommand\cG{\mathcal G}
\newcommand\cH{\mathcal H}
\newcommand\cI{\mathcal I}
\newcommand\cJ{\mathcal J}
\newcommand\cK{\mathcal K}
\newcommand\cL{\mathcal L}
\newcommand\cM{\mathcal M}
\newcommand\cN{\mathcal N}
\newcommand\cO{\mathcal O}
\newcommand\cP{\mathcal P}
\newcommand\cQ{\mathcal Q}
\newcommand\cR{\mathcal R}
\newcommand\cS{\mathcal S}
\newcommand\cT{\mathcal T}
\newcommand\cU{\mathcal U}
\newcommand\cV{\mathcal V}
\newcommand\cW{\mathcal W}
\newcommand\cX{\mathcal X}
\newcommand\cY{\mathcal Y}
\newcommand\cZ{\mathcal Z}

\newcommand\fA{\mathfrak A}
\newcommand\fB{\mathfrak B}
\newcommand\fC{\mathfrak C}
\newcommand\fD{\mathfrak D}
\newcommand\fE{\mathfrak E}
\newcommand\fF{\mathfrak F}
\newcommand\fG{\mathfrak G}
\newcommand\fH{\mathfrak H}
\newcommand\fI{\mathfrak I}
\newcommand\fJ{\mathfrak J}
\newcommand\fK{\mathfrak K}
\newcommand\fL{\mathfrak L}
\newcommand\fM{\mathfrak M}
\newcommand\fN{\mathfrak N}
\newcommand\fO{\mathfrak O}
\newcommand\fP{\mathfrak P}
\newcommand\fQ{\mathfrak Q}
\newcommand\fR{\mathfrak R}
\newcommand\fS{\mathfrak S}
\newcommand\fT{\mathfrak T}
\newcommand\fU{\mathfrak U}
\newcommand\fV{\mathfrak V}
\newcommand\fW{\mathfrak W}
\newcommand\fX{\mathfrak X}
\newcommand\fY{\mathfrak Y}
\newcommand\fZ{\mathfrak Z}

\newcommand\fa{\mathfrak a}
\newcommand\fb{\mathfrak b}
\newcommand\fc{\mathfrak c}
\newcommand\fd{\mathfrak d}
\newcommand\fe{\mathfrak e}
\newcommand\ff{\mathfrak f}
\newcommand\fg{\mathfrak g}
\newcommand\fh{\mathfrak h}
%\newcommand\fi{\mathfrak i}
\newcommand\fj{\mathfrak j}
\newcommand\fk{\mathfrak k}
\newcommand\fl{\mathfrak l}
\newcommand\fm{\mathfrak m}
\newcommand\fn{\mathfrak n}
\newcommand\fo{\mathfrak o}
\newcommand\fp{\mathfrak p}
\newcommand\fq{\mathfrak q}
\newcommand\fr{\mathfrak r}
\newcommand\fs{\mathfrak s}
\newcommand\ft{\mathfrak t}
\newcommand\fu{\mathfrak u}
\newcommand\fv{\mathfrak v}
\newcommand\fw{\mathfrak w}
\newcommand\fx{\mathfrak x}
\newcommand\fy{\mathfrak y}
\newcommand\fz{\mathfrak z}

\newcommand\vA{\vec A}
\newcommand\vB{\vec B}
\newcommand\vC{\vec C}
\newcommand\vD{\vec D}
\newcommand\vE{\vec E}
\newcommand\vF{\vec F}
\newcommand\vG{\vec G}
\newcommand\vH{\vec H}
\newcommand\vI{\vec I}
\newcommand\vJ{\vec J}
\newcommand\vK{\vec K}
\newcommand\vL{\vec L}
\newcommand\vM{\vec M}
\newcommand\vN{\vec N}
\newcommand\vO{\vec O}
\newcommand\vP{\vec P}
\newcommand\vQ{\vec Q}
\newcommand\vR{\vec R}
\newcommand\vS{\vec S}
\newcommand\vT{\vec T}
\newcommand\vU{\vec U}
\newcommand\vV{\vec V}
\newcommand\vW{\vec W}
\newcommand\vX{\vec X}
\newcommand\vY{\vec Y}
\newcommand\vZ{\vec Z}

\newcommand\va{\vec a}
\newcommand\vb{\vec b}
\newcommand\vc{\vec c}
\newcommand\vd{\vec d}
\newcommand\ve{\vec e}
\newcommand\vf{\vec f}
\newcommand\vg{\vec g}
\newcommand\vh{\vec h}
\newcommand\vi{\vec i}
\newcommand\vj{\vec j}
\newcommand\vk{\vec k}
\newcommand\vl{\vec l}
\newcommand\vm{\vec m}
\newcommand\vn{\vec n}
\newcommand\vo{\vec o}
\newcommand\vp{\vec p}
\newcommand\vq{\vec q}
\newcommand\vr{\vec r}
\newcommand\vs{\vec s}
\newcommand\vt{\vec t}
\newcommand\vu{\vec u}
\newcommand\vv{\vec v}
\newcommand\vw{\vec w}
\newcommand\vx{\vec x}
\newcommand\vy{\vec y}
\newcommand\vz{\vec z}

\renewcommand\AA{\mathbb A}
\newcommand\NN{\mathbb N}
\newcommand\ZZ{\mathbb Z}
\newcommand\PP{\mathbb P}
\newcommand\QQ{\mathbb Q}
\newcommand\RR{\mathbb R}
\renewcommand\SS{\mathbb S}
\newcommand\CC{\mathbb C}

\newcommand{\ord}{\mathrm{ord}}
\newcommand{\id}{\mathrm{id}}
\newcommand{\pr}{\mathrm{P}}
\newcommand{\Vol}{\mathrm{vol}}
\newcommand\norm[1]{\left\|{#1}\right\|} 
\newcommand\sign{\mathrm{sign}}
\newcommand{\eps}{\varepsilon}
\newcommand{\abs}[1]{\left|#1\right|}
\newcommand\bc[1]{\left({#1}\right)} 
\newcommand\cbc[1]{\left\{{#1}\right\}} 
\newcommand\bcfr[2]{\bc{\frac{#1}{#2}}} 
\newcommand{\bck}[1]{\left\langle{#1}\right\rangle} 
\newcommand\brk[1]{\left\lbrack{#1}\right\rbrack} 
\newcommand\scal[2]{\bck{{#1},{#2}}} 
\newcommand{\vecone}{\mathbb{1}}
\newcommand{\tensor}{\otimes}
\newcommand{\diag}{\mathrm{diag}}
\newcommand{\ggt}{\mathrm{ggT}}
\newcommand{\kgv}{\mathrm{kgV}}
\newcommand{\trans}{\top}

\newcommand{\Karonski}{Karo\'nski}
\newcommand{\Erdos}{Erd\H{o}s}
\newcommand{\Renyi}{R\'enyi}
\newcommand{\Lovasz}{Lov\'asz}
\newcommand{\Juhasz}{Juh\'asz}
\newcommand{\Bollobas}{Bollob\'as}
\newcommand{\Furedi}{F\"uredi}
\newcommand{\Komlos}{Koml\'os}
\newcommand{\Luczak}{\L uczak}
\newcommand{\Kucera}{Ku\v{c}era}
\newcommand{\Szemeredi}{Szemer\'edi}

\renewcommand{\ae}{\"a}
\renewcommand{\oe}{\"o}
\newcommand{\ue}{\"u}
\newcommand{\Ae}{\"A}
\newcommand{\Oe}{\"O}
\newcommand{\Ue}{\"U}

\newcommand{\im}{\mathrm{im}}
\newcommand{\rrk}{\mathrm{zrg}}
\newcommand{\crk}{\mathrm{srg}}
\newcommand{\rk}{\mathrm{rg}}
\newcommand{\GL}{\mathrm{GL}}
\newcommand{\SL}{\mathrm{SL}}
\newcommand{\SO}{\mathrm{SO}}
\newcommand{\nul}{\mathrm{nul}}
\newcommand{\eig}{\mathrm{eig}}

\newcommand{\mytitle}{Grenzwerte}

\title[Annuma]{\mytitle}
\author[Amin Coja-Oghlan]{Amin Coja-Oghlan}
\institute[Frankfurt]{JWGUFFM}
\date{}

\begin{document}

\frame[plain]{\titlepage}

\begin{frame}\frametitle{\mytitle}
	\begin{block}{Folgen und Grenzwerte}
		\begin{itemize}
			\item Reelle Zahlen sind Zahlen mit einer endlichen oder unendlichen Dezimalentwicklung wie beispielsweise
				\begin{align*}
					\frac{3}{4}&=0.75\\
					-\frac{1}{3}&=-0.\overline 3\\
					\pi&=3.1415926535897932384626433833\cdots
				\end{align*}
			\item A prior wissen wir nicht, wie wir mit solchen Zahlen rechnen sollen.
			\item Die Begriffe \emph{Folge} und \emph{Grenzwert} beantworten diese Frage.
		\end{itemize}
	\end{block}
\end{frame}

\begin{frame}\frametitle{\mytitle}
	\begin{block}{Folgen}
		\begin{itemize}
			\item Sei $\NN=\{1,2,3,\ldots\}$ die Menge der nat\ue rlichen Zahlen.
			\item Eine \emph{Folge} $(a_n)_{n\in\NN}$ ist eine Abbildung $\NN\to\RR$ die jeder Zahl $n\in\NN$ eine reelle Zahl $a_n$ zuordnet.
			\item Stellen Sie sich eine Folge als eine unendliche Liste von Zahlen vor.
		\end{itemize}
	\end{block}
\end{frame}

\begin{frame}\frametitle{\mytitle}
	\begin{block}{Beispiele}
		\begin{itemize}
			\item Die Folge $(a_n)_{n\in\NN}$ mit $a_n=1/n$ beginnt wie folgt:
				\begin{align*}
				1,\frac{1}{2},\frac{1}{3},\frac{1}{4},\frac{1}{5},\ldots
				\end{align*}
			\item Folgen k\oe nnen rekursiv definiert werden; ein Beispiel ist die \emph{Fibonacci-Folge} $(b_n)_n$ mit $b_1=0,b_2=1$ und $b_{n+1}=b_{n-1}+b_n$:
				\begin{align*}
				0,1,1,2,3,5,8,13,21,\ldots
				\end{align*}
			\item Ein weiteres Beispiel ist die \emph{Leibniz-Folge} $(c_n)_n$ mit $c_1=1$ und
				\begin{align*}
					c_{n}&=c_{n-1}+\frac{(-1)^{n-1}}{2n-1}&&(n>1)\\
					c_2&=1-\frac{1}{3}&
					c_3&=1-\frac{1}{3}+\frac{1}{5}&
					c_4&=1-\frac{1}{3}+\frac{1}{5}-\frac{1}{7}
				\end{align*}
		\end{itemize}
	\end{block}
\end{frame}

\begin{frame}\frametitle{\mytitle}
	\begin{block}{Beispiele}
		\begin{itemize}
			\item F\ue r eine unendliche Dezimalzahl erhalten wir eine Folge, wenn wir nur die ersten $n$ Nachkommastellen ber\ue cksichtigen:
				\begin{align*}
				&0.1\\
				&0.11\\
				&0.111\\
				&\vdots
				\end{align*}
			\item Oder f\ue r die Kreiszahl $\pi$:
				\begin{align*}
				&3.1\\
				&3.14\\
				&3.141\\
				&\vdots
				\end{align*}
		\end{itemize}
	\end{block}
\end{frame}

\begin{frame}\frametitle{\mytitle}
	\begin{block}{Definition}
		\begin{itemize}
			\item Sei $(a_n)_n$ eine Folge und $z\in\RR$.
			\item Wir nennen $z$ den \emph{Grenzwert} von $(a_n)_n$, wenn es zu jeder reellen Zahl $\eps>0$ eine Zahl $N\in\NN$ gibt, so da\ss\ 
				\begin{align*}
					|a_n-z|&<\eps\qquad\mbox{f\ue r alle }n>N.
				\end{align*}
			\item In diesem Fall schreiben wir 
				\begin{align*}
					z=\lim_{n\to\infty}a_n.
				\end{align*}
			\item \alert{Erinnerung:} f\ue r $s\in\RR$ ist der \emph{Betrag} definiert als
				\begin{align*}
					|s|&=\begin{cases}
						s&\mbox{ falls }s\geq0\\
						-s&\mbox{ falls }s<0
					\end{cases}
				\end{align*}
			\item \emph{Intuition:} f\ue r gro\ss e $n$ liegen die $a_n$ beliebig nahe an $z$.
		\end{itemize}
	\end{block}
\end{frame}

\begin{frame}\frametitle{\mytitle}
	\begin{block}{Beispiel}
		\begin{itemize}
			\item Betrachte die Folge $a_n=\frac{1}{n}$; wir zeigen, da\ss\
				\begin{align*}
					\lim_{n\to\infty}a_n=0.
				\end{align*}
			\item Sei dazu $\eps>0$ und w\ae hle $N=1/\eps$.
			\item Wenn $n>N$, dann gilt
				\begin{align*}
				|a_n-0|=|a_n|=\frac{1}{n}<\frac{1}{N}=\eps.
				\end{align*}
			\item Damit ist die Bedingung aus der Definition erf\ue llt.
		\end{itemize}
	\end{block}
\end{frame}

\begin{frame}\frametitle{\mytitle}
	\begin{block}{Beispiel}
		\begin{itemize}
			\item Die Kreiszahl $\pi$ hat eine unendliche Dezimalentwicklung.
			\item Dennoch k\oe nne wir eine Folge von Dezimalzahlen endlicher L\ae nge konstruieren, deren Grenzwert $\pi$ ist, indem wir nur die ersten $n$ Nachkommastellen ``mitnehmen'':
				\begin{align*}
					a_1&=3.1\\
					a_2&=3.14\\
					a_3&=3.141\\
					a_4&=3.1415\qquad\mbox{etc.}
				\end{align*}
			\item Die Folge $(a_n)_n$ konvergiert gegen $\pi$, d.h.\
				\begin{align*}
					\lim_{n\to\infty}a_n&=\pi.
				\end{align*}
		\end{itemize}
	\end{block}
\end{frame}

\begin{frame}\frametitle{\mytitle}
	\begin{block}{Beispiel (fortgesetzt)}
		\begin{itemize}
			\item Um das einzusehen, berechnen wir die Differenzen:
				\begin{align*}
					\abs{\pi-3.1}&=3.1415\cdots-3.1=0.04159\cdots\leq0.1\\
					\abs{\pi-3.14}&=3.1415\cdots-3.14=0.00159\cdots\leq0.01\\
					\abs{\pi-3.141}&=3.1415\cdots-3.141=0.00059\cdots\leq0.001\\
					\abs{\pi-3.1415}&=3.1415\cdots-3.1415=0.00009\cdots\leq0.0001\\
									&\vdots
				\end{align*}
			\item Der ``Approximationsfehler'' wird also beliebig klein.
			\item Der Grenzwert der Folge ist daher $\pi$.
		\end{itemize}
	\end{block}
\end{frame}

\begin{frame}\frametitle{\mytitle}
	\begin{block}{Gegenbeispiel}
		\begin{itemize}
			\item Die Fibonaccifolge war definiert durch $a_1=0$, $a_2=1$ und 
				\begin{align*}
					a_n=a_{n-2}+a_{n-1}\qquad\mbox{f\ue r $n\geq3$}.
				\end{align*}
			\item Diese Folge besitzt keinen Grenzwert.
			\item Denn $a_n\geq n-1$ f\ue r alle $n$.
			\item Die Folgenglieder $a_n$ werden also unbeschr\ae nkt gro\ss.
		\end{itemize}
	\end{block}
\end{frame}

\begin{frame}\frametitle{\mytitle}
	\begin{block}{Gegenbeispiel}
		\begin{itemize}
			\item Die Folge $a_n=(-1)^n$ alterniert zwischen den beiden Zahlen $-1$ und $+1$:
				\begin{align*}
				-1,+1,-1,+1,-1,+1,\ldots
				\end{align*}
			\item Daher konvergiert die Folge nicht.
		\end{itemize}
	\end{block}
\end{frame}

\begin{frame}\frametitle{\mytitle}
	\begin{block}{Definition}
	\begin{itemize}
		\item Eine Folge $(a_n)_n$ hei\ss t \emph{konvergent}, wenn sie einen Grenzwert hat.
		\item Eine Folge $(a_n)_n$ hei\ss t \emph{beschr\ae nkt}, falls die Menge
			\begin{align*}
				\cbc{a_n:n\in\NN}
			\end{align*}
			beschr\ae nkt ist.
		\item Eine Folge $(a_n)_n$ hei\ss t \emph{monoton wachsend}, wenn
			\begin{align*}
				a_{n+1}\geq a_n\qquad\mbox{f\ue r alle }n\in\NN.
			\end{align*}
		\item Eine Folge $(a_n)_n$ hei\ss t \emph{monoton fallend}, wenn
			\begin{align*}
				a_{n+1}\leq a_n\qquad\mbox{f\ue r alle }n\in\NN.
			\end{align*}
	\end{itemize}
	\end{block}
\end{frame}

\begin{frame}\frametitle{\mytitle}
	\begin{block}{Proposition}
	\begin{itemize}
		\item Jede konvergente Folge ist beschr\ae nkt.
		\item Eine monoton wachsende beschr\ae nkte Folge $(a_n)_n$ konvergiert und
			\begin{align*}
				\lim_{n\to\infty}a_n&=\sup\cbc{a_n:n\in\NN}.
			\end{align*}
		\item Eine monoton fallende beschr\ae nkte Folge $(a_n)_n$ konvergiert und
			\begin{align*}
				\lim_{n\to\infty}a_n&=\inf\cbc{a_n:n\in\NN}.
			\end{align*}
		\item Die konstante Folge $(a_n)_n$ mit $a_n=\alpha\in\RR$ hat den Grenzwert $\alpha$.
		\item F\ue r jede reelle Zahl $z$ gibt es eine Folge $(a_n)_n$ ratinaler Zahlen $a_n\in\QQ$ mit
			\begin{align*}
				\lim_{n\to\infty}a_n=z.
			\end{align*}
	\end{itemize}
	\end{block}
\end{frame}

\begin{frame}\frametitle{\mytitle}
	\begin{block}{Proposition}
		Angenommen $(a_n)_n$ und $(b_n)_n$ sind Folgen mit
		\begin{align*}
			\lim_{n\to\infty}a_n=\alpha\qquad\mbox{und}\qquad\lim_{n\to\infty}b_n=\beta.
		\end{align*}
		\begin{itemize}
			\item Die Folge $(a_n+b_n)_n$ konvergiert und $ \lim_{n\to\infty}a_n+b_n=\alpha+\beta.  $
			\item Die Folge $(a_n-b_n)_n$ konvergiert und $ \lim_{n\to\infty}a_n-b_n=\alpha-\beta.  $
			\item Die Folge $(a_n\cdot b_n)_n$ konvergiert und $ \lim_{n\to\infty}a_n\cdot b_n=\alpha\cdot\beta.  $
			\item Wenn $\beta>0$, dann konvergiert die Folge $(a_n\cdot b_n)_n$ und $$ \lim_{n\to\infty}\frac{a_n}{b_n}=\frac\alpha\beta.  $$
		\end{itemize}
	\end{block}
\end{frame}

\begin{frame}\frametitle{\mytitle}
	\begin{block}{Rechnen mit reellen Zahlen}
		\begin{itemize}
			\item Wir k\oe nnen also \alert{indirekt} mit reellen Zahlen rechnen, indem wir mit Approximationen rechnen.
			\item Angenommen wir k\oe nnen beliebig gute Approximationen von zwei reellen Zahlen $x,y$ berechnen.
			\item Dann k\oe nnen wir auch beliebig gute Approximationen von
				\begin{align*}
					x\cdot y\quad x+y\quad x-y
				\end{align*}
				berechnen.
			\item Wenn $y\neq0$ k\oe nnen wir au\ss erdem $\frac xy$ beliebig gut approximieren.
		\end{itemize}
	\end{block}
\end{frame}

\begin{frame}\frametitle{\mytitle}
	\begin{block}{Rechnen mit reellen Zahlen}
		\begin{itemize}
			\item Wir k\oe nnen also \alert{indirekt} mit reellen Zahlen rechnen, indem wir mit Approximationen rechnen.
			\item Angenommen wir k\oe nnen beliebig gute Approximationen von zwei reellen Zahlen $x,y$ berechnen.
			\item Dann k\oe nnen wir auch beliebig gute Approximationen von
				\begin{align*}
					x\cdot y\quad x+y\quad x-y
				\end{align*}
				berechnen.
			\item Wenn $y\neq0$ k\oe nnen wir au\ss erdem $\frac xy$ beliebig gut approximieren.
			\item \itshape Im Vorlesungsteil Numerik werden wir uns damit befassen, \emph{wie} gut diese Approximationen sind.
		\end{itemize}
	\end{block}
\end{frame}

\begin{frame}\frametitle{\mytitle}
	\begin{block}{Reihen}
		\begin{itemize}
			\item Reihen sind eine besondere, besonders wichtige Art von Folgen.
			\item Sei $(a_n)_n$ eine Folge.
			\item Dann k\oe nnen wir eine weitere Folge $(A_n)_n$ definieren durch
				\begin{align*}
					A_n=\sum_{i=1}^na_i=a_1+a_2+\cdots+a_n.
				\end{align*}
			\item Die Folge $(A_n)_n$ hei\ss t die \emph{Reihe mit Gliedern $a_n$}.
			\item \alert{Schreibweise:} wenn $(A_n)_n$ konvergiert, schreiben wir
				\begin{align*}
					\lim_{n\to\infty}A_n=\lim_{n\to\infty}\sum_{i=1}^na_i=\sum_{n=1}^\infty a_n.
				\end{align*}
		\end{itemize}
	\end{block}
\end{frame}

\begin{frame}\frametitle{\mytitle}
	\begin{block}{Beispiel: die geometrische Reihe}
		\begin{itemize}
			\item Sei $a_n=10^{-n}$.
			\item Dann ist also $ A_n=\sum_{i=1}^n10^{-i} $.
			\item Wir berechnen
				\begin{align*}
					(1-1/10)A_n&=A_n-A_n/10=\sum_{i=1}^n10^{-i}-\sum_{i=1}^n10^{-i-1}\\
							   &=\sum_{i=1}^n10^{-i}-\sum_{i=2}^{n+1}10^{-i}=0.1-10^{-n-1}.
				\end{align*}
			\item Also gilt $\displaystyle A_n=\frac{0.1-10^{-n-1}}{1-10^{-1}}.  $
		\end{itemize}
	\end{block}
\end{frame}

\begin{frame}\frametitle{\mytitle}
	\begin{block}{Beispiel: die geometrische Reihe (fortgesetzt)}
		\begin{itemize}
			\item Es gilt $\displaystyle A_n=\frac{0.1-10^{-n-1}}{1-10^{-1}}.  $
			\item Die Folge $10^{-n-1}$ konvergiert rasch gegen 0:
				\begin{align*}
				&0.01\\
				&0.001\\
				&0.0001\\
				&\vdots
				\end{align*}
			\item Also sehen wir, da\ss
				\begin{align*}
					\lim_{n\to\infty}A_n&=\frac{0.1}{0.9}=\frac{1}{9}=0.\overline 1\enspace.
				\end{align*}
		\end{itemize}
	\end{block}
\end{frame}

\begin{frame}\frametitle{\mytitle}
	\begin{block}{Allgemein: die geometrische Reihe}
		\begin{itemize}
			\item Sei $x$ eine reelle Zahl mit $|x|<1$.
			\item Dann konvergiert die Reihe $\sum_{n=0}^\infty x^n=\sum_{n=1}^\infty x^{n-1}$ gegen den Grenzwert $1/(1-x)$.
		\end{itemize}
	\end{block}
\end{frame}

\begin{frame}\frametitle{\mytitle}
	\begin{block}{Zahlenr\ae tsel}
		\begin{itemize}
			\item Wie steht es um die Reihe
				$$\sum_{n=1}^\infty\frac{(-1)^{n+1}}n=1-\frac{1}{2}+\frac{1}{3}-\frac{1}{4}+\cdots$$
			\item Wir \ue berlegen uns zun\ae chst, da\ss\ diese Reihe konvergiert.
			\item Dazu kombinieren wir immer zwei aufeinanderfolgende Summanden:
				\begin{align*}
					\frac{1}{2n-1}-\frac{1}{2n}&=\frac{2n-(2n-1)}{4n(n-1)}=\frac{1}{2n(2n-1)}.
				\end{align*}
			\item Die ``gerade'' Reihe hat also lauter positive Glieder:
				\begin{align*}
					\sum_{n=1}^{N}\bc{\frac{1}{2n-1}-\frac{1}{2n}}=\sum_{n=1}^N\frac{1}{2n(2n-1)}
				\end{align*}
		\end{itemize}
	\end{block}
\end{frame}

\begin{frame}\frametitle{\mytitle}
	\begin{block}{Zahlenr\ae tsel}
		\begin{itemize}
			\item Diese Folge ist daher monoton wachsend.
			\item Wir m\ue ssen uns noch \ue berlegen, da\ss\ sie auch beschr\ae nkt ist.
			\item Weil $4n(n-1)>n^2$ f\ue r $n>1$, reicht es aus, die Reihe
				\begin{align*}
					\zeta(2)=\sum_{n=1}^\infty\frac{1}{n^2}=\frac{1}{1}+\frac{1}{4}+\frac{1}{9}+\frac{1}{16}+\frac{1}{25}+\frac{1}{36}+\frac{1}{49}+\cdots
				\end{align*}
				zu betrachten.
			\item Dazu fassen wir jeweils $2^k$, $k\geq0$, aufeinanderfolgende Summanden zusammen:
				\begin{align*}
					\underbrace{\textcolor{red}{\frac{1}{1}}}_{2^0=1}+\underbrace{\textcolor{blue}{\frac{1}{4}+\frac{1}{9}}}_{2^1=2}+\underbrace{\textcolor{magenta}{\frac{1}{16}+\frac{1}{25}+\frac{1}{36}+\frac{1}{49}}}_{2^2=4}+\cdots
				\end{align*}
		\end{itemize}
	\end{block}
\end{frame}

\begin{frame}\frametitle{\mytitle}
	\begin{block}{Zahlenr\ae tsel}
		\begin{itemize}
			\item Jeses Teilst\ue ck enth\ae lt $2^k$ Summanden, von denen der gr\oe\ss te den Wert $2^{-2k}$ hat.
			\item Deshalb ist jede Teilsumme durch $2^{-k}$ beschr\ae nkt:
				\begin{align*}
					\textcolor{red}{\frac{1}{1}}&=1=2^{-0}\\
					\textcolor{blue}{\frac{1}{4}+\frac{1}{9}}&=0.36\overline 1<0.5=2^{-1}\\
					\textcolor{magenta}{\frac{1}{16}+\frac{1}{25}+\frac{1}{36}+\frac{1}{49}}&=\frac{26581}{176400}<0.151<0.25=2^{-2}\\
																							&\vdots
				\end{align*}
			\item Wir sehen also, da\ss\ $\displaystyle \zeta(2)\leq\sum_{n=0}^\infty2^{-n}=2.$
		\end{itemize}
	\end{block}
\end{frame}

\begin{frame}\frametitle{\mytitle}
	\begin{block}{Zahlenr\ae tsel}
		\begin{itemize}
			\item Die Reihe 
				$$\sum_{n=1}^\infty\frac{(-1)^{n+1}}n=1-\frac{1}{2}+\frac{1}{3}-\frac{1}{4}+\cdots$$
				ist also ebenfalls beschr\ae nkt.
			\item Weil die ``gerade'' Teilsummen
				$$\sum_{n=1}^{2N}\frac{(-1)^{n+1}}n$$
				monton wachsen und
				\begin{align*}
				\abs{\sum_{n=1}^{2N+1}\frac{(-1)^{n+1}}n-\sum_{n=1}^{2N}\frac{(-1)^{n+1}}n}\leq\frac{1}{2N+1}
				\end{align*}
				beliebig klein wird, konvergiert die Reihe $\sum_{n=1}^\infty(-1)^{n+1}/n$.
		\end{itemize}
	\end{block}
\end{frame}

\begin{frame}\frametitle{\mytitle}
	\begin{block}{Zusammenfassung}
		\begin{itemize}
			\item Eine Folge konvergiert gegen einen Grenzwert $z$, wenn die Folgenglieder $z$ schlie\ss lich beliebig nahe kommen.
			\item Grenzwerte sind mit den Grundrechenarten vertr\ae glich, daher k\oe nnen wir jetzt prinzipiell mit reellen Zahlen rechnen.
			\item Reihen sind spezielle Folgen, die eine Summenform haben.
		\end{itemize}
	\end{block}
\end{frame}
\end{document}
