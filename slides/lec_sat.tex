\documentclass{beamer}
\usepackage{amsmath,graphics}
\usepackage{amssymb}

\usetheme{default}
\usepackage{xcolor}

\definecolor{solarizedBase03}{HTML}{002B36}
\definecolor{solarizedBase02}{HTML}{073642}
\definecolor{solarizedBase01}{HTML}{586e75}
\definecolor{solarizedBase00}{HTML}{657b83}
\definecolor{solarizedBase0}{HTML}{839496}
\definecolor{solarizedBase1}{HTML}{93a1a1}
\definecolor{solarizedBase2}{HTML}{EEE8D5}
\definecolor{solarizedBase3}{HTML}{FDF6E3}
\definecolor{solarizedYellow}{HTML}{B58900}
\definecolor{solarizedOrange}{HTML}{CB4B16}
\definecolor{solarizedRed}{HTML}{DC322F}
\definecolor{solarizedMagenta}{HTML}{D33682}
\definecolor{solarizedViolet}{HTML}{6C71C4}
%\definecolor{solarizedBlue}{HTML}{268BD2}
\definecolor{solarizedBlue}{HTML}{134676}
\definecolor{solarizedCyan}{HTML}{2AA198}
\definecolor{solarizedGreen}{HTML}{859900}
\definecolor{myBlue}{HTML}{162DB0}%{261CA4}
\setbeamercolor*{item}{fg=myBlue}
\setbeamercolor{normal text}{fg=solarizedBase03, bg=solarizedBase3}
\setbeamercolor{alerted text}{fg=myBlue}
\setbeamercolor{example text}{fg=myBlue, bg=solarizedBase3}
\setbeamercolor*{frametitle}{fg=solarizedRed}
\setbeamercolor*{title}{fg=solarizedRed}
\setbeamercolor{block title}{fg=myBlue, bg=solarizedBase3}
\setbeameroption{hide notes}
\setbeamertemplate{note page}[plain]
\beamertemplatenavigationsymbolsempty
\usefonttheme{professionalfonts}
\usefonttheme{serif}

\usepackage{fourier}

\def\vec#1{\mathchoice{\mbox{\boldmath$\displaystyle#1$}}
{\mbox{\boldmath$\textstyle#1$}}
{\mbox{\boldmath$\scriptstyle#1$}}
{\mbox{\boldmath$\scriptscriptstyle#1$}}}
\definecolor{OwnGrey}{rgb}{0.560,0.000,0.000} % #999999
\definecolor{OwnBlue}{rgb}{0.121,0.398,0.711} % #1f64b0
\definecolor{red4}{rgb}{0.5,0,0}
\definecolor{blue4}{rgb}{0,0,0.5}
\definecolor{Blue}{rgb}{0,0,0.66}
\definecolor{LightBlue}{rgb}{0.9,0.9,1}
\definecolor{Green}{rgb}{0,0.5,0}
\definecolor{LightGreen}{rgb}{0.9,1,0.9}
\definecolor{Red}{rgb}{0.9,0,0}
\definecolor{LightRed}{rgb}{1,0.9,0.9}
\definecolor{White}{gray}{1}
\definecolor{Black}{gray}{0}
\definecolor{LightGray}{gray}{0.8}
\definecolor{Orange}{rgb}{0.1,0.2,1}
\setbeamerfont{sidebar right}{size=\scriptsize}
\setbeamercolor{sidebar right}{fg=Black}

\renewcommand{\emph}[1]{{\textcolor{solarizedRed}{\itshape #1}}}

\newcommand\tay{T}
\newcommand\dd{\mathrm d}
\newcommand\eul{\mathrm e}

\newcommand\cA{\mathcal A}
\newcommand\cB{\mathcal B}
\newcommand\cC{\mathcal C}
\newcommand\cD{\mathcal D}
\newcommand\cE{\mathcal E}
\newcommand\cF{\mathcal F}
\newcommand\cG{\mathcal G}
\newcommand\cH{\mathcal H}
\newcommand\cI{\mathcal I}
\newcommand\cJ{\mathcal J}
\newcommand\cK{\mathcal K}
\newcommand\cL{\mathcal L}
\newcommand\cM{\mathcal M}
\newcommand\cN{\mathcal N}
\newcommand\cO{\mathcal O}
\newcommand\cP{\mathcal P}
\newcommand\cQ{\mathcal Q}
\newcommand\cR{\mathcal R}
\newcommand\cS{\mathcal S}
\newcommand\cT{\mathcal T}
\newcommand\cU{\mathcal U}
\newcommand\cV{\mathcal V}
\newcommand\cW{\mathcal W}
\newcommand\cX{\mathcal X}
\newcommand\cY{\mathcal Y}
\newcommand\cZ{\mathcal Z}

\newcommand\fA{\mathfrak A}
\newcommand\fB{\mathfrak B}
\newcommand\fC{\mathfrak C}
\newcommand\fD{\mathfrak D}
\newcommand\fE{\mathfrak E}
\newcommand\fF{\mathfrak F}
\newcommand\fG{\mathfrak G}
\newcommand\fH{\mathfrak H}
\newcommand\fI{\mathfrak I}
\newcommand\fJ{\mathfrak J}
\newcommand\fK{\mathfrak K}
\newcommand\fL{\mathfrak L}
\newcommand\fM{\mathfrak M}
\newcommand\fN{\mathfrak N}
\newcommand\fO{\mathfrak O}
\newcommand\fP{\mathfrak P}
\newcommand\fQ{\mathfrak Q}
\newcommand\fR{\mathfrak R}
\newcommand\fS{\mathfrak S}
\newcommand\fT{\mathfrak T}
\newcommand\fU{\mathfrak U}
\newcommand\fV{\mathfrak V}
\newcommand\fW{\mathfrak W}
\newcommand\fX{\mathfrak X}
\newcommand\fY{\mathfrak Y}
\newcommand\fZ{\mathfrak Z}

\newcommand\fa{\mathfrak a}
\newcommand\fb{\mathfrak b}
\newcommand\fc{\mathfrak c}
\newcommand\fd{\mathfrak d}
\newcommand\fe{\mathfrak e}
\newcommand\ff{\mathfrak f}
\newcommand\fg{\mathfrak g}
\newcommand\fh{\mathfrak h}
%\newcommand\fi{\mathfrak i}
\newcommand\fj{\mathfrak j}
\newcommand\fk{\mathfrak k}
\newcommand\fl{\mathfrak l}
\newcommand\fm{\mathfrak m}
\newcommand\fn{\mathfrak n}
\newcommand\fo{\mathfrak o}
\newcommand\fp{\mathfrak p}
\newcommand\fq{\mathfrak q}
\newcommand\fr{\mathfrak r}
\newcommand\fs{\mathfrak s}
\newcommand\ft{\mathfrak t}
\newcommand\fu{\mathfrak u}
\newcommand\fv{\mathfrak v}
\newcommand\fw{\mathfrak w}
\newcommand\fx{\mathfrak x}
\newcommand\fy{\mathfrak y}
\newcommand\fz{\mathfrak z}

\newcommand\vA{\vec A}
\newcommand\vB{\vec B}
\newcommand\vC{\vec C}
\newcommand\vD{\vec D}
\newcommand\vE{\vec E}
\newcommand\vF{\vec F}
\newcommand\vG{\vec G}
\newcommand\vH{\vec H}
\newcommand\vI{\vec I}
\newcommand\vJ{\vec J}
\newcommand\vK{\vec K}
\newcommand\vL{\vec L}
\newcommand\vM{\vec M}
\newcommand\vN{\vec N}
\newcommand\vO{\vec O}
\newcommand\vP{\vec P}
\newcommand\vQ{\vec Q}
\newcommand\vR{\vec R}
\newcommand\vS{\vec S}
\newcommand\vT{\vec T}
\newcommand\vU{\vec U}
\newcommand\vV{\vec V}
\newcommand\vW{\vec W}
\newcommand\vX{\vec X}
\newcommand\vY{\vec Y}
\newcommand\vZ{\vec Z}

\newcommand\va{\vec a}
\newcommand\vb{\vec b}
\newcommand\vc{\vec c}
\newcommand\vd{\vec d}
\newcommand\ve{\vec e}
\newcommand\vf{\vec f}
\newcommand\vg{\vec g}
\newcommand\vh{\vec h}
\newcommand\vi{\vec i}
\newcommand\vj{\vec j}
\newcommand\vk{\vec k}
\newcommand\vl{\vec l}
\newcommand\vm{\vec m}
\newcommand\vn{\vec n}
\newcommand\vo{\vec o}
\newcommand\vp{\vec p}
\newcommand\vq{\vec q}
\newcommand\vr{\vec r}
\newcommand\vs{\vec s}
\newcommand\vt{\vec t}
\newcommand\vu{\vec u}
\newcommand\vv{\vec v}
\newcommand\vw{\vec w}
\newcommand\vx{\vec x}
\newcommand\vy{\vec y}
\newcommand\vz{\vec z}

\renewcommand\AA{\mathbb A}
\newcommand\NN{\mathbb N}
\newcommand\ZZ{\mathbb Z}
\newcommand\PP{\mathbb P}
\newcommand\QQ{\mathbb Q}
\newcommand\RR{\mathbb R}
\newcommand\RRpos{\mathbb R_{\geq0}}
\renewcommand\SS{\mathbb S}
\newcommand\CC{\mathbb C}

\newcommand{\ord}{\mathrm{ord}}
\newcommand{\id}{\mathrm{id}}
\newcommand{\pr}{\mathrm{P}}
\newcommand{\Vol}{\mathrm{vol}}
\newcommand\norm[1]{\left\|{#1}\right\|} 
\newcommand\sign{\mathrm{sign}}
\newcommand{\eps}{\varepsilon}
\newcommand{\abs}[1]{\left|#1\right|}
\newcommand\bc[1]{\left({#1}\right)} 
\newcommand\cbc[1]{\left\{{#1}\right\}} 
\newcommand\bcfr[2]{\bc{\frac{#1}{#2}}} 
\newcommand{\bck}[1]{\left\langle{#1}\right\rangle} 
\newcommand\brk[1]{\left\lbrack{#1}\right\rbrack} 
\newcommand\scal[2]{\bck{{#1},{#2}}} 
\newcommand{\vecone}{\mathbb{1}}
\newcommand{\tensor}{\otimes}
\newcommand{\diag}{\mathrm{diag}}
\newcommand{\ggt}{\mathrm{ggT}}
\newcommand{\kgv}{\mathrm{kgV}}
\newcommand{\trans}{\top}

\newcommand{\Karonski}{Karo\'nski}
\newcommand{\Erdos}{Erd\H{o}s}
\newcommand{\Renyi}{R\'enyi}
\newcommand{\Lovasz}{Lov\'asz}
\newcommand{\Juhasz}{Juh\'asz}
\newcommand{\Bollobas}{Bollob\'as}
\newcommand{\Furedi}{F\"uredi}
\newcommand{\Komlos}{Koml\'os}
\newcommand{\Luczak}{\L uczak}
\newcommand{\Kucera}{Ku\v{c}era}
\newcommand{\Szemeredi}{Szemer\'edi}

\renewcommand{\ae}{\"a}
\renewcommand{\oe}{\"o}
\newcommand{\ue}{\"u}
\newcommand{\Ae}{\"A}
\newcommand{\Oe}{\"O}
\newcommand{\Ue}{\"U}

\newcommand{\im}{\mathrm{im}}
\newcommand{\rrk}{\mathrm{zrg}}
\newcommand{\crk}{\mathrm{srg}}
\newcommand{\rk}{\mathrm{rg}}
\newcommand{\GL}{\mathrm{GL}}
\newcommand{\SL}{\mathrm{SL}}
\newcommand{\SO}{\mathrm{SO}}
\newcommand{\nul}{\mathrm{nul}}
\newcommand{\eig}{\mathrm{eig}}

\newcommand{\mytitle}{Anwendung: zuf\"alliges 3-SAT}

\title[Annuma]{\mytitle}
\author[Amin Coja-Oghlan]{Amin Coja-Oghlan}
\institute[Frankfurt]{JWGUFFM}
\date{}

\begin{document}

\frame[plain]{\titlepage}

\begin{frame}\frametitle{\mytitle}
	\begin{block}{Das 3-SAT-Problem}
		\begin{itemize}
			\item Das aussagenlogische Erf\"ullbarkeitsproblem spielt eine zentrale Rolle in der Informatik:
			\begin{itemize}
			\item Komlexit\"atstheorie
			\item Datenbanken
			\item {\em model checking}
			\end{itemize}
\item Ein besonders wichtiger Spezialfall ist das 3-SAT-Problem.
\item Gegeben ist hierbei eine aussagenlogische Formel \"uber Booleschen Variablen $x_1,\ldots,x_n$ in \emph{konjuktiver Normalform}
		\begin{align*}
			\Phi&=C_1\wedge C_2\wedge\cdots\wedge C_m&&\mbox{mit Klauseln}\\
			C_i&=\ell_{i1}\vee\ell_{i2}\vee\ell_{i3}&&\mbox{aus Literalen}\\
			\ell_{ij}&\in\cbc{x_1,\neg x_1,x_2,\neg x_2,\ldots,x_n,\neg x_n}.
		\end{align*}
		\item Die Aufgabe ist, die Variablen $x_1,\ldots,x_n$ mit Wahrheitswerten zu belegen, so da\ss\ die Formel insgesamt erf\"ullt ist.
		\end{itemize}
	\end{block}
\end{frame}

\begin{frame}\frametitle{\mytitle}
	\begin{block}{Beispiel}
		\begin{itemize}
			\item Gegeben sei die Formel
				\begin{align*}
					\Phi&=\bc{x_1\vee\neg x_2\vee x_3}\wedge\bc{\neg x_1\vee\neg x_2\vee\neg x_3}\wedge\bc{x_1\vee x_2\vee\neg x_3}\\&\quad\wedge\bc{\neg x_1\vee x_2\vee x_3}\wedge\bc{\neg x_1\vee \neg x_2\vee x_3}
				\end{align*}
			\item Wir repr\"asentieren die Wahrheitswerte durch $0$ und $1$.
			\item Die Belegung
				\begin{align*}
					x_1&=1&x_2&=0&x_3=1
				\end{align*}
				erf\"ullt die Formel.
		\end{itemize}
	\end{block}
\end{frame}

\begin{frame}\frametitle{\mytitle}
	\begin{block}{Algoritmen f\"ur 3-SAT}
		\begin{itemize}
			\item Das 3-SAT-Problem ist \alert{NP-schwer}.
			\item Es ist also kein effizienter Algorithmus bekannt, der das Problem jedenfalls l\"ost.
			\item Dennoch tritt das Problem in der Praxis st\"andig auf.
			\item Daher wurden verschiedene Heuristiken entwickelt.
			\item Eine solche Heuristik wird \alert{Unit Clause Propagation} (``UCP'') genannt.
		\end{itemize}
	\end{block}
\end{frame}

\begin{frame}\frametitle{\mytitle}
	\begin{block}{Algorithmus UCP}
		\alert{Eingabe:} eine 3-SAT-Formel $\Phi$.
		\begin{enumerate}
			\item Solange noch nicht alle Variablen belegt sind
			\item $\qquad$pr\"ufe, ob eine unerf\"ullte Klausel der L\"ange 1 hat
			\item $\qquad$falls ja, erf\"ulle eine solche Klausel
			\item $\qquad$sonst w\"ahle eine unbelegte Variable aus\dots 
			\item $\qquad$\dots und weise ihr einen zuf\"alligen Wert zu.
		\end{enumerate}
	\end{block}
\end{frame}

\begin{frame}\frametitle{\mytitle}
	\begin{block}{Beispiel}
		\begin{itemize}
			\item Wir betrachten die Formel
				\begin{align*}
					\Phi&=\bc{x_1\vee\neg x_2\vee x_3}\wedge\bc{\neg x_1\vee\neg x_2\vee\neg x_3}\wedge\bc{x_1\vee x_2\vee\neg x_3}\\&\quad\wedge\bc{\neg x_1\vee x_2\vee x_3}\wedge\bc{\neg x_1\vee \neg x_2\vee x_3}
				\end{align*}
			\item Anfangs haben alle Klauseln die L\"ange 3.
			\item Wir w\"ahlen also eine Variable zuf\"allig, z.B.\ $x_1$, und weisen ihr einen zuf\"alligen Wert zu: $x_1=1$.
			\item Die erste und dritte Klausel sind jetzt erf\"ullt.
			\item Es verbleibt die Restformel
				\begin{align*}
					\bc{\neg x_2\vee\neg x_3}\wedge\bc{x_2\vee x_3}\wedge\bc{\neg x_2\vee x_3}
				\end{align*}
		\end{itemize}
	\end{block}
\end{frame}

\begin{frame}\frametitle{\mytitle}
	\begin{block}{Beispiel}
		\begin{itemize}
			\item Da keine Klausel L\"ange 1 hat, w\"ahlen wir eine Variable zuf\"allig und weisen ihr einen zuf\"alligen Wert zu.
			\item Sagen wir $x_2=1$.
			\item Jetzt verbleit die Restformel
				\begin{align*}
					\bc{\neg x_3}\wedge\bc{x_3}
				\end{align*}
			\item Zwei der Klauseln haben L\"ange 1.
			\item Wir w\"ahlen die erste und erf\"ullen sie, indem wir $x_3=0$ setzen.
			\item Es verbleibt eine unerf\"ullte \alert{leere Klausel}.
			\item UCP schl\"agt also fehl.
		\end{itemize}
	\end{block}
\end{frame}

\begin{frame}\frametitle{\mytitle}
	\begin{block}{Zuf\"alliges 3-SAT}
		\begin{itemize}
			\item Zuf\"allig erzeugte 3-SAT-Formeln sind ein n\"utzlicher Benchmark.
			\item Dabei w\"ahlen wir die Klauseln der Formel $\Phi$ unabh\"angig voneinander zuf\"allig unter den $8n^3$ m\"oglichen Klauseln auf den Variablen
				\begin{align*}
				x_1,x_2,\ldots,x_n
				\end{align*}
				aus.
			\item Je gr\"o\ss er die Zahl $m$ der Klauseln ist, desto schwieriger wird es, eine erf\"ullende Belegung zu finden.
			\item \alert{Bis zu welcher Klauselzahl $m$ funktioniert UCP?}
		\end{itemize}
	\end{block}
\end{frame}

\begin{frame}\frametitle{\mytitle}
	\begin{block}{Kombinatorische Beobachtung}
		\begin{itemize}
			\item Wende UCP auf eine zuf\"allige Formel an.
			\item Dann ist die \alert{Restformel} nach $t$ Belegungen zuf\"allig gegeben
				\begin{itemize}
				\item die Menge der unbelegten Variablen
				\item die Zahlen $$m_1(t),m_2(t),m_3(t)$$ 
					der verbleibenden Klauseln der L\"angen $1,2,3$.
				\end{itemize}
			\item Deshalb reicht es aus, $m_1(t),m_2(t),m_3(t)$ zu analysieren.
		\end{itemize}
	\end{block}
\end{frame}

\begin{frame}\frametitle{\mytitle}
	\begin{block}{Erwartete Ver\"anderung}
		\begin{itemize}
			\item Mit $t$ bezeichnen wir die Anzahl der bereits gesetzen Variablen.
			\item Dann erhalten wir f\"ur die erwartete Ver\"anderung der Klauseln der L\"ange drei:
				\begin{align*}
					\Delta m_3(t)&=-\frac{3m_3(t)}{n-t}&&m_3(0)=m
				\end{align*}
			\item \alert{Erkl\"arung:} in jedem Schritt wird eine zuf\"allige Variable gesetzt
			\item Eine bestimmte 3-Klausel wird daher mit Wahrscheinlichkeit $3/(n-t)$ getroffen.
		\end{itemize}
	\end{block}
\end{frame}

\begin{frame}\frametitle{\mytitle}
	\begin{block}{Erwartete Ver\"anderung (fortgesetzt)}
		\begin{itemize}
			\item Analog bestimmen wir die erwartete Ver\"anderung der Klauseln der L\"ange zwei:
				\begin{align*}
					\Delta m_2(t)&=\frac{3m_3(t)}{2(n-t)}-\frac{2m_2(t)}{n-t}&&m_2(0)=0
				\end{align*}
			\item \alert{Erkl\"arung:} eine getroffene 3-Klausel wird entweder zur 2-Klausel oder sie wird durch den eingesetzen Wert erf\"ullt.
			\item Eine bestimmte 2-Klausel wird wie oben mit Wahrscheinlichkeit $2/(n-t)$ getroffen.
		\end{itemize}
	\end{block}
\end{frame}


\begin{frame}\frametitle{\mytitle}
	\begin{block}{Differentialgleichungen}
		\begin{itemize}
			\item Der UCP-Proze\ss\ ist eigentlich diskret.
			\item Aber im Grenzwert $n\to\infty$ k\"onnen wir $m_3(t)/n$ und $m_2(t)/n$ durch kontinuierliche Gr\"o\ss en $f_3,f_2$ ann\"ahern.
			\item Wir setzen ferner 
				\begin{align*}
					x&=\frac{t}{n}&&\mbox{und}\\
					r&=\frac{m}{n}
				\end{align*}
		\end{itemize}
	\end{block}
\end{frame}

\begin{frame}\frametitle{\mytitle}
	\begin{block}{Differentialgleichungen (fortgesetzt)}
		\begin{itemize}
		\item Aus den Gleichungen
				\begin{align*}
					\Delta m_3(t)&=-\frac{3m_3(t)}{n-t}&m_3(0)&=m\\
					\Delta m_2(t)&=\frac{3m_3(t)}{2(n-t)}-\frac{2m_2(t)}{n-t}&m_2(0)&=0
				\end{align*}
				werden dann Differentialgleichungen
				\begin{align*}
					\frac{\dd f_3(x)}{\dd x}&=-\frac{3f_3(x)}{1-x}&f_3(0)&=r\\
					\frac{\dd f_2(x)}{\dd x}&=\frac{3f_3(x)}{2(1-x)}-\frac{2f_2(x)}{1-x}&f_2(0)&=0
				\end{align*}
		\end{itemize}
	\end{block}
\end{frame}

\begin{frame}\frametitle{\mytitle}
	\begin{block}{L\"osen der Differentialgleichungen}
		\begin{itemize}
		\item Diese Differentialgleichungen besitzen die L\"osung
			\begin{align*}
				f_3(x)&=r(1-x)^3\\
				f_2(x)&=\frac{3}{2}rx(1-x)^2
			\end{align*}
		\item Was lernen wir daraus f\"ur UCP?
		\end{itemize}
	\end{block}
\end{frame}

\begin{frame}\frametitle{\mytitle}
	\begin{block}{Unit Clauses}
		\begin{itemize}
		\item Der UCP-Algorithmus ``l\"ost'' 1-Klauseln sofort wenn sie entstehen.
		\item Solange diese also mit einer Rate von weniger als 1 Klausel pro Zeitschritt produziert werden, werden kaum Widerspr\"uche entstehen.
		\item \"Ahnlich wie oben entstehen $1$-Klauseln mit Rate
			\begin{align*}
				\Delta m_1(t)&=\frac{2m_2(t)}{2(n-t)}=\frac{m_2(t)}{n-t}.
			\end{align*}
		\item \alert{Erkl\"arung:} eine getroffene $2$-Klausel wird entweder sofort erf\"ullt oder wird zur 1-Klausel.
		\end{itemize}
	\end{block}
\end{frame}

\begin{frame}\frametitle{\mytitle}
	\begin{block}{Unit Clauses (fortgesetzt)}
		\begin{itemize}
		\item Wenn wir diese Gleichung reskalieren, erhalten wir die Bedingung
			\begin{align*}
				\frac{f_2(x)}{1-x}<1&&\mbox{f\"ur alle }x\in(0,1)
			\end{align*}
			oder
			\begin{align*}
				f_2(x)=\frac{3}{2}rx(1-x)^2<1-x&&\mbox{f\"ur alle }x\in(0,1).
			\end{align*}
		\item Diese Ungleichung ist genau dann erf\"ullt, wenn
			\begin{align*}
				r<\frac{8}{3}\quad\mbox{also}\quad m<\frac{8}{3}n.
			\end{align*}
		\end{itemize}
	\end{block}
\end{frame}

\begin{frame}\frametitle{\mytitle}
	\begin{block}{Zusammenfassung}
		\begin{itemize}
			\item Wir haben eine Anwendung von Differentialgleichungen in der Analyse von Algorithmen gesehen.
			\item Die Analyse des UCP-Algorithmus f\"uhrte auf ein System von Differentialgleichungen.
			\item An der L\"osung konnten wir Ablesen, bis zu welcher Klauseldichte das Verfahren funktioniert.
			\item \alert{Quelle:} D.~Achlioptas: Theoretical Computer Science {\bf265} (2001) 159--185.
		\end{itemize}
	\end{block}
\end{frame}

\end{document}
