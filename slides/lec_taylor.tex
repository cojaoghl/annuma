\documentclass{beamer}
\usepackage{amsmath,graphics}
\usepackage{amssymb}

\usetheme{default}
\usepackage{xcolor}

\definecolor{solarizedBase03}{HTML}{002B36}
\definecolor{solarizedBase02}{HTML}{073642}
\definecolor{solarizedBase01}{HTML}{586e75}
\definecolor{solarizedBase00}{HTML}{657b83}
\definecolor{solarizedBase0}{HTML}{839496}
\definecolor{solarizedBase1}{HTML}{93a1a1}
\definecolor{solarizedBase2}{HTML}{EEE8D5}
\definecolor{solarizedBase3}{HTML}{FDF6E3}
\definecolor{solarizedYellow}{HTML}{B58900}
\definecolor{solarizedOrange}{HTML}{CB4B16}
\definecolor{solarizedRed}{HTML}{DC322F}
\definecolor{solarizedMagenta}{HTML}{D33682}
\definecolor{solarizedViolet}{HTML}{6C71C4}
%\definecolor{solarizedBlue}{HTML}{268BD2}
\definecolor{solarizedBlue}{HTML}{134676}
\definecolor{solarizedCyan}{HTML}{2AA198}
\definecolor{solarizedGreen}{HTML}{859900}
\definecolor{myBlue}{HTML}{162DB0}%{261CA4}
\setbeamercolor*{item}{fg=myBlue}
\setbeamercolor{normal text}{fg=solarizedBase03, bg=solarizedBase3}
\setbeamercolor{alerted text}{fg=myBlue}
\setbeamercolor{example text}{fg=myBlue, bg=solarizedBase3}
\setbeamercolor*{frametitle}{fg=solarizedRed}
\setbeamercolor*{title}{fg=solarizedRed}
\setbeamercolor{block title}{fg=myBlue, bg=solarizedBase3}
\setbeameroption{hide notes}
\setbeamertemplate{note page}[plain]
\beamertemplatenavigationsymbolsempty
\usefonttheme{professionalfonts}
\usefonttheme{serif}

\usepackage{fourier}

\def\vec#1{\mathchoice{\mbox{\boldmath$\displaystyle#1$}}
{\mbox{\boldmath$\textstyle#1$}}
{\mbox{\boldmath$\scriptstyle#1$}}
{\mbox{\boldmath$\scriptscriptstyle#1$}}}
\definecolor{OwnGrey}{rgb}{0.560,0.000,0.000} % #999999
\definecolor{OwnBlue}{rgb}{0.121,0.398,0.711} % #1f64b0
\definecolor{red4}{rgb}{0.5,0,0}
\definecolor{blue4}{rgb}{0,0,0.5}
\definecolor{Blue}{rgb}{0,0,0.66}
\definecolor{LightBlue}{rgb}{0.9,0.9,1}
\definecolor{Green}{rgb}{0,0.5,0}
\definecolor{LightGreen}{rgb}{0.9,1,0.9}
\definecolor{Red}{rgb}{0.9,0,0}
\definecolor{LightRed}{rgb}{1,0.9,0.9}
\definecolor{White}{gray}{1}
\definecolor{Black}{gray}{0}
\definecolor{LightGray}{gray}{0.8}
\definecolor{Orange}{rgb}{0.1,0.2,1}
\setbeamerfont{sidebar right}{size=\scriptsize}
\setbeamercolor{sidebar right}{fg=Black}

\renewcommand{\emph}[1]{{\textcolor{solarizedRed}{\itshape #1}}}

\newcommand\tay{T}
\newcommand\dd{\mathrm d}
\newcommand\eul{\mathrm e}

\newcommand\cA{\mathcal A}
\newcommand\cB{\mathcal B}
\newcommand\cC{\mathcal C}
\newcommand\cD{\mathcal D}
\newcommand\cE{\mathcal E}
\newcommand\cF{\mathcal F}
\newcommand\cG{\mathcal G}
\newcommand\cH{\mathcal H}
\newcommand\cI{\mathcal I}
\newcommand\cJ{\mathcal J}
\newcommand\cK{\mathcal K}
\newcommand\cL{\mathcal L}
\newcommand\cM{\mathcal M}
\newcommand\cN{\mathcal N}
\newcommand\cO{\mathcal O}
\newcommand\cP{\mathcal P}
\newcommand\cQ{\mathcal Q}
\newcommand\cR{\mathcal R}
\newcommand\cS{\mathcal S}
\newcommand\cT{\mathcal T}
\newcommand\cU{\mathcal U}
\newcommand\cV{\mathcal V}
\newcommand\cW{\mathcal W}
\newcommand\cX{\mathcal X}
\newcommand\cY{\mathcal Y}
\newcommand\cZ{\mathcal Z}

\newcommand\fA{\mathfrak A}
\newcommand\fB{\mathfrak B}
\newcommand\fC{\mathfrak C}
\newcommand\fD{\mathfrak D}
\newcommand\fE{\mathfrak E}
\newcommand\fF{\mathfrak F}
\newcommand\fG{\mathfrak G}
\newcommand\fH{\mathfrak H}
\newcommand\fI{\mathfrak I}
\newcommand\fJ{\mathfrak J}
\newcommand\fK{\mathfrak K}
\newcommand\fL{\mathfrak L}
\newcommand\fM{\mathfrak M}
\newcommand\fN{\mathfrak N}
\newcommand\fO{\mathfrak O}
\newcommand\fP{\mathfrak P}
\newcommand\fQ{\mathfrak Q}
\newcommand\fR{\mathfrak R}
\newcommand\fS{\mathfrak S}
\newcommand\fT{\mathfrak T}
\newcommand\fU{\mathfrak U}
\newcommand\fV{\mathfrak V}
\newcommand\fW{\mathfrak W}
\newcommand\fX{\mathfrak X}
\newcommand\fY{\mathfrak Y}
\newcommand\fZ{\mathfrak Z}

\newcommand\fa{\mathfrak a}
\newcommand\fb{\mathfrak b}
\newcommand\fc{\mathfrak c}
\newcommand\fd{\mathfrak d}
\newcommand\fe{\mathfrak e}
\newcommand\ff{\mathfrak f}
\newcommand\fg{\mathfrak g}
\newcommand\fh{\mathfrak h}
%\newcommand\fi{\mathfrak i}
\newcommand\fj{\mathfrak j}
\newcommand\fk{\mathfrak k}
\newcommand\fl{\mathfrak l}
\newcommand\fm{\mathfrak m}
\newcommand\fn{\mathfrak n}
\newcommand\fo{\mathfrak o}
\newcommand\fp{\mathfrak p}
\newcommand\fq{\mathfrak q}
\newcommand\fr{\mathfrak r}
\newcommand\fs{\mathfrak s}
\newcommand\ft{\mathfrak t}
\newcommand\fu{\mathfrak u}
\newcommand\fv{\mathfrak v}
\newcommand\fw{\mathfrak w}
\newcommand\fx{\mathfrak x}
\newcommand\fy{\mathfrak y}
\newcommand\fz{\mathfrak z}

\newcommand\vA{\vec A}
\newcommand\vB{\vec B}
\newcommand\vC{\vec C}
\newcommand\vD{\vec D}
\newcommand\vE{\vec E}
\newcommand\vF{\vec F}
\newcommand\vG{\vec G}
\newcommand\vH{\vec H}
\newcommand\vI{\vec I}
\newcommand\vJ{\vec J}
\newcommand\vK{\vec K}
\newcommand\vL{\vec L}
\newcommand\vM{\vec M}
\newcommand\vN{\vec N}
\newcommand\vO{\vec O}
\newcommand\vP{\vec P}
\newcommand\vQ{\vec Q}
\newcommand\vR{\vec R}
\newcommand\vS{\vec S}
\newcommand\vT{\vec T}
\newcommand\vU{\vec U}
\newcommand\vV{\vec V}
\newcommand\vW{\vec W}
\newcommand\vX{\vec X}
\newcommand\vY{\vec Y}
\newcommand\vZ{\vec Z}

\newcommand\va{\vec a}
\newcommand\vb{\vec b}
\newcommand\vc{\vec c}
\newcommand\vd{\vec d}
\newcommand\ve{\vec e}
\newcommand\vf{\vec f}
\newcommand\vg{\vec g}
\newcommand\vh{\vec h}
\newcommand\vi{\vec i}
\newcommand\vj{\vec j}
\newcommand\vk{\vec k}
\newcommand\vl{\vec l}
\newcommand\vm{\vec m}
\newcommand\vn{\vec n}
\newcommand\vo{\vec o}
\newcommand\vp{\vec p}
\newcommand\vq{\vec q}
\newcommand\vr{\vec r}
\newcommand\vs{\vec s}
\newcommand\vt{\vec t}
\newcommand\vu{\vec u}
\newcommand\vv{\vec v}
\newcommand\vw{\vec w}
\newcommand\vx{\vec x}
\newcommand\vy{\vec y}
\newcommand\vz{\vec z}

\renewcommand\AA{\mathbb A}
\newcommand\NN{\mathbb N}
\newcommand\ZZ{\mathbb Z}
\newcommand\PP{\mathbb P}
\newcommand\QQ{\mathbb Q}
\newcommand\RR{\mathbb R}
\newcommand\RRpos{\mathbb R_{\geq0}}
\renewcommand\SS{\mathbb S}
\newcommand\CC{\mathbb C}

\newcommand{\ord}{\mathrm{ord}}
\newcommand{\id}{\mathrm{id}}
\newcommand{\pr}{\mathrm{P}}
\newcommand{\Vol}{\mathrm{vol}}
\newcommand\norm[1]{\left\|{#1}\right\|} 
\newcommand\sign{\mathrm{sign}}
\newcommand{\eps}{\varepsilon}
\newcommand{\abs}[1]{\left|#1\right|}
\newcommand\bc[1]{\left({#1}\right)} 
\newcommand\cbc[1]{\left\{{#1}\right\}} 
\newcommand\bcfr[2]{\bc{\frac{#1}{#2}}} 
\newcommand{\bck}[1]{\left\langle{#1}\right\rangle} 
\newcommand\brk[1]{\left\lbrack{#1}\right\rbrack} 
\newcommand\scal[2]{\bck{{#1},{#2}}} 
\newcommand{\vecone}{\mathbb{1}}
\newcommand{\tensor}{\otimes}
\newcommand{\diag}{\mathrm{diag}}
\newcommand{\ggt}{\mathrm{ggT}}
\newcommand{\kgv}{\mathrm{kgV}}
\newcommand{\trans}{\top}

\newcommand{\Karonski}{Karo\'nski}
\newcommand{\Erdos}{Erd\H{o}s}
\newcommand{\Renyi}{R\'enyi}
\newcommand{\Lovasz}{Lov\'asz}
\newcommand{\Juhasz}{Juh\'asz}
\newcommand{\Bollobas}{Bollob\'as}
\newcommand{\Furedi}{F\"uredi}
\newcommand{\Komlos}{Koml\'os}
\newcommand{\Luczak}{\L uczak}
\newcommand{\Kucera}{Ku\v{c}era}
\newcommand{\Szemeredi}{Szemer\'edi}

\renewcommand{\ae}{\"a}
\renewcommand{\oe}{\"o}
\newcommand{\ue}{\"u}
\newcommand{\Ae}{\"A}
\newcommand{\Oe}{\"O}
\newcommand{\Ue}{\"U}

\newcommand{\im}{\mathrm{im}}
\newcommand{\rrk}{\mathrm{zrg}}
\newcommand{\crk}{\mathrm{srg}}
\newcommand{\rk}{\mathrm{rg}}
\newcommand{\GL}{\mathrm{GL}}
\newcommand{\SL}{\mathrm{SL}}
\newcommand{\SO}{\mathrm{SO}}
\newcommand{\nul}{\mathrm{nul}}
\newcommand{\eig}{\mathrm{eig}}

\newcommand{\mytitle}{Die Taylorformel}

\title[Annuma]{\mytitle}
\author[Amin Coja-Oghlan]{Amin Coja-Oghlan}
\institute[Frankfurt]{JWGUFFM}
\date{}

\begin{document}

\frame[plain]{\titlepage}

\begin{frame}\frametitle{\mytitle}
	\begin{block}{Worum geht es?}
		\begin{itemize}
			\item Mit Hilfe der Ableitung k\oe nnen wir Funktionen lokal beschreiben.
			\item Wir lernen h\oe here Ableitungen und Approximationen von Funktionen durch Polynome kennen.
			\item Ultimativ f\ue hren diese \Ue berlegungen auf Reihenentwicklungen.
		\end{itemize}
	\end{block}
\end{frame}

\begin{frame}\frametitle{\mytitle}
	\begin{block}{H\oe here Ableitungen}
		\begin{itemize}
			\item Wenn $f:[a,b]\to\RR$ differenzierbar ist, erhalten wir die Ableitung
				\begin{align*}
					f':[a,b]\to\RR,\qquad x\mapsto f'(x).
				\end{align*}
			\item Diese kann (aber mu\ss\ nicht) selbst differenzierbar sein.
			\item Dann erhalten wir die zweite Ableitung
				\begin{align*}
					f'':[a,b]\to\RR,\qquad x\mapsto f''(x).
				\end{align*}
			\item Entsprechend kann eine Funktion eine dritte, vierte etc.\ Ableitung besitzen.
		\end{itemize}
	\end{block}
\end{frame}

\begin{frame}\frametitle{\mytitle}
	\begin{block}{H\oe here Ableitungen (fortgesetzt)}
		\begin{itemize}
			\item Eine Funktion, die $k$ Ableitungen besitzt, nennen wir \emph{$k$-mal differenzierbar}.
			\item Die $k$-te Ableitung wird dann auch durch $f^{[k]}$ bezeichnet.
			\item \alert{Konvention:} $f^{[0]}=f$.
			\item Die Funktion ist \emph{$k$-mal stetig differenzierbar}, wenn sie $k$-mal differenzierbar und zus\ae tzlich $f^{[k]}$ stetig ist.
		\end{itemize}
	\end{block}
\end{frame}

\begin{frame}\frametitle{\mytitle}
	\begin{block}{H\oe here Ableitungen (fortgesetzt)}
		\begin{itemize}
			\item Auf dem Weg \ue ber h\oe here Ableitungen kann man lokale Maxima und Minima ausfindig machen.
			\item Ist $f:[a,b]\to\RR$ zweimal differenzierbar und $p\in(a,b)$ ein Punkt mit
				\begin{align*}
					f'(p)=0\quad\mbox{und}\quad f''(p)>0,
				\end{align*}
				so handelt es sich um ein lokales Minimum.
			\item Ist $f:[a,b]\to\RR$ zweimal differenzierbar und $p\in(a,b)$ ein Punkt mit
				\begin{align*}
					f'(p)=0\quad\mbox{und}\quad f''(p)<0,
				\end{align*}
				so handelt es sich um ein lokales Maximum.
		\end{itemize}
	\end{block}
\end{frame}

\begin{frame}\frametitle{\mytitle}
	\begin{block}{Das Taylorpolynom}
		\begin{itemize}
			\item Sei $f:[a,b]\to\RR$ $k$-mal differenzierbar und sei $z\in(a,b)$.
			\item Dann nennen wir die Funktion
				\begin{align*}
					T_{f,k,z}(x)&=\sum_{i=0}^k\frac{f^{[i]}(z)}{i!}(x-z)^i
				\end{align*}
				das \emph{$k$-te Taylorpolynom von $f$ in $z$.}
			\item \alert{Erinnerung:} $i!=1\cdot2\cdot3\cdot\enspace\cdots\enspace\cdot i$ \hfill(``$i$-Fakult\ae t'').
		\end{itemize}
	\end{block}
\end{frame}

\begin{frame}\frametitle{\mytitle}
	\begin{block}{Satz von Taylor (``Taylorformel'')}
		Sei $f:[a,b]\to\RR$ eine Funktion, die $(k+1)$-mal stetig differenzierbar ist.
		Dann gilt f\ue r alle $x,z\in(a,b)$ die Formel
		\begin{align*}
			f(x)&=T_{f,k,z}(x)+\int_z^x\frac{(x-y)^k}{k!}f^{[k+1]}(y)\dd y.
		\end{align*}
	\end{block}
\end{frame}

\begin{frame}\frametitle{\mytitle}
	\begin{block}{Restglieddarstellung nach Lagrange}
		Sei $f:[a,b]\to\RR$ eine Funktion, die $(k+1)$-mal stetig differenzierbar ist.
		Dann gibt es zu $x,z\in(a,b)$ stets eine Zahl $s$ mit
		\begin{align*}
			f(x)&=T_{f,k,z}(x)+\frac{(x-z)^{k+1}}{(k+1)!}f^{[k+1]}(s).
		\end{align*}
		Die Zahl $s$ liegt zwischen $x$ und $z$, d.h.
		\begin{itemize}
			\item $x\leq s\leq z$, falls $x< z$, bzw.
			\item $z\leq s\leq x$, falls $x> z$.
		\end{itemize}
	\end{block}
\end{frame}

\begin{frame}\frametitle{\mytitle}
	\begin{block}{Taylorreihen}
	\begin{itemize}
		\item Eine Funktion $f:[a,b]\to\RR$, die $k$-mal differenzierbar ist f\ue r jedes $k\in\NN$, hei\ss t \emph{unendlich oft differenzierbar}.
		\item F\ue r eine solche Funktion und einen Punkt $z\in(a,b)$ definieren wir die \emph{Taylorreihe} als
			\begin{align*}
				T_{f,z}(x)&=\sum_{i=0}^\infty\frac{(x-z)^i}{i!}f^{[i]}(z).
			\end{align*}
	\end{itemize}
	\end{block}
\end{frame}

\begin{frame}\frametitle{\mytitle}
	\begin{block}{Taylorreihen (fortgesetzt)}
	\begin{itemize}
		\item \itshape Es ist nicht garantiert, da\ss\ diese Reihe konvergiert!
		\item Selbst wenn, ist nicht garantiert, da\ss\ sie gegen die Funktion $f$ konvergiert!
		\item Um die Konvergenz der Taylorreihe gegen die Funktion $f$ nachzuweisen, ist eine Analyse der Konvergenzeigenschaften der Reihe notwenedig.
		\item Au\ss erdem ist es notwendig, den Fehlerterm in der Taylorformel abzusch\ae tzen.
	\end{itemize}
	\end{block}
\end{frame}

\begin{frame}\frametitle{\mytitle}
	\begin{block}{Beispiel: Exponentialfunktion}
	\begin{itemize}
		\item Die Exponentialfunktion $\exp:\RR\to\RR$ hat die Eigenschaften
			\begin{align*}
				\exp(0)=1&&\exp'(x)=\exp(x).
			\end{align*}
		\item Folglich gilt f\ue r alle $k\in\NN$, da\ss
			\begin{align*}
				\exp^{[k]}(0)=\exp(0)=1.
			\end{align*}
		\item Die Taylorreihe der Exponentialfunktion in $z=0$ lautet also
			\begin{align*}
				\sum_{i=0}^\infty\frac{x^i}{i!}=1+x+\frac{x^2}{2}+\frac{x^3}{6}+\frac{x^4}{24}+\cdots
			\end{align*}
		\item Die Reihe konvergiert f\ue r alle $x\in\RR$.
	\end{itemize}
	\end{block}
\end{frame}

\begin{frame}\frametitle{\mytitle}
	\begin{block}{Beispiel: Exponentialfunktion (fortgesetzt)}
	\begin{itemize}
		\item Die Restglieddarstellung nach Lagrange zeigt ferner, da\ss
			\begin{align*}
				\exp(x)-\sum_{i=0}^k\frac{x^i}{i!}\leq\frac{x^{k+1}}{(k+1)!}\cdot\exp(|x|).
			\end{align*}
		\item Weil der Nenner unbeschr\ae nkt ist, gilt f\ue r jedes $x\in\RR$
			\begin{align*}
				\lim_{k\to\infty}\frac{x^{k+1}}{(k+1)!}\cdot\exp(|x|)=0.
			\end{align*}
	\end{itemize}
	\end{block}
\end{frame}

\begin{frame}\frametitle{\mytitle}
	\begin{block}{Beispiel: Exponentialfunktion (fortgesetzt)}
	\begin{itemize}
		\item Daher konvergiert die Taylorreihe f\ue r alle $x\in\RR$ gegen die Exponentialfunktion:
			$$ \exp(x)=\sum_{i=0}^\infty\frac{x^i}{i!} .$$
		\item Insbesondere erhalten wir $$\eul=\exp(1)=\sum_{i=0}^\infty\frac{1}{i!}.$$
	\end{itemize}
	\end{block}
\end{frame}

\begin{frame}\frametitle{\mytitle}
	\begin{block}{Beispiel: Sinus und Cosinus}
	\begin{itemize}
		\item Sinus und Cosinus erf\ue llen die Gleichungen
			\begin{align*}
				\sin(0)&=0&\sin'(x)&=\cos(x)\\
				\cos(0)&=1&\cos'(x)&=-\sin(x).
			\end{align*}
		\item Daraus leiten wir her:
			\begin{align*}
				\cos^{[2k]}(0)&=(-1)^k&\cos^{[2k+1]}(0)&=0\\
				\sin^{[2k+1]}(0)&=(-1)^{k+1}&\sin^{[2k]}(0)&=(-1)^k
			\end{align*}
		\item Die Taylorreihen von Cosinus und Sinus lauten daher
			\begin{align*}
				\sum_{k=0}^\infty\frac{(-1)^k}{(2k)!}x^{2k}&&
				\sum_{k=0}^\infty\frac{(-1)^{k}}{(2k+1)!}x^{2k+1}
			\end{align*}
	\end{itemize}
	\end{block}
\end{frame}

\begin{frame}\frametitle{\mytitle}
	\begin{block}{Beispiel: Sinus und Cosinus}
	\begin{itemize}
		\item Diese beiden Reihen konveriergen f\ue r alle $x\in\RR$.
		\item Mittels der Restglieddarstellung von Lagrange sieht man ferner, da\ss\ die beiden Reihen gegen $\cos(x)$ und $\sin(x)$ konvergieren.
		\item Wir erhalten daher die Reihenentwicklungen
			\begin{align*}
				\cos(x)=\sum_{k=0}^\infty\frac{(-1)^k}{(2k)!}x^{2k}&&
				\sin(x)=\sum_{k=0}^\infty\frac{(-1)^{k}}{(2k+1)!}x^{2k+1}
			\end{align*}
	\end{itemize}
	\end{block}
\end{frame}

\begin{frame}\frametitle{\mytitle}
	\begin{block}{Beispiel: die Mercatorreihe}
	\begin{itemize}
		\item Die Funktion $f:x\mapsto\ln(1+x)$ ist auf $(-1,1)$ unendlich oft differenzierbar.
		\item Aufgrund der Formel $\ln'(x)=1/x$ sind die Ableitungen
			\begin{align*}
				f^{[0]}(x)&=f(x)=\ln(1+x)\\
				f^{[1]}(x)&=f'(x)=\frac{1}{1+x}\\
				f^{[2]}(x)&=f''(x)=\frac{-1}{(1+x)^2}\\
				f^{[3]}(x)&=f'''(x)=\frac{2(1+x)}{(1+x)^4}=\frac{2}{(1+x)^3}\\
						  &\vdots\\
				f^{[k]}(x)&=\frac{(-1)^{k+1}(k-1)!}{(1+x)^k}
			\end{align*}
	\end{itemize}
	\end{block}
\end{frame}

\begin{frame}\frametitle{\mytitle}
	\begin{block}{Beispiel: die Mercatorreihe}
	\begin{itemize}
		\item Die Taylorreihe im Nullpunkt lautet also
			\begin{align*}
				\sum_{i=1}^\infty(-1)^{i+1}\frac{x^i}{i}
			\end{align*}
		\item Diese Reihe hei\ss t die \emph{Mercatorreihe}.
		\item Sie konvergiert f\ue r alle $-1<x\leq 1$ gegen $f(x)$.
		\item Wir erhalten also die Reihendarstellung
			\begin{align*}
				\ln(1+x)&=\sum_{i=1}^\infty(-1)^{i+1}\frac{x^i}{i}&&(-1<x\leq1).
			\end{align*}
		\item Insbesondere gilt $\ln(2)=\sum_{i=1}^\infty(-1)^{i+1}/i\enspace .$
	\end{itemize}
	\end{block}
\end{frame}

\begin{frame}\frametitle{\mytitle}
	\begin{block}{Zusammenfassung}
	\begin{itemize}
		\item Mit Hilfe der Taylorformel k\oe nnen wir viele Funktionen durch Polynome approximieren.
		\item Die Approximationsg\ue te wird durch einen Fehlerterm dargestellt.
		\item H\ae ufig ist die Restglieddarstellung nach Lagrange hilfreich.
		\item In der Konsequenz k\oe nnen einige wichtige Funktionen in Taylorreihen entwickelt werden.
	\end{itemize}
	\end{block}
\end{frame}
\end{document}
