\documentclass{beamer}
\usepackage{amsmath,graphics}
\usepackage{amssymb}

\usetheme{default}
\usepackage{xcolor}

\definecolor{solarizedBase03}{HTML}{002B36}
\definecolor{solarizedBase02}{HTML}{073642}
\definecolor{solarizedBase01}{HTML}{586e75}
\definecolor{solarizedBase00}{HTML}{657b83}
\definecolor{solarizedBase0}{HTML}{839496}
\definecolor{solarizedBase1}{HTML}{93a1a1}
\definecolor{solarizedBase2}{HTML}{EEE8D5}
\definecolor{solarizedBase3}{HTML}{FDF6E3}
\definecolor{solarizedYellow}{HTML}{B58900}
\definecolor{solarizedOrange}{HTML}{CB4B16}
\definecolor{solarizedRed}{HTML}{DC322F}
\definecolor{solarizedMagenta}{HTML}{D33682}
\definecolor{solarizedViolet}{HTML}{6C71C4}
%\definecolor{solarizedBlue}{HTML}{268BD2}
\definecolor{solarizedBlue}{HTML}{134676}
\definecolor{solarizedCyan}{HTML}{2AA198}
\definecolor{solarizedGreen}{HTML}{859900}
\definecolor{myBlue}{HTML}{162DB0}%{261CA4}
\setbeamercolor*{item}{fg=myBlue}
\setbeamercolor{normal text}{fg=solarizedBase03, bg=solarizedBase3}
\setbeamercolor{alerted text}{fg=myBlue}
\setbeamercolor{example text}{fg=myBlue, bg=solarizedBase3}
\setbeamercolor*{frametitle}{fg=solarizedRed}
\setbeamercolor*{title}{fg=solarizedRed}
\setbeamercolor{block title}{fg=myBlue, bg=solarizedBase3}
\setbeameroption{hide notes}
\setbeamertemplate{note page}[plain]
\beamertemplatenavigationsymbolsempty
\usefonttheme{professionalfonts}
\usefonttheme{serif}

\usepackage{fourier}

\def\vec#1{\mathchoice{\mbox{\boldmath$\displaystyle#1$}}
{\mbox{\boldmath$\textstyle#1$}}
{\mbox{\boldmath$\scriptstyle#1$}}
{\mbox{\boldmath$\scriptscriptstyle#1$}}}
\definecolor{OwnGrey}{rgb}{0.560,0.000,0.000} % #999999
\definecolor{OwnBlue}{rgb}{0.121,0.398,0.711} % #1f64b0
\definecolor{red4}{rgb}{0.5,0,0}
\definecolor{blue4}{rgb}{0,0,0.5}
\definecolor{Blue}{rgb}{0,0,0.66}
\definecolor{LightBlue}{rgb}{0.9,0.9,1}
\definecolor{Green}{rgb}{0,0.5,0}
\definecolor{LightGreen}{rgb}{0.9,1,0.9}
\definecolor{Red}{rgb}{0.9,0,0}
\definecolor{LightRed}{rgb}{1,0.9,0.9}
\definecolor{White}{gray}{1}
\definecolor{Black}{gray}{0}
\definecolor{LightGray}{gray}{0.8}
\definecolor{Orange}{rgb}{0.1,0.2,1}
\setbeamerfont{sidebar right}{size=\scriptsize}
\setbeamercolor{sidebar right}{fg=Black}

\renewcommand{\emph}[1]{{\textcolor{solarizedRed}{\itshape #1}}}

\newcommand\tay{T}
\newcommand\dd{\mathrm d}
\newcommand\eul{\mathrm e}

\newcommand\cA{\mathcal A}
\newcommand\cB{\mathcal B}
\newcommand\cC{\mathcal C}
\newcommand\cD{\mathcal D}
\newcommand\cE{\mathcal E}
\newcommand\cF{\mathcal F}
\newcommand\cG{\mathcal G}
\newcommand\cH{\mathcal H}
\newcommand\cI{\mathcal I}
\newcommand\cJ{\mathcal J}
\newcommand\cK{\mathcal K}
\newcommand\cL{\mathcal L}
\newcommand\cM{\mathcal M}
\newcommand\cN{\mathcal N}
\newcommand\cO{\mathcal O}
\newcommand\cP{\mathcal P}
\newcommand\cQ{\mathcal Q}
\newcommand\cR{\mathcal R}
\newcommand\cS{\mathcal S}
\newcommand\cT{\mathcal T}
\newcommand\cU{\mathcal U}
\newcommand\cV{\mathcal V}
\newcommand\cW{\mathcal W}
\newcommand\cX{\mathcal X}
\newcommand\cY{\mathcal Y}
\newcommand\cZ{\mathcal Z}

\newcommand\fA{\mathfrak A}
\newcommand\fB{\mathfrak B}
\newcommand\fC{\mathfrak C}
\newcommand\fD{\mathfrak D}
\newcommand\fE{\mathfrak E}
\newcommand\fF{\mathfrak F}
\newcommand\fG{\mathfrak G}
\newcommand\fH{\mathfrak H}
\newcommand\fI{\mathfrak I}
\newcommand\fJ{\mathfrak J}
\newcommand\fK{\mathfrak K}
\newcommand\fL{\mathfrak L}
\newcommand\fM{\mathfrak M}
\newcommand\fN{\mathfrak N}
\newcommand\fO{\mathfrak O}
\newcommand\fP{\mathfrak P}
\newcommand\fQ{\mathfrak Q}
\newcommand\fR{\mathfrak R}
\newcommand\fS{\mathfrak S}
\newcommand\fT{\mathfrak T}
\newcommand\fU{\mathfrak U}
\newcommand\fV{\mathfrak V}
\newcommand\fW{\mathfrak W}
\newcommand\fX{\mathfrak X}
\newcommand\fY{\mathfrak Y}
\newcommand\fZ{\mathfrak Z}

\newcommand\fa{\mathfrak a}
\newcommand\fb{\mathfrak b}
\newcommand\fc{\mathfrak c}
\newcommand\fd{\mathfrak d}
\newcommand\fe{\mathfrak e}
\newcommand\ff{\mathfrak f}
\newcommand\fg{\mathfrak g}
\newcommand\fh{\mathfrak h}
%\newcommand\fi{\mathfrak i}
\newcommand\fj{\mathfrak j}
\newcommand\fk{\mathfrak k}
\newcommand\fl{\mathfrak l}
\newcommand\fm{\mathfrak m}
\newcommand\fn{\mathfrak n}
\newcommand\fo{\mathfrak o}
\newcommand\fp{\mathfrak p}
\newcommand\fq{\mathfrak q}
\newcommand\fr{\mathfrak r}
\newcommand\fs{\mathfrak s}
\newcommand\ft{\mathfrak t}
\newcommand\fu{\mathfrak u}
\newcommand\fv{\mathfrak v}
\newcommand\fw{\mathfrak w}
\newcommand\fx{\mathfrak x}
\newcommand\fy{\mathfrak y}
\newcommand\fz{\mathfrak z}

\newcommand\vA{\vec A}
\newcommand\vB{\vec B}
\newcommand\vC{\vec C}
\newcommand\vD{\vec D}
\newcommand\vE{\vec E}
\newcommand\vF{\vec F}
\newcommand\vG{\vec G}
\newcommand\vH{\vec H}
\newcommand\vI{\vec I}
\newcommand\vJ{\vec J}
\newcommand\vK{\vec K}
\newcommand\vL{\vec L}
\newcommand\vM{\vec M}
\newcommand\vN{\vec N}
\newcommand\vO{\vec O}
\newcommand\vP{\vec P}
\newcommand\vQ{\vec Q}
\newcommand\vR{\vec R}
\newcommand\vS{\vec S}
\newcommand\vT{\vec T}
\newcommand\vU{\vec U}
\newcommand\vV{\vec V}
\newcommand\vW{\vec W}
\newcommand\vX{\vec X}
\newcommand\vY{\vec Y}
\newcommand\vZ{\vec Z}

\newcommand\va{\vec a}
\newcommand\vb{\vec b}
\newcommand\vc{\vec c}
\newcommand\vd{\vec d}
\newcommand\ve{\vec e}
\newcommand\vf{\vec f}
\newcommand\vg{\vec g}
\newcommand\vh{\vec h}
\newcommand\vi{\vec i}
\newcommand\vj{\vec j}
\newcommand\vk{\vec k}
\newcommand\vl{\vec l}
\newcommand\vm{\vec m}
\newcommand\vn{\vec n}
\newcommand\vo{\vec o}
\newcommand\vp{\vec p}
\newcommand\vq{\vec q}
\newcommand\vr{\vec r}
\newcommand\vs{\vec s}
\newcommand\vt{\vec t}
\newcommand\vu{\vec u}
\newcommand\vv{\vec v}
\newcommand\vw{\vec w}
\newcommand\vx{\vec x}
\newcommand\vy{\vec y}
\newcommand\vz{\vec z}

\renewcommand\AA{\mathbb A}
\newcommand\NN{\mathbb N}
\newcommand\ZZ{\mathbb Z}
\newcommand\PP{\mathbb P}
\newcommand\QQ{\mathbb Q}
\newcommand\RR{\mathbb R}
\newcommand\RRpos{\mathbb R_{\geq0}}
\renewcommand\SS{\mathbb S}
\newcommand\CC{\mathbb C}

\newcommand{\ord}{\mathrm{ord}}
\newcommand{\id}{\mathrm{id}}
\newcommand{\pr}{\mathrm{P}}
\newcommand{\Vol}{\mathrm{vol}}
\newcommand\norm[1]{\left\|{#1}\right\|} 
\newcommand\sign{\mathrm{sign}}
\newcommand{\eps}{\varepsilon}
\newcommand{\abs}[1]{\left|#1\right|}
\newcommand\bc[1]{\left({#1}\right)} 
\newcommand\cbc[1]{\left\{{#1}\right\}} 
\newcommand\bcfr[2]{\bc{\frac{#1}{#2}}} 
\newcommand{\bck}[1]{\left\langle{#1}\right\rangle} 
\newcommand\brk[1]{\left\lbrack{#1}\right\rbrack} 
\newcommand\scal[2]{\bck{{#1},{#2}}} 
\newcommand{\vecone}{\mathbb{1}}
\newcommand{\tensor}{\otimes}
\newcommand{\diag}{\mathrm{diag}}
\newcommand{\ggt}{\mathrm{ggT}}
\newcommand{\kgv}{\mathrm{kgV}}
\newcommand{\trans}{\top}

\newcommand{\Karonski}{Karo\'nski}
\newcommand{\Erdos}{Erd\H{o}s}
\newcommand{\Renyi}{R\'enyi}
\newcommand{\Lovasz}{Lov\'asz}
\newcommand{\Juhasz}{Juh\'asz}
\newcommand{\Bollobas}{Bollob\'as}
\newcommand{\Furedi}{F\"uredi}
\newcommand{\Komlos}{Koml\'os}
\newcommand{\Luczak}{\L uczak}
\newcommand{\Kucera}{Ku\v{c}era}
\newcommand{\Szemeredi}{Szemer\'edi}

\renewcommand{\ae}{\"a}
\renewcommand{\oe}{\"o}
\newcommand{\ue}{\"u}
\newcommand{\Ae}{\"A}
\newcommand{\Oe}{\"O}
\newcommand{\Ue}{\"U}

\newcommand{\im}{\mathrm{im}}
\newcommand{\rrk}{\mathrm{zrg}}
\newcommand{\crk}{\mathrm{srg}}
\newcommand{\rk}{\mathrm{rg}}
\newcommand{\GL}{\mathrm{GL}}
\newcommand{\SL}{\mathrm{SL}}
\newcommand{\SO}{\mathrm{SO}}
\newcommand{\nul}{\mathrm{nul}}
\newcommand{\eig}{\mathrm{eig}}

\newcommand{\mytitle}{Ableitungen im Mehrdimensionalen}

\title[Annuma]{\mytitle}
\author[Amin Coja-Oghlan]{Amin Coja-Oghlan}
\institute[Frankfurt]{JWGUFFM}
\date{}

\begin{document}

\frame[plain]{\titlepage}

\begin{frame}\frametitle{\mytitle}
	\begin{block}{Worum geht es?}
		\begin{itemize}
			\item Wir haben die Ableitung einer Funktion $f:A\to\RR$ kennengelernt.
			\item In vielen Anwendungen begegnen aber Funktionen von mehreren Ver\ae nderlichen.
			\item Ebenfalls begegnen h\ae ufig vektorwertige Funktionen.
			\item In diesem Abschnitt verallgemeinern wir den Ableitungsbegriff entsprechend.
		\end{itemize}
	\end{block}
\end{frame}

\begin{frame}\frametitle{\mytitle}
	\begin{block}{Partielle Ableitungen}
		\begin{itemize}
			\item Sei $f(x,y)$ eine reellwertige Funktion von zwei Variablen $x,y$.
			\item Die \emph{partielle Ableitung}
				\begin{align*}
				\frac{\partial f}{\partial x}
				\end{align*}
				ist einfach die Ableitung von der Funktion nach $x$, wenn wir $y$ \alert{als Konstante} behandeln.
			\item Entsprechend ist
				\begin{align*}
				\frac{\partial f}{\partial y}
				\end{align*}
				die Ableitung von $f$ nach $y$, wenn wir $x$ als Konstante ansehen.
		\end{itemize}
	\end{block}
\end{frame}

\begin{frame}\frametitle{\mytitle}
	\begin{block}{Beispiel}
		\begin{itemize}
			\item Sei $f(x,y)=x^2+3xy+y^4$.
			\item Die partiellen Ableitungen sind
\begin{align*}
	\frac{\partial f}{\partial x}&=2x+3y\\
	\frac{\partial f}{\partial x}&=3x+4y^3.
\end{align*}
			\item Wir leiten also nach der einen Variablen ab, und behandeln die andere wie einen konstanten Wert.
		\end{itemize}
	\end{block}
\end{frame}

\begin{frame}\frametitle{\mytitle}
	\begin{block}{Funktionen von mehr als zwei Variablen}
		\begin{itemize}
			\item Der Begriff der partiellen Ableitung l\ae\ss t sich auf Funktionen von mehr als zwei Variablen verallgemeinern.
			\item Sei $f(x_1,\ldots,x_n)$ eine reellwertige Funktion.
			\item Dann ist
				\begin{align*}
				\frac{\partial f}{\partial x_i}
				\end{align*}
				die Ableitung von $f$ nach der Variablen $x_i$, wobei wir die Variablen $x_j$ mit $j\neq i$ als Konstanten behandeln.
		\end{itemize}
	\end{block}
\end{frame}

\begin{frame}\frametitle{\mytitle}
	\begin{block}{Beispiel}
		\begin{itemize}
			\item Sei $g(x,y,z)=\exp(x^2+y^2+xyz)$.
			\item Die partiellen Ableitungen lauten
				\begin{align*}
					\frac{\partial g}{\partial x}&=\exp(x^2+y^2+xyz)\cdot\frac{\partial}{\partial x}(x^2+y^2+xyz)\\
												 &=\exp(x^2+y^2+xyz)\cdot\bc{2x+yz}\\
					\frac{\partial g}{\partial y}&=\exp(x^2+y^2+xyz)\cdot\frac{\partial}{\partial y}(x^2+y^2+xyz)\\
												 &=\exp(x^2+y^2+xyz)\cdot(2y+xz)\\
					\frac{\partial g}{\partial z}&=\exp(x^2+y^2+xyz)\cdot\frac{\partial}{\partial z}(x^2+y^2+xyz)\\
												 &=\exp(x^2+y^2+xyz)\cdot xy
				\end{align*}
		\end{itemize}
	\end{block}
\end{frame}

\begin{frame}\frametitle{\mytitle}
	\begin{block}{Auswerten der partiellen Ableitungen}
		\begin{itemize}
			\item Sei $f(x_1,\ldots,x_n)$ eine reellwertige Funktion.
			\item Werden die partiellen Ableitungen an einer bestimmten Stelle ausgewertet, verwendet man die Schreibweise
				\begin{align*}
					\frac{\partial f}{\partial x_i}\bigg|_{x_1=c_1,\ldots,x_n=c_n}
				\end{align*}
			\item Wir bestimmen also zun\ae chst die partielle Ableitung von $f$. 
			\item Anschlie\ss end setzen wir f\ue r $x_1,\ldots,x_n$ die Werte $c_1,\ldots,c_n$ ein.
		\end{itemize}
	\end{block}
\end{frame}

\begin{frame}\frametitle{\mytitle}
	\begin{block}{Beispiel}
		\begin{itemize}
			\item Sei $f(x,y)=\ln(x^2+y^2)$.
			\item Wir bestimmen
				\begin{align*}
					\frac{\partial f}{\partial x}\bigg|_{x=1,y=0}
				\end{align*}
			\item Dazu berechnen wir zun\ae chst die partielle Ableitung:
				\begin{align*}
					\frac{\partial f}{\partial x}&=\frac{2x}{x^2+y^2}.
				\end{align*}
			\item Jetzt setzen wir f\ue r $x$ den Wert $1$ und f\ue r $y$ den Wert $0$ ein:
\begin{align*}
					\frac{\partial f}{\partial x}\bigg|_{x=1,y=0}=\frac{2\cdot 1}{1^2+0^2}=2.
				\end{align*}
		\end{itemize}
	\end{block}
\end{frame}

\begin{frame}\frametitle{\mytitle}
	\begin{block}{Vektorwertige Funktionen}
		\begin{itemize}
			\item Mitunter treten Funktionen $f(x_1,\ldots,x_n)$ auf, deren Werte Vektoren in $\RR^m$ und nicht einfache Zahlen sind.
			\item In diesem Fall zerlegen wir die Funktion $f$ in ihre $m$ Komponenten:
				\begin{align*}
					f(x_1,\ldots,x_n)&=\begin{pmatrix}
						f_1(x_1,\ldots,x_n)\\\vdots\\f_m(x_1,\ldots,x_n)
					\end{pmatrix}.
				\end{align*}
			\item Die einzelnen Funktion $f_i(x_1,\ldots,x_n)$ sind reellwertig.
		\end{itemize}
	\end{block}
\end{frame}

\begin{frame}\frametitle{\mytitle}
	\begin{block}{Vektorwertige Funktionen}
		\begin{itemize}
			\item Daher k\oe nnen wir die partiellen Ableitungen
				\begin{align*}
				\frac{\partial f_i}{\partial x_j}
				\end{align*}
				bilden.
			\item Die Matrix, 
\begin{align*}
	Df(x)&=\bc{\frac{\partial f_i}{\partial x_j}}_{i=1,\ldots m;j=1,\ldots,n}
\end{align*}
die sich aus diesen partiellen Ableitungen zusammensetzt, wird die \emph{Jacobimatrix} von $f$ genannt.
		\end{itemize}
	\end{block}
\end{frame}

\begin{frame}\frametitle{\mytitle}
	\begin{block}{Beispiel}
		\begin{itemize}
			\item Sei $f:\RR^2\to\RR^2$, $(x,y)\mapsto\binom{\cos(x+2y)}{\sin(xy)}$.
			\item Die partiellen Ableitungen lauten
				\begin{align*}
					\frac{\partial f_1}{\partial x}&=\frac{\partial}{\partial x}\cos(x+2y)=-\sin(x+2y)\\
					\frac{\partial f_1}{\partial y}&=\frac{\partial}{\partial y}\cos(x+2y)=-2\sin(x+2y)\\
					\frac{\partial f_2}{\partial x}&=\frac{\partial}{\partial x}\sin(xy)=y\cos(xy)\\
					\frac{\partial f_2}{\partial y}&=\frac{\partial}{\partial y}\sin(xy)=x\cos(xy)
				\end{align*}
		\end{itemize}
	\end{block}
\end{frame}

\begin{frame}\frametitle{\mytitle}
	\begin{block}{Beispiel (fortgesetzt)}
		\begin{itemize}
			\item Daher erhalten wir die Jacobimatrix
				\begin{align*}
					Df(x,y)&=\begin{pmatrix}
						-\sin(x+2y)&-2\sin(x+2y)\\y\cos(xy)&x\cos(xy)
					\end{pmatrix}
				\end{align*}
		\end{itemize}
	\end{block}
\end{frame}

\begin{frame}\frametitle{\mytitle}
	\begin{block}{Zusammenfassung}
		\begin{itemize}
			\item Die partielle Ableitung einer reellwertigen Funktion behandelt die Variablen, nach denen nicht abgeleitet wird, als Konstanten.
			\item Die Jacobimatrix fa\ss t die partiellen Ableitungen der einzelnen Komponenten einer vektorwertigen Funktion zusammen.
		\end{itemize}
	\end{block}
\end{frame}
\end{document}
