\documentclass{beamer}
\usepackage{amsmath,graphics}
\usepackage{amssymb}

\usetheme{default}
\usepackage{xcolor}

\definecolor{solarizedBase03}{HTML}{002B36}
\definecolor{solarizedBase02}{HTML}{073642}
\definecolor{solarizedBase01}{HTML}{586e75}
\definecolor{solarizedBase00}{HTML}{657b83}
\definecolor{solarizedBase0}{HTML}{839496}
\definecolor{solarizedBase1}{HTML}{93a1a1}
\definecolor{solarizedBase2}{HTML}{EEE8D5}
\definecolor{solarizedBase3}{HTML}{FDF6E3}
\definecolor{solarizedYellow}{HTML}{B58900}
\definecolor{solarizedOrange}{HTML}{CB4B16}
\definecolor{solarizedRed}{HTML}{DC322F}
\definecolor{solarizedMagenta}{HTML}{D33682}
\definecolor{solarizedViolet}{HTML}{6C71C4}
%\definecolor{solarizedBlue}{HTML}{268BD2}
\definecolor{solarizedBlue}{HTML}{134676}
\definecolor{solarizedCyan}{HTML}{2AA198}
\definecolor{solarizedGreen}{HTML}{859900}
\definecolor{myBlue}{HTML}{162DB0}%{261CA4}
\setbeamercolor*{item}{fg=myBlue}
\setbeamercolor{normal text}{fg=solarizedBase03, bg=solarizedBase3}
\setbeamercolor{alerted text}{fg=myBlue}
\setbeamercolor{example text}{fg=myBlue, bg=solarizedBase3}
\setbeamercolor*{frametitle}{fg=solarizedRed}
\setbeamercolor*{title}{fg=solarizedRed}
\setbeamercolor{block title}{fg=myBlue, bg=solarizedBase3}
\setbeameroption{hide notes}
\setbeamertemplate{note page}[plain]
\beamertemplatenavigationsymbolsempty
\usefonttheme{professionalfonts}
\usefonttheme{serif}

\usepackage{fourier}

\def\vec#1{\mathchoice{\mbox{\boldmath$\displaystyle#1$}}
{\mbox{\boldmath$\textstyle#1$}}
{\mbox{\boldmath$\scriptstyle#1$}}
{\mbox{\boldmath$\scriptscriptstyle#1$}}}
\definecolor{OwnGrey}{rgb}{0.560,0.000,0.000} % #999999
\definecolor{OwnBlue}{rgb}{0.121,0.398,0.711} % #1f64b0
\definecolor{red4}{rgb}{0.5,0,0}
\definecolor{blue4}{rgb}{0,0,0.5}
\definecolor{Blue}{rgb}{0,0,0.66}
\definecolor{LightBlue}{rgb}{0.9,0.9,1}
\definecolor{Green}{rgb}{0,0.5,0}
\definecolor{LightGreen}{rgb}{0.9,1,0.9}
\definecolor{Red}{rgb}{0.9,0,0}
\definecolor{LightRed}{rgb}{1,0.9,0.9}
\definecolor{White}{gray}{1}
\definecolor{Black}{gray}{0}
\definecolor{LightGray}{gray}{0.8}
\definecolor{Orange}{rgb}{0.1,0.2,1}
\setbeamerfont{sidebar right}{size=\scriptsize}
\setbeamercolor{sidebar right}{fg=Black}

\renewcommand{\emph}[1]{{\textcolor{solarizedRed}{\itshape #1}}}

\newcommand\nan{{\tt nan}}
\newcommand\tay{T}
\newcommand\dd{\mathrm d}
\newcommand\eul{\mathrm e}

\newcommand\cA{\mathcal A}
\newcommand\cB{\mathcal B}
\newcommand\cC{\mathcal C}
\newcommand\cD{\mathcal D}
\newcommand\cE{\mathcal E}
\newcommand\cF{\mathcal F}
\newcommand\cG{\mathcal G}
\newcommand\cH{\mathcal H}
\newcommand\cI{\mathcal I}
\newcommand\cJ{\mathcal J}
\newcommand\cK{\mathcal K}
\newcommand\cL{\mathcal L}
\newcommand\cM{\mathcal M}
\newcommand\cN{\mathcal N}
\newcommand\cO{\mathcal O}
\newcommand\cP{\mathcal P}
\newcommand\cQ{\mathcal Q}
\newcommand\cR{\mathcal R}
\newcommand\cS{\mathcal S}
\newcommand\cT{\mathcal T}
\newcommand\cU{\mathcal U}
\newcommand\cV{\mathcal V}
\newcommand\cW{\mathcal W}
\newcommand\cX{\mathcal X}
\newcommand\cY{\mathcal Y}
\newcommand\cZ{\mathcal Z}

\newcommand\fA{\mathfrak A}
\newcommand\fB{\mathfrak B}
\newcommand\fC{\mathfrak C}
\newcommand\fD{\mathfrak D}
\newcommand\fE{\mathfrak E}
\newcommand\fF{\mathfrak F}
\newcommand\fG{\mathfrak G}
\newcommand\fH{\mathfrak H}
\newcommand\fI{\mathfrak I}
\newcommand\fJ{\mathfrak J}
\newcommand\fK{\mathfrak K}
\newcommand\fL{\mathfrak L}
\newcommand\fM{\mathfrak M}
\newcommand\fN{\mathfrak N}
\newcommand\fO{\mathfrak O}
\newcommand\fP{\mathfrak P}
\newcommand\fQ{\mathfrak Q}
\newcommand\fR{\mathfrak R}
\newcommand\fS{\mathfrak S}
\newcommand\fT{\mathfrak T}
\newcommand\fU{\mathfrak U}
\newcommand\fV{\mathfrak V}
\newcommand\fW{\mathfrak W}
\newcommand\fX{\mathfrak X}
\newcommand\fY{\mathfrak Y}
\newcommand\fZ{\mathfrak Z}

\newcommand\fa{\mathfrak a}
\newcommand\fb{\mathfrak b}
\newcommand\fc{\mathfrak c}
\newcommand\fd{\mathfrak d}
\newcommand\fe{\mathfrak e}
\newcommand\ff{\mathfrak f}
\newcommand\fg{\mathfrak g}
\newcommand\fh{\mathfrak h}
%\newcommand\fi{\mathfrak i}
\newcommand\fj{\mathfrak j}
\newcommand\fk{\mathfrak k}
\newcommand\fl{\mathfrak l}
\newcommand\fm{\mathfrak m}
\newcommand\fn{\mathfrak n}
\newcommand\fo{\mathfrak o}
\newcommand\fp{\mathfrak p}
\newcommand\fq{\mathfrak q}
\newcommand\fr{\mathfrak r}
\newcommand\fs{\mathfrak s}
\newcommand\ft{\mathfrak t}
\newcommand\fu{\mathfrak u}
\newcommand\fv{\mathfrak v}
\newcommand\fw{\mathfrak w}
\newcommand\fx{\mathfrak x}
\newcommand\fy{\mathfrak y}
\newcommand\fz{\mathfrak z}

\newcommand\vA{\vec A}
\newcommand\vB{\vec B}
\newcommand\vC{\vec C}
\newcommand\vD{\vec D}
\newcommand\vE{\vec E}
\newcommand\vF{\vec F}
\newcommand\vG{\vec G}
\newcommand\vH{\vec H}
\newcommand\vI{\vec I}
\newcommand\vJ{\vec J}
\newcommand\vK{\vec K}
\newcommand\vL{\vec L}
\newcommand\vM{\vec M}
\newcommand\vN{\vec N}
\newcommand\vO{\vec O}
\newcommand\vP{\vec P}
\newcommand\vQ{\vec Q}
\newcommand\vR{\vec R}
\newcommand\vS{\vec S}
\newcommand\vT{\vec T}
\newcommand\vU{\vec U}
\newcommand\vV{\vec V}
\newcommand\vW{\vec W}
\newcommand\vX{\vec X}
\newcommand\vY{\vec Y}
\newcommand\vZ{\vec Z}

\newcommand\va{\vec a}
\newcommand\vb{\vec b}
\newcommand\vc{\vec c}
\newcommand\vd{\vec d}
\newcommand\ve{\vec e}
\newcommand\vf{\vec f}
\newcommand\vg{\vec g}
\newcommand\vh{\vec h}
\newcommand\vi{\vec i}
\newcommand\vj{\vec j}
\newcommand\vk{\vec k}
\newcommand\vl{\vec l}
\newcommand\vm{\vec m}
\newcommand\vn{\vec n}
\newcommand\vo{\vec o}
\newcommand\vp{\vec p}
\newcommand\vq{\vec q}
\newcommand\vr{\vec r}
\newcommand\vs{\vec s}
\newcommand\vt{\vec t}
\newcommand\vu{\vec u}
\newcommand\vv{\vec v}
\newcommand\vw{\vec w}
\newcommand\vx{\vec x}
\newcommand\vy{\vec y}
\newcommand\vz{\vec z}

\renewcommand\AA{\mathbb A}
\newcommand\NN{\mathbb N}
\newcommand\ZZ{\mathbb Z}
\newcommand\PP{\mathbb P}
\newcommand\QQ{\mathbb Q}
\newcommand\RR{\mathbb R}
\newcommand\RRpos{\mathbb R_{\geq0}}
\renewcommand\SS{\mathbb S}
\newcommand\CC{\mathbb C}

\newcommand{\ord}{\mathrm{ord}}
\newcommand{\id}{\mathrm{id}}
\newcommand{\pr}{\mathrm{P}}
\newcommand{\Vol}{\mathrm{vol}}
\newcommand\norm[1]{\left\|{#1}\right\|} 
\newcommand\sign{\mathrm{sign}}
\newcommand{\eps}{\varepsilon}
\newcommand{\abs}[1]{\left|#1\right|}
\newcommand\bc[1]{\left({#1}\right)} 
\newcommand\cbc[1]{\left\{{#1}\right\}} 
\newcommand\bcfr[2]{\bc{\frac{#1}{#2}}} 
\newcommand{\bck}[1]{\left\langle{#1}\right\rangle} 
\newcommand\brk[1]{\left\lbrack{#1}\right\rbrack} 
\newcommand\scal[2]{\bck{{#1},{#2}}} 
\newcommand{\vecone}{\mathbb{1}}
\newcommand{\tensor}{\otimes}
\newcommand{\diag}{\mathrm{diag}}
\newcommand{\ggt}{\mathrm{ggT}}
\newcommand{\kgv}{\mathrm{kgV}}
\newcommand{\trans}{\top}

\newcommand{\Karonski}{Karo\'nski}
\newcommand{\Erdos}{Erd\H{o}s}
\newcommand{\Renyi}{R\'enyi}
\newcommand{\Lovasz}{Lov\'asz}
\newcommand{\Juhasz}{Juh\'asz}
\newcommand{\Bollobas}{Bollob\'as}
\newcommand{\Furedi}{F\"uredi}
\newcommand{\Komlos}{Koml\'os}
\newcommand{\Luczak}{\L uczak}
\newcommand{\Kucera}{Ku\v{c}era}
\newcommand{\Szemeredi}{Szemer\'edi}

\renewcommand{\ae}{\"a}
\renewcommand{\oe}{\"o}
\newcommand{\ue}{\"u}
\newcommand{\Ae}{\"A}
\newcommand{\Oe}{\"O}
\newcommand{\Ue}{\"U}

\newcommand{\im}{\mathrm{im}}
\newcommand{\rrk}{\mathrm{zrg}}
\newcommand{\crk}{\mathrm{srg}}
\newcommand{\rk}{\mathrm{rg}}
\newcommand{\GL}{\mathrm{GL}}
\newcommand{\SL}{\mathrm{SL}}
\newcommand{\SO}{\mathrm{SO}}
\newcommand{\nul}{\mathrm{nul}}
\newcommand{\eig}{\mathrm{eig}}

\newcommand{\mytitle}{Flie\ss kommazahlen}

\title[Annuma]{\mytitle}
\author[Amin Coja-Oghlan]{Amin Coja-Oghlan}
\institute[Frankfurt]{JWGUFFM}
\date{}

\begin{document}

\frame[plain]{\titlepage}

\begin{frame}\frametitle{\mytitle}
	\begin{block}{Worum geht es?}
		\begin{itemize}
			\item In der Numerik geht es, allgemein gesprochen, um das rechnerische L\"osen analytischer Probleme.
			\item Insbesondere geht es um die L\"osung auf dem Computer.
			\item Dabei tritt das Problem der Zahldarstellung auf.
			\item Naturgem\ae\ss\ k\oe nnen wir praktisch nicht mit unbegrenzter Genauigkeit rechnen.
			\item Daher m\ue ssen wir Approximationsfehler kontrollieren lernen.
		\end{itemize}
	\end{block}
\end{frame}

\begin{frame}\frametitle{\mytitle}
	\begin{block}{Das Dezimalsystem}
		\begin{itemize}
			\item Aus der Schule kennen Sie bereits das \alert{Dezimalsystem}, in dem Zahlen zur Basis $10$ dargestellt werden.
			\item \alert{Beispiele:}
				\begin{align*}
					107&&1.07&&-10.7&&-0.0107
				\end{align*}
			\item Das Dezimalsystem ist ein \alert{Stellenwertsystem}, d.h.\ der Wert einer Ziffer h\ae ngt von ihrer Position ab.
			\item Der \alert{Dezimalpunkt} ist der ``Nullpunkt'' f\"ur die Stellenwerte.
			\item Au\ss erdem hat jede Zahl ein \alert{Vorzeichen}.
		\end{itemize}
	\end{block}
\end{frame}

\begin{frame}\frametitle{\mytitle}
	\begin{block}{Das Dezimalsystem (fortgesetzt)}
		\begin{itemize}
			\item Viele Zahlen erlauben eine endliche Dezimaldarstellung
				\begin{align*}
					\pm\sum_{i=0}^ka_i10^{k-\ell}
				\end{align*}
				mit $a_0,\ldots,a_k\in\{0,1,2,3,4,5,6,7,8,9\}$ und $\ell\in\NN$.
			\item \alert{Beispiel:} mit $k=2$, $\ell=-1$, $a_0=7$, $a_1=0$, $a_2=1$ erhalten wir
				\begin{align*}
					-10.7&=-\bc{7\cdot10^{-1}+0\cdot10^0+1\cdot10^1}
				\end{align*}
		\end{itemize}
	\end{block}
\end{frame}

\begin{frame}\frametitle{\mytitle}
	\begin{block}{Das Dezimalsystem (fortgesetzt)}
		\begin{itemize}
			\item Andere Zahlen ben\"otigen eine unendliche Darstellung.
			\item Eine endliche oder \alert{periodische Darstellung} ist genau dann m\"ogich, wenn es sich um eine rationale Zahl handelt.
			\item \alert{Beispiel:} $\frac{3}{8}=0.375$.
			\item \alert{Beispiel:} $\frac{1}{3}=0.\bar 3$.
			\item Allgemeine reelle Zahlen erlauben keine endliche Codierung, sondern lassen sich nur als unendiche Ziffernfolgen darstellen.
			\item \alert{Beispiel:} $\pi=3.141592\cdots$.
		\end{itemize}
	\end{block}
\end{frame}

\begin{frame}\frametitle{\mytitle}
	\begin{block}{Allgemeine Stellenwertsysteme}
		\begin{itemize}
			\item Anstelle der Basis $10$ des Dezimalsystems kann man jede andere Basis $1<b\in\NN$ verwenden.
			\item Statt der Dezimaldarstellung erhalten wir dann die \emph{$b$-adische Darstellung} 
				\begin{align*}
					\pm\sum_{i=0}^ka_ib^{k-\ell}
				\end{align*}
				mit $a_0,\ldots,a_k\in\{0,1,\ldots,b-1\}$, $k,\ell\geq0$.
			\item H\"aufige Basen sind $b=12$ und $b=60$.
			\item In der Informatik finden $b=2$, $b=8$ und $b=16$ Verwendung.
		\end{itemize}
	\end{block}
\end{frame}

\begin{frame}\frametitle{\mytitle}
	\begin{block}{Beispiel: verschiedene Darstellungen von 250}
		\begin{itemize}
			\item \alert{Basis $b=2$:} $11111010$
			\item \alert{Basis $b=8$:} $372$
			\item \alert{Basis $b=12$:} $18A$
			\item \alert{Basis $b=16$:} $FA$
		\end{itemize}
	\end{block}
\end{frame}

\begin{frame}\frametitle{\mytitle}
	\begin{block}{Beispiel: verschiedene Darstellungen von $-2.5$}
		\begin{itemize}
			\item \alert{Basis $b=2$:} $-10.1$
			\item \alert{Basis $b=8$:} $-2.4$
			\item \alert{Basis $b=12$:} $-2.6$
			\item \alert{Basis $b=16$:} $-2.8$
		\end{itemize}
	\end{block}
\end{frame}

\begin{frame}\frametitle{\mytitle}
	\begin{block}{Flie\ss kommazahlen: Motivation}
\begin{itemize}
\item Der Effizienz halber m\"ochten wir im Computer Zahlen in Speicherbereichen konstanter Gr\oe\ss e ablegen.
	\item Um sowohl sehr gro\ss e als auch sehr kleine gebrochene Zahlen abspeichern zu k\oe nnen, merken wir uns dabei die Position des Punktes.
	\item Vor Rechenoperationen m\ue ssen die Punkte geeignet verschoben werden.
	\item Au\ss erdem speichern wir in einem Bit das Vorzeichen ab.
\end{itemize}
	\end{block}
\end{frame}

\begin{frame}\frametitle{\mytitle}
	\begin{block}{Definition (Flie\ss kommazahlen)}
Sei $\cF(b,p,q)$ die Menge aller Zahlen der Form
	$$s\cdot f\cdot b^{e-q}$$
wobei
	\begin{itemize}
	\item $b\geq2$ die \emph{Basis} ist,
	\item $q\in\NN_0$ der \emph{Exzess} hei\ss t,
	\item $e\in\{0,1,\ldots,2q\}$ der \emph{Exponent} hei\ss t,
	\item $f$, der \emph{gebrochene Teil}, rational ist, so da\ss\ %$-b^p<b^pf<b^p$ f\"ur eine feste Zahl $p>0$,
		$b^pf\geq0$ ganz ist und die \alert{Normalisierungsbedingung}
			\begin{align*}
				e=f=0\quad\mbox{ oder }\quad\frac1b\leq|f|<1
			\end{align*}
erf\ue llt ist,
	\item $s\in\{\pm1\}$.
	\end{itemize}

	\end{block}
\end{frame}

\begin{frame}\frametitle{\mytitle}
	\begin{block}{Definition (Flie\ss kommazahlen, Erl\ae uterungen)}
	\begin{itemize}
	\item $p$ bestimmt die maximale Anzahl Ziffern von $f$.
		\item Auch Gr\"o\ss e des Exponenten ist beschr\"ankt.
		\item Die Normalisierungsbedingung stellt sicher, da\ss\ die ``f\"uhrende Ziffer'' von $f$ nicht $0$ ist, es sei denn, $f$ ist Null.
		\item Sofern $b,p,q$ bekannt sind, ist die Flie\ss kommazahl durch das Paar $(s,e,f)$ beschrieben.
		\item In dieser Form wird sie auch im Rechner abgespeichert.
	\end{itemize}
	\end{block}
\end{frame}

\begin{frame}\frametitle{\mytitle}
	\begin{block}{Beispiel: $b=10,p=2,q=3$}
	\begin{itemize}
		\item $(+1,5,0.19)$ beschreibt $+1\cdot0.19\cdot10^{5-3}=19$
		\item $(-1,0,0.19)$ beschreibt $-1\cdot0.19\cdot10^{0-3}=-0.00019$
		\item $(\pm1,0,0)$ beschreibt $\pm1\cdot0\cdot10^{0-3}=0$
		\item $(\pm1,0,1.9)$ verletzt die Normalisierungsbedingung
		\item $(\pm1,0,0.01)$ verletzt die Normalisierungsbedingung
		\item $(\pm1,1,0)$ verletzt die Normalisierungsbedingung
	\end{itemize}
	\end{block}
\end{frame}

\begin{frame}\frametitle{\mytitle}
	\begin{block}{Rechnen mit Flie\ss kommazahlen}
		\begin{itemize}
			\item Die Details sind kompliziert und standard- und maschinenabh\"angig.
			\item In der Regel kommt \alert{nicht} das ``algebraische'' Ergebnis heraus.
			\item Die Menge $\cF(b,p,q)$ ist \alert{nicht} unter Addition und Multiplikation abgeschlossen.
			\item Das Ergebnis einer Operation kann undefiniert sein (``\nan'').
			\item Wir verwenden daher die Symbole $\oplus$, $\ominus$, $\odot$, $\oslash$.
			\item Zwischenzeitlich wird mit h\oe herer Genauigkeit gerechnet.
		\end{itemize}
	\end{block}
\end{frame}

\begin{frame}\frametitle{\mytitle}
	\begin{block}{Addition}
		{\em Eingabe:} Flie\ss kommazahlen $x=(s_x,e_x,f_x)$, $y=(s_y,e_y,f_y)$.\\
		{\em Ausgabe:} entwder eine Flie\ss kommazahl $z=(s_z,e_z,f_z)$ oder \nan.
		\begin{enumerate}
			\item Falls $e_x<e_y$, vertausche $x,y$.
			\item Setze $e_z=e_x$.
			\item Falls $e_x-e_y\geq p+2$, setze $f_z= f_x$ und $s_z=s_x$ und halte.
			\item Bestimme $f_y b^{e_y-e_x}$ \hfill\alert{\# $e_x-e_y$ Positionen nach rechts}
			\item Berechne $\xi= s_xf_x+s_yf_y b^{e_y-e_x}$.
			\item Setze $f_z=|\xi|$ und $s_z=\sign(\xi)$.
			\item Normalisiere $z$.
		\end{enumerate}
	\end{block}
\end{frame}

\begin{frame}\frametitle{\mytitle}
	\begin{block}{Normalisierung}
		{\em Eingabe:} $x=(s,e,f)$.\\
		{\em Ausgabe:} eine Flie\ss kommazahl $x'=(s',e',f')$ oder \nan.
		\begin{enumerate}
			\item Falls $f\geq1$, gehe zu Schritt 4.
			\item Falls $f=0$ gib $(s,0,0)$ aus und halte.
			\item Solange $f<1$, multipliziere $f$ mit $b$ und vermindere $e$ um $1$.
			\item Dividiere $f$ durch $b$ und  erh\"ohe $e$ um $1$.
			\item Runde $f$ auf $p$ Stellen; falls danach $f=1$, geh nach (3).
			\item Falls $e>2q$ oder $e<0$, gib \nan\ aus und halte.
			\item Sonst gib $(s,e,f)$ aus.
		\end{enumerate}
	\end{block}
\end{frame}

\begin{frame}\frametitle{\mytitle}
	\begin{block}{Beispiel: $b=10,p=2,q=3$}
	\begin{itemize}
		\item Wir addieren
			\begin{align*}
				19=(s_x,e_x,f_x)=(+1,5,0.19)\oplus(s_y,e_y,f_y)=(-1,4,0.97)=-9.7\enspace.
			\end{align*}
		\item Setze $e_z=e_x=5$ und berechne
			\begin{align*}
				\xi=s_xf_x+s_yf_yb^{e_y-e_x}=0.19-0.97\cdot10^{4-5}=0.19-0.097=0.093
			\end{align*}
		\item Jetzt ist $(+1,5,0.093)$ zu normalisieren.
		\item Das Ergebnis ist $(+1,4,0.93)$.
	\end{itemize}
	\end{block}
\end{frame}

\begin{frame}\frametitle{\mytitle}
	\begin{block}{Beispiel: $b=10,p=2,q=3$}
		\begin{itemize}
			\item Wir addieren
				\begin{align*}
					19=(s_x,e_x,f_x)=(+1,5,0.19)\oplus(s_y,e_y,f_y)=(+1,4,0.12)=1.2\enspace.
				\end{align*}
			\item Setze $e_z=e_x=5$ und berechne
				\begin{align*}
					\xi=s_xf_x+s_yf_yb^{e_y-e_x}=0.19+0.12\cdot10^{4-5}=0.19-0.012=0.202
				\end{align*}
			\item Jetzt ist $(+1,5,0.202)$ zu normalisieren.
			\item Das Ergebnis ist $(+1,5,0.20)$.
		\end{itemize}
	\end{block}
\end{frame}

\begin{frame}\frametitle{\mytitle}
	\begin{block}{Beispiel: $b=10,p=2,q=3$}
		\begin{itemize}
			\item Wir addieren
				\begin{align*}
					19=(s_x,e_x,f_x)=(+1,5,0.19)\oplus(s_y,e_y,f_y)=(+1,1,0.99)=0.0099\enspace.
				\end{align*}
			\item Weil $4=e_x-e_y\geq p+2=4$, ist das Ergebnis einfach 
				\begin{align*}
					19=(s_x,e_x,f_x)
				\end{align*}
		\end{itemize}
	\end{block}
\end{frame}

\begin{frame}\frametitle{\mytitle}
	\begin{block}{Multiplikation}
		{\em Eingabe:} Flie\ss kommazahlen $x=(s_x,e_x,f_x)$, $y=(s_y,e_y,f_y)$.\\
		{\em Ausgabe:} eine Flie\ss kommazahl $z=(s_z,e_z,f_z)$ oder \nan.
		\begin{enumerate}
			\item Setze $e_z= e_x+e_y-q$, $f_z=  f_x\cdot f_y$ und $s_z= s_x\cdot s_y$.
			\item Normalisiere $z$.
		\end{enumerate}
	\end{block}
\end{frame}

\begin{frame}\frametitle{\mytitle}
	\begin{block}{Beispiel: $b=10,p=2,q=3$}
		\begin{itemize}
			\item Wir multiplizieren
				\begin{align*}
					19=(s_x,e_x,f_x)=(+1,5,0.19)\odot(s_y,e_y,f_y)=(+1,4,0.11)=1.1\enspace.
				\end{align*}
			\item Setze $e_z=e_x+e_y-q=5+4-3=6$ und berechne
				\begin{align*}
					f_x\cdot f_y=0.19\cdot0.11=0.0209
				\end{align*}
			\item Jetzt ist $(+1,6,0.0209)$ zu normalisieren.
			\item Das Ergebnis ist $(+1,5,0.21)=21$.
		\end{itemize}
	\end{block}
\end{frame}

\begin{frame}\frametitle{\mytitle}
	\begin{block}{Division}
		{\em Eingabe:} Flie\ss kommazahlen $x=(s_x,e_x,f_x)$, $y=(s_y,e_y,f_y)$.\\
		{\em Ausgabe:} eine Flie\ss kommazahl $z=(s_z,e_z,f_z)$ oder \nan.
		\begin{enumerate}
			\item Teste, ob $y=0$; falls ja, gib \nan\ aus.
			\item Setze $e_z= e_x-e_y+q+1$, $s_z= s_xs_y$ und $$f_z= \frac{f_x}{b\cdot f_y}\enspace.$$ 
			\item Normalisiere $z$.
		\end{enumerate}
	\end{block}
\end{frame}

\begin{frame}\frametitle{\mytitle}
	\begin{block}{Beispiel: $b=10,p=2,q=3$}
		\begin{itemize}
			\item Wir dividieren
				\begin{align*}
					19=(s_x,e_x,f_x)=(+1,5,0.19)\oslash(s_y,e_y,f_y)=(+1,4,0.11)=1.1\enspace.
				\end{align*}
			\item Setze $e_z=e_x-e_y+q+1=5-4+4=5$, $s_z=1$ und berechne
				\begin{align*}
					f_z= \frac{f_x}{b\cdot f_y}=\frac{0.19}{10\cdot0.11}=\frac{0.19}{1.1}=0.1\overline{72}
				\end{align*}
			\item Jetzt ist $(+1,6,0.1\overline{72})$ zu normalisieren.
			\item Das Ergebnis ist $(+1,5,0.17)=17$.
		\end{itemize}
	\end{block}
\end{frame}

\begin{frame}\frametitle{\mytitle}
	\begin{block}{Zusammenfassung}
		\begin{itemize}
			\item Flie\ss kommazahlen sind ein Kompromi\ss\ aus Effizienz und Genauigkeit.
			\item Die Details von Flie\ss kommaoperationen sind kompliziert.
			\item Algebraische Rechenregeln f\ue r reelle Zahlen wie etwa das Assoziativgesetz gelten \alert{nicht} f\ue r Flie\ss kommazahlen!
			\item Wir brauchen daher einen pragmatischen Zugang zu Entwurf und Analyse von numerischen Verfahren.
		\end{itemize}
	\end{block}
\end{frame}
\end{document}
