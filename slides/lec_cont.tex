\documentclass{beamer}
\usepackage{amsmath,graphics}
\usepackage{amssymb}

\usetheme{default}
\usepackage{xcolor}

\definecolor{solarizedBase03}{HTML}{002B36}
\definecolor{solarizedBase02}{HTML}{073642}
\definecolor{solarizedBase01}{HTML}{586e75}
\definecolor{solarizedBase00}{HTML}{657b83}
\definecolor{solarizedBase0}{HTML}{839496}
\definecolor{solarizedBase1}{HTML}{93a1a1}
\definecolor{solarizedBase2}{HTML}{EEE8D5}
\definecolor{solarizedBase3}{HTML}{FDF6E3}
\definecolor{solarizedYellow}{HTML}{B58900}
\definecolor{solarizedOrange}{HTML}{CB4B16}
\definecolor{solarizedRed}{HTML}{DC322F}
\definecolor{solarizedMagenta}{HTML}{D33682}
\definecolor{solarizedViolet}{HTML}{6C71C4}
%\definecolor{solarizedBlue}{HTML}{268BD2}
\definecolor{solarizedBlue}{HTML}{134676}
\definecolor{solarizedCyan}{HTML}{2AA198}
\definecolor{solarizedGreen}{HTML}{859900}
\definecolor{myBlue}{HTML}{162DB0}%{261CA4}
\setbeamercolor*{item}{fg=myBlue}
\setbeamercolor{normal text}{fg=solarizedBase03, bg=solarizedBase3}
\setbeamercolor{alerted text}{fg=myBlue}
\setbeamercolor{example text}{fg=myBlue, bg=solarizedBase3}
\setbeamercolor*{frametitle}{fg=solarizedRed}
\setbeamercolor*{title}{fg=solarizedRed}
\setbeamercolor{block title}{fg=myBlue, bg=solarizedBase3}
\setbeameroption{hide notes}
\setbeamertemplate{note page}[plain]
\beamertemplatenavigationsymbolsempty
\usefonttheme{professionalfonts}
\usefonttheme{serif}

\usepackage{fourier}

\def\vec#1{\mathchoice{\mbox{\boldmath$\displaystyle#1$}}
{\mbox{\boldmath$\textstyle#1$}}
{\mbox{\boldmath$\scriptstyle#1$}}
{\mbox{\boldmath$\scriptscriptstyle#1$}}}
\definecolor{OwnGrey}{rgb}{0.560,0.000,0.000} % #999999
\definecolor{OwnBlue}{rgb}{0.121,0.398,0.711} % #1f64b0
\definecolor{red4}{rgb}{0.5,0,0}
\definecolor{blue4}{rgb}{0,0,0.5}
\definecolor{Blue}{rgb}{0,0,0.66}
\definecolor{LightBlue}{rgb}{0.9,0.9,1}
\definecolor{Green}{rgb}{0,0.5,0}
\definecolor{LightGreen}{rgb}{0.9,1,0.9}
\definecolor{Red}{rgb}{0.9,0,0}
\definecolor{LightRed}{rgb}{1,0.9,0.9}
\definecolor{White}{gray}{1}
\definecolor{Black}{gray}{0}
\definecolor{LightGray}{gray}{0.8}
\definecolor{Orange}{rgb}{0.1,0.2,1}
\setbeamerfont{sidebar right}{size=\scriptsize}
\setbeamercolor{sidebar right}{fg=Black}

\renewcommand{\emph}[1]{{\textcolor{solarizedRed}{\itshape #1}}}

\newcommand\cA{\mathcal A}
\newcommand\cB{\mathcal B}
\newcommand\cC{\mathcal C}
\newcommand\cD{\mathcal D}
\newcommand\cE{\mathcal E}
\newcommand\cF{\mathcal F}
\newcommand\cG{\mathcal G}
\newcommand\cH{\mathcal H}
\newcommand\cI{\mathcal I}
\newcommand\cJ{\mathcal J}
\newcommand\cK{\mathcal K}
\newcommand\cL{\mathcal L}
\newcommand\cM{\mathcal M}
\newcommand\cN{\mathcal N}
\newcommand\cO{\mathcal O}
\newcommand\cP{\mathcal P}
\newcommand\cQ{\mathcal Q}
\newcommand\cR{\mathcal R}
\newcommand\cS{\mathcal S}
\newcommand\cT{\mathcal T}
\newcommand\cU{\mathcal U}
\newcommand\cV{\mathcal V}
\newcommand\cW{\mathcal W}
\newcommand\cX{\mathcal X}
\newcommand\cY{\mathcal Y}
\newcommand\cZ{\mathcal Z}

\newcommand\fA{\mathfrak A}
\newcommand\fB{\mathfrak B}
\newcommand\fC{\mathfrak C}
\newcommand\fD{\mathfrak D}
\newcommand\fE{\mathfrak E}
\newcommand\fF{\mathfrak F}
\newcommand\fG{\mathfrak G}
\newcommand\fH{\mathfrak H}
\newcommand\fI{\mathfrak I}
\newcommand\fJ{\mathfrak J}
\newcommand\fK{\mathfrak K}
\newcommand\fL{\mathfrak L}
\newcommand\fM{\mathfrak M}
\newcommand\fN{\mathfrak N}
\newcommand\fO{\mathfrak O}
\newcommand\fP{\mathfrak P}
\newcommand\fQ{\mathfrak Q}
\newcommand\fR{\mathfrak R}
\newcommand\fS{\mathfrak S}
\newcommand\fT{\mathfrak T}
\newcommand\fU{\mathfrak U}
\newcommand\fV{\mathfrak V}
\newcommand\fW{\mathfrak W}
\newcommand\fX{\mathfrak X}
\newcommand\fY{\mathfrak Y}
\newcommand\fZ{\mathfrak Z}

\newcommand\fa{\mathfrak a}
\newcommand\fb{\mathfrak b}
\newcommand\fc{\mathfrak c}
\newcommand\fd{\mathfrak d}
\newcommand\fe{\mathfrak e}
\newcommand\ff{\mathfrak f}
\newcommand\fg{\mathfrak g}
\newcommand\fh{\mathfrak h}
%\newcommand\fi{\mathfrak i}
\newcommand\fj{\mathfrak j}
\newcommand\fk{\mathfrak k}
\newcommand\fl{\mathfrak l}
\newcommand\fm{\mathfrak m}
\newcommand\fn{\mathfrak n}
\newcommand\fo{\mathfrak o}
\newcommand\fp{\mathfrak p}
\newcommand\fq{\mathfrak q}
\newcommand\fr{\mathfrak r}
\newcommand\fs{\mathfrak s}
\newcommand\ft{\mathfrak t}
\newcommand\fu{\mathfrak u}
\newcommand\fv{\mathfrak v}
\newcommand\fw{\mathfrak w}
\newcommand\fx{\mathfrak x}
\newcommand\fy{\mathfrak y}
\newcommand\fz{\mathfrak z}

\newcommand\vA{\vec A}
\newcommand\vB{\vec B}
\newcommand\vC{\vec C}
\newcommand\vD{\vec D}
\newcommand\vE{\vec E}
\newcommand\vF{\vec F}
\newcommand\vG{\vec G}
\newcommand\vH{\vec H}
\newcommand\vI{\vec I}
\newcommand\vJ{\vec J}
\newcommand\vK{\vec K}
\newcommand\vL{\vec L}
\newcommand\vM{\vec M}
\newcommand\vN{\vec N}
\newcommand\vO{\vec O}
\newcommand\vP{\vec P}
\newcommand\vQ{\vec Q}
\newcommand\vR{\vec R}
\newcommand\vS{\vec S}
\newcommand\vT{\vec T}
\newcommand\vU{\vec U}
\newcommand\vV{\vec V}
\newcommand\vW{\vec W}
\newcommand\vX{\vec X}
\newcommand\vY{\vec Y}
\newcommand\vZ{\vec Z}

\newcommand\va{\vec a}
\newcommand\vb{\vec b}
\newcommand\vc{\vec c}
\newcommand\vd{\vec d}
\newcommand\ve{\vec e}
\newcommand\vf{\vec f}
\newcommand\vg{\vec g}
\newcommand\vh{\vec h}
\newcommand\vi{\vec i}
\newcommand\vj{\vec j}
\newcommand\vk{\vec k}
\newcommand\vl{\vec l}
\newcommand\vm{\vec m}
\newcommand\vn{\vec n}
\newcommand\vo{\vec o}
\newcommand\vp{\vec p}
\newcommand\vq{\vec q}
\newcommand\vr{\vec r}
\newcommand\vs{\vec s}
\newcommand\vt{\vec t}
\newcommand\vu{\vec u}
\newcommand\vv{\vec v}
\newcommand\vw{\vec w}
\newcommand\vx{\vec x}
\newcommand\vy{\vec y}
\newcommand\vz{\vec z}

\renewcommand\AA{\mathbb A}
\newcommand\NN{\mathbb N}
\newcommand\ZZ{\mathbb Z}
\newcommand\PP{\mathbb P}
\newcommand\QQ{\mathbb Q}
\newcommand\RR{\mathbb R}
\newcommand\RRpos{\mathbb R_{\geq0}}
\renewcommand\SS{\mathbb S}
\newcommand\CC{\mathbb C}

\newcommand{\ord}{\mathrm{ord}}
\newcommand{\id}{\mathrm{id}}
\newcommand{\pr}{\mathrm{P}}
\newcommand{\Vol}{\mathrm{vol}}
\newcommand\norm[1]{\left\|{#1}\right\|} 
\newcommand\sign{\mathrm{sign}}
\newcommand{\eps}{\varepsilon}
\newcommand{\abs}[1]{\left|#1\right|}
\newcommand\bc[1]{\left({#1}\right)} 
\newcommand\cbc[1]{\left\{{#1}\right\}} 
\newcommand\bcfr[2]{\bc{\frac{#1}{#2}}} 
\newcommand{\bck}[1]{\left\langle{#1}\right\rangle} 
\newcommand\brk[1]{\left\lbrack{#1}\right\rbrack} 
\newcommand\scal[2]{\bck{{#1},{#2}}} 
\newcommand{\vecone}{\mathbb{1}}
\newcommand{\tensor}{\otimes}
\newcommand{\diag}{\mathrm{diag}}
\newcommand{\ggt}{\mathrm{ggT}}
\newcommand{\kgv}{\mathrm{kgV}}
\newcommand{\trans}{\top}

\newcommand{\Karonski}{Karo\'nski}
\newcommand{\Erdos}{Erd\H{o}s}
\newcommand{\Renyi}{R\'enyi}
\newcommand{\Lovasz}{Lov\'asz}
\newcommand{\Juhasz}{Juh\'asz}
\newcommand{\Bollobas}{Bollob\'as}
\newcommand{\Furedi}{F\"uredi}
\newcommand{\Komlos}{Koml\'os}
\newcommand{\Luczak}{\L uczak}
\newcommand{\Kucera}{Ku\v{c}era}
\newcommand{\Szemeredi}{Szemer\'edi}

\renewcommand{\ae}{\"a}
\renewcommand{\oe}{\"o}
\newcommand{\ue}{\"u}
\newcommand{\Ae}{\"A}
\newcommand{\Oe}{\"O}
\newcommand{\Ue}{\"U}

\newcommand{\im}{\mathrm{im}}
\newcommand{\rrk}{\mathrm{zrg}}
\newcommand{\crk}{\mathrm{srg}}
\newcommand{\rk}{\mathrm{rg}}
\newcommand{\GL}{\mathrm{GL}}
\newcommand{\SL}{\mathrm{SL}}
\newcommand{\SO}{\mathrm{SO}}
\newcommand{\nul}{\mathrm{nul}}
\newcommand{\eig}{\mathrm{eig}}

\newcommand{\mytitle}{Stetigkeit}

\title[Annuma]{\mytitle}
\author[Amin Coja-Oghlan]{Amin Coja-Oghlan}
\institute[Frankfurt]{JWGUFFM}
\date{}

\begin{document}

\frame[plain]{\titlepage}

\begin{frame}\frametitle{\mytitle}
	\begin{block}{Worum geht es?}
		\begin{itemize}
			\item Stetige Funktionen sind anschaulich gesprochen Funktionen, deren Graph man ``in einem Zug'' durchzeichnen kann.
			\item Wir werden die exakte Definition und einige Eigenschaften kennenlernen.
		\end{itemize}
	\end{block}
\end{frame}

\begin{frame}\frametitle{\mytitle}
	\begin{block}{Definition}
		\begin{itemize}
			\item Sei $f:A\to\RR$ eine Funktion und $x\in A$.
			\item $f$ hei\ss t \emph{stetig in $x$}, wenn f\ue r jede Folge $(a_n)_n$ mit $a_n\in A$ und $\lim_{n\to\infty}a_n=x$ gilt:
				\begin{align*}
					\lim_{n\to\infty}f(a_n)&=f(x).
				\end{align*}
			\item {\itshape Anders geschrieben:}
				\begin{align*}
					\lim_{n\to\infty}f(a_n)&=f(\lim_{n\to\infty}a_n).
				\end{align*}
			\item Wir nennen $f$ \emph{stetig}, wenn $f$ stetig in $x$ ist f\ue r alle $x\in A$.
		\end{itemize}
	\end{block}
\end{frame}

\begin{frame}\frametitle{\mytitle}
	\begin{block}{Beispiel}
		\begin{itemize}
			\item F\ue r jede Zahl $c\in\RR$ ist die konstante Funktion $f:\RR\to\RR$, $x\mapsto c$ stetig.
			\item Denn angenommen $\lim_{n\to\infty}a_n=x$.
			\item Dann gilt
				\begin{align*}
					f(x)=c=\lim_{n\to\infty}f(a_n).
				\end{align*}
		\end{itemize}
	\end{block}
\end{frame}

\begin{frame}\frametitle{\mytitle}
	\begin{block}{Beispiel}
		\begin{itemize}
			\item Die Funktion $f:\RR\to\RR$, $x\mapsto x$ ist stetig.
			\item Denn angenommen $\lim_{n\to\infty}a_n=x$.
			\item Dann gilt
				\begin{align*}
					\lim_{n\to\infty}f(a_n)=\lim_{n\to\infty}a_n=x=f(x).
				\end{align*}
		\end{itemize}
	\end{block}
\end{frame}

\begin{frame}\frametitle{\mytitle}
	\begin{block}{Beispiel}
		\begin{itemize}
			\item Die Funktion $f:\RR\to\RR$, $x\mapsto 1/x$ ist stetig.
			\item Denn angenommen $\lim_{n\to\infty}a_n=x$.
			\item Dann gilt
				\begin{align*}
					\lim_{n\to\infty}f(a_n)=\lim_{n\to\infty}1/a_n=1/x=f(x).
				\end{align*}
		\end{itemize}
	\end{block}
\end{frame}

\begin{frame}\frametitle{\mytitle}
	\begin{block}{Proposition}
		Angenommen $f,g:A\to\RR$ sind stetig in $x\in A$.
		\begin{itemize}
			\item Dann ist auch $f+g:A\to\RR$ stetig in $x$.
			\item Ebenso ist $f\cdot g:A\to\RR$ stetig in $x$.
		\end{itemize}
	\end{block}
\end{frame}

\begin{frame}\frametitle{\mytitle}
	\begin{block}{Korollar}
		Jedes Polynom $f:\RR\to\RR$, $x\mapsto\sum_{i=0}^ka_ix^i$ ist stetig.
	\end{block}
\end{frame}

\begin{frame}\frametitle{\mytitle}
	\begin{block}{Korollar}
		Sind $f,g:\RR\to\RR$ Polynome, so ist die rationale Funktion
		\begin{align*}
			\frac{f}{g}:A\to\RR\qquad x\mapsto\frac{f(x)}{g(x)}
		\end{align*}
		stetig auf der Menge
		\begin{align*}
			A=A(g)=\cbc{x\in\RR:g(x)\neq0}.
		\end{align*}
	\end{block}
\end{frame}

\begin{frame}\frametitle{\mytitle}
	\begin{block}{Proposition}
		Sind $f:A\to B$ und $g:B\to\RR$ Funktionen, so da\ss\ $f$ stetig ist in $x\in A$ und $g$ stetig ist in $f(x)\in B$, so ist $g\circ f$ stetig in $x$.
	\end{block}
	\begin{block}{Beweis}
	\begin{itemize}
		\item Angenommen $(a_n)_n$ konvergiert gegen $x$.
		\item Weil $f$ stetig ist, konvergiert $f(a_n)$ gegen $y=f(x)$.
		\item Weil $g$ stetig ist, konvergiert $g(f(a_n))$ gegen $g(y)=g(f(x))$.
	\end{itemize}
	\end{block}
\end{frame}

\begin{frame}\frametitle{\mytitle}
	\begin{block}{Gegenbeispiel}
	\begin{itemize}
	\item Die Funktion
		\begin{align*}
			f:\RR\to\RR&&x\mapsto\begin{cases}-1&\mbox{falls }x<0\\1&\mbox{sonst}\end{cases}
		\end{align*}
		ist unstetig in $x=0$.
	\item Denn die Folge $(a_n)_n$ mit $a_n=(-1)^n/n$ konvergiert zwar gegen $0$, aber die Folge $(f(a_n))_n$ alterniert:
		\begin{align*}
		-1,1,-1,1,-1,1,-1,1,\ldots
		\end{align*}
	\item Diese Folge konvergiert also gar nicht, und ganz sicher nicht gegen $f(0)=1$.
	\end{itemize}	
	\end{block}
\end{frame}

\begin{frame}\frametitle{\mytitle}
	\begin{block}{Gegenbeispiel}
	\begin{itemize}
	\item Die Funktion
		\begin{align*}
			f:\RR\to\RR&&x\mapsto\begin{cases}-1&\mbox{falls }x\in\QQ\\1&\mbox{sonst}\end{cases}
		\end{align*}
		ist unstetig in jedem Punkt $x\in\RR$.
	\item Denn einerseits gibt es zu jedem $x\in\RR$ eine Folge $(q_n)_n$ mit $q_n\in\QQ$, die gegen $x$ konvergiert.
	\item Andererseits gibt es auch eine Folge $(r_n)_n$ mit $r_n\in\RR\setminus\QQ$, die gegen $x$ konvergiert.
	\item Wenn $x\in\QQ$ ist, k\oe nnte man beispielsweise die Folge
		$ r_n=x+\sqrt 2/n $ w\ae hlen.
	\item Wenn $x\in\RR\setminus\QQ$, w\ae hle beispielsweise $r_n=x$ f\ue r alle $n$.
	\end{itemize}	
	\end{block}
\end{frame}

\begin{frame}\frametitle{\mytitle}
	\begin{block}{Gegenbeispiel}
	\begin{itemize}
		\item Die Folge
			\begin{align*}
				a_n&=\begin{cases}
					q_n&\mbox{ falls $n$ ungerade}\\
					r_n&\mbox{ falls $n$ gerade}
				\end{cases}
			\end{align*}
			konvergiert nun gegen $x$.
		\item Aber wiederum ist $f(a_n)$ die alternierende Folge:
			\begin{align*}
			-1,1,-1,1,-1,1,\ldots,
			\end{align*}
			die nicht konvergiert.
	\end{itemize}	
	\end{block}
\end{frame}

\begin{frame}\frametitle{\mytitle}
	\begin{block}{Zwischenwertsatz}
		Angenommen $f:[0,1]\to\RR$ ist eine stetige Funktion mit $f(0)<0$ und $f(1)>0$.
		Dann gibt es eine Zahl $z\in(0,1)$ mit $f(z)=0$.
	\end{block}
	\begin{block}{Beweis}
	\begin{itemize}
	\item Die Menge $S=\cbc{x\in[0,1]:f(x)<0}$ ist beschr\ae nkt.
	\item Nach dem Vollst\ae ndigkeitsaxiom existiert $z=\sup S$.
	\item Weil $f$ stetig ist, gilt $f(z)=0$.
	\end{itemize}
	\end{block}
\end{frame}

\begin{frame}\frametitle{\mytitle}
	\begin{block}{Korollar}
		Angenommen $f:[0,1]\to\RR$ ist eine stetige Funktion.
		Dann gibt es eine Zahl $z\in[0,1]$ mit $$f(z)=\sup\cbc{f(x):x\in[0,1]}.$$
	\end{block}
	\begin{block}{}
	\begin{itemize}
	\item Eine stetige Funktion auf dem Intervall $[0,1]$ mit nimmt also immer ein Maximum an.
	\item Eine solche Funktion ist insbesondere beschr\ae nkt, d.h.\
		$$\cbc{f(x):x\in[0,1]}$$
		ist beschr\ae nkt.
	\item Entsprechendes gilt f\ue r andere Intervalle $[a,b]$ mit $a,b\in\RR$ und f\ue r Minima anstelle von Maxima.
	\end{itemize}
	\end{block}
\end{frame}

\begin{frame}\frametitle{\mytitle}
	\begin{block}{Zusammenfassung}
	\begin{itemize}
		\item Stetige Funktionen kann man in einem Zug durchzeichnen.
		\item Beispiels sind Polynome und rationale Funktionen.
		\item Wir haben einige Gegenbeispiele kennengelernt.
	\end{itemize}	
	\end{block}
\end{frame}

\end{document}
